% Standardinkluderingsfil
%
%  untitled
%
%  Created by David Granqvist on 2008-09-08.
%  Modified by Martin Erola
%

% Set document format/class
\documentclass[a4paper,twoside]{article}

%%%%%%%%%%%%%%%%%%%
% Include packages
%
\usepackage[utf8]{inputenc}   % Use utf-8 encoding for foreign characters
\usepackage[swedish]{babel}   % Support for swedish letters
\usepackage{fullpage}         % Setup for fullpage use
\usepackage{fancyhdr}         % Running Headers and footers
\usepackage{boxedminipage}    % Surround parts of graphics with box
\usepackage{listings}         % Package for including code in the document
\usepackage{ifpdf}            % Recommended way for checking for PDFLaTeX:
\usepackage{tabularx}         % Tabeller med automatisk stretch
% \usepackage[nofancy]{svninfo} % Extract Subversion info about the file
% \usepackage{color}          % Color
% \usepackage{lastpage}       % Total page count

% Graphics
\ifpdf
\usepackage[pdftex]{graphicx}
\else
\usepackage{graphicx}
\fi

%%%%%%%%%%%%%%%%%%%%%%%%%%%%%%%%%%%%%%%%%%%%%%%%%%%%%%%%%%
% Uncomment some of the following if you use the features
%

% Multipart figures
%\usepackage{subfigure}

% More symbols
%\usepackage{amsmath}
%\usepackage{amssymb}
%\usepackage{latexsym}

% If you want to generate a toc for each chapter (use with book)
% \usepackage{minitoc}

%%%%%%%%%%%%%%%%%%%%
% Document settings
%

% Header
\pagestyle{fancy}
% Sätter en marginal mellan header och (ovanstående?) text %
\setlength\headsep{10pt}
% Sätter höjden på headern
\setlength{\headheight}{32pt}

% Sätter styckesinställningar
\setlength\parindent{0pt}
\setlength\parskip{10pt}



\ifpdf
  \DeclareGraphicsExtensions{.pdf, .jpg, .tif, .png}
  \pdfinfo{            
    /Title  (Vision)
    /Author (PUM-grupp 1)
  }
\else
  \DeclareGraphicsExtensions{.eps, .jpg}
\fi

\title{Distribuerad wiki \\ Vision}
\author{PUM-grupp 1}
\date{\today}

\begin{document}

\maketitle

\thispagestyle{empty}
\newpage
\section{Inledning}
En distribuerad wiki är en wiki som istället för att vara en traditonell centraliserad webapplikation ska vara en wiki som distribueras med en versionhanterare som är distruberad. Så varje användare har en lokal kopia av hela wiki:n, med alla förändringar. Alla ändringar som sker på wikin ska distrubueras automatiskt med hjälp av ett p2p protokoll till användarna. 
\section{Positionering}


\subsection{Problem formulering}
\begin{tabular}{|c|m{15 cm}|}
\hline
Problem: & En centraliserad lösning kräver resurser som bandbredd, servar och underhåll. Centraliseringen gör den sårbar för attacker och påtryckningar. \\
\hline
Påverkan: & Ideella projekt måste förlita sig på donationer för att klara av kostnaderna,  donationerna kan komma med en politisk agenda. För företag så kan det vara skillnaden om projektet är lönsamt eller ej.   \\
\hline
Följderna: & Ideella projekt riskerar sitt obereoende om dom tar emot pengar från t.ex. politisk aktör eller någon med en agenda. Företagen inte vågar ta riskerna om kostnaderna är stora. Ett projekt läggsner om det blir stora omkostnader eller dataförluster pågrund av attacker mot projeket. \\
\hline
Lösning: & Med en distribuerad lösning som utnyttjar snabbheten i peer-to-peer så skulle även konstnaderna distribueras till kunderna/medlemmarna. Projekten klarar sig på färre donationer och blir mindre beroende. Den distrubuerade lösningen skulle även minska risken av dataförluster och skapar ett robust system mot attacker och en kraftig ökning av användare. \\
\hline
\end{tabular}

\subsection{Produktens positions formulering} % vet inte om det här är en bra översättning!
\begin{tabular}{|c|m{15 cm}|}
\hline
Målgrupp & Små projektgrupper eller organisationer\\
Tilllämpining 

\end{tabular}
\end{document}
