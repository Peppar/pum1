% Standardinkluderingsfil
%
%  untitled
%
%  Created by David Granqvist on 2008-09-08.
%  Modified by Martin Erola
%

% Set document format/class
\documentclass[a4paper,twoside]{article}

%%%%%%%%%%%%%%%%%%%
% Include packages
%
\usepackage[utf8]{inputenc}   % Use utf-8 encoding for foreign characters
\usepackage[swedish]{babel}   % Support for swedish letters
\usepackage{fullpage}         % Setup for fullpage use
\usepackage{fancyhdr}         % Running Headers and footers
\usepackage{boxedminipage}    % Surround parts of graphics with box
\usepackage{listings}         % Package for including code in the document
\usepackage{ifpdf}            % Recommended way for checking for PDFLaTeX:
\usepackage{tabularx}         % Tabeller med automatisk stretch
% \usepackage[nofancy]{svninfo} % Extract Subversion info about the file
% \usepackage{color}          % Color
% \usepackage{lastpage}       % Total page count

% Graphics
\ifpdf
\usepackage[pdftex]{graphicx}
\else
\usepackage{graphicx}
\fi

%%%%%%%%%%%%%%%%%%%%%%%%%%%%%%%%%%%%%%%%%%%%%%%%%%%%%%%%%%
% Uncomment some of the following if you use the features
%

% Multipart figures
%\usepackage{subfigure}

% More symbols
%\usepackage{amsmath}
%\usepackage{amssymb}
%\usepackage{latexsym}

% If you want to generate a toc for each chapter (use with book)
% \usepackage{minitoc}

%%%%%%%%%%%%%%%%%%%%
% Document settings
%

% Header
\pagestyle{fancy}
% Sätter en marginal mellan header och (ovanstående?) text %
\setlength\headsep{10pt}
% Sätter höjden på headern
\setlength{\headheight}{32pt}

% Sätter styckesinställningar
\setlength\parindent{0pt}
\setlength\parskip{10pt}



\ifpdf
  \DeclareGraphicsExtensions{.pdf, .jpg, .tif, .png}
  \pdfinfo{            
    /Title  (Vision)
    /Author (PUM-grupp 1)
  }
\else
  \DeclareGraphicsExtensions{.eps, .jpg}
\fi

\title{Distribuerad wiki \\ Vision}
\author{PUM-grupp 1}
\date{\today}

\begin{document}

\maketitle

\thispagestyle{empty}
\newpage
\section{Inledning}
En distribuerad wiki är en wiki som istället för att vara en traditionell centraliserad webbapplikation ska vara en wiki som distribueras med en versionhanterare som är distribuerad. Så varje användare har en lokal kopia av hela wikin, med hela historian. Alla ändringar som sker på wikin ska distribueras automatiskt med hjälp av ett peer-to-peer-protokoll till användarna.
\section{Positionering}

\subsection{Problemformulering}
\begin{tabular}{|c|m{15 cm}|}
\hline
Problem: & En centraliserad lösning kräver resurser som bandbredd, servar och underhåll. Centraliseringen gör den sårbar för attacker och påtryckningar. \\
\hline
Påverkan: & Ideella projekt måste förlita sig på donationer för att klara av kostnaderna, donationerna kan komma med en politisk agenda. För företag så kan det vara skillnaden om projektet är lönsamt eller ej.   \\
\hline
Följderna: & Ideella projekt riskerar sitt oberoende om de tar emot pengar från till exempel en politisk aktör eller någon med en agenda. Företagen vågar inte ta riskerna om kostnaderna är stora. Ett projekt läggs ned om det blir stora omkostnader eller dataförluster på grund av attacker mot projektet. \\
\hline
Lösning: & Med en distribuerad lösning som utnyttjar snabbheten och flexibiliteten i peer-to-peer så skulle även kostnaderna distribueras till kunderna/medlemmarna. Projekten klarar sig på färre donationer och blir mindre beroende. Den distribuerade lösningen skulle även minska risken av dataförluster och skapar ett robust system mot attacker och en kraftig ökning av användare. \\
\hline
\end{tabular}

\subsection{Produktens lägesformulering} % vet inte om det här är en bra översättning!
\begin{tabular}{|c|m{15 cm}|}
\hline
Målgrupp & Små projektgrupper eller organisationer\\
\hline
Tillämpning & En robust wiki som är distribuerad på användarna \\
\hline
Namn & Är en distribuerad wiki  \\
\hline 
Fördelar & Distribuerad, tillförlitlig, kostnadeffektiv och snabb \\
\hline
Olikt & Centraliserad lösning som är kostsam och resurskrävande \\
\hline
Vår produkt & Öppnar för ett nytt sätt att samarbeta i molnet \\
\hline
\end{tabular}
\section{Intressenter}
Här samlas information om de intressenter (``stakeholders'') som berörs av projektet och dess utveckling.
\subsection{Sammanfattning av Intressenter}
\begin{tabular}{|c|m{14 cm}|}
\hline
PUM-gruppen & Är ansvariga för utveckla systemet, göra projektnära beslut och implementera funktionerna som tillsammans formuleras av gruppen och kunden. PUM-gruppen bör ha nära kontakt med kunden. \\ \hline
Kunden & VISIARC AB är ansvarig för beställning av systemet. Ska samarbeta med PUM-gruppen för att formulerar krav och funktioner till systemet. \\ \hline
Användargrupp 1 & Användare som ej har tillgång till en traditionell klient-serverlösning. Vi ser att det finns två sorters användare i den här gruppen. De som av ekonomiska skäl väljer bort en serverlösning. Serverar är kostsamma i drift; el, underhåll och lokal. Den andra gruppen är användare utan större datorvana. För en oerfaren datoranvändare kan det vara svårt att få tillgång till en server. \\ \hline
Användargrupp 2 & Användare som väljer en distribuerad lösning på grund av strukturen och robusthet i systemet. Användarna sitter mycket i offlinemiljöer eller i otillförlitliga nätverk. Robusthet blir intressant då användaren ska publicera material som är kontroversiellt eller populärt och som kräver redundans och hög tillgänglighet.\\ \hline
Ny personal & Om det tillkommer utvecklare eller annat folk till gemenskapen så måste dokumenten vara i ordning. Att översätta dokumenten till engelska kan bli intressant i framtiden. All kod och alla kommentarer kommer skrivas på engelska.  \\ \hline
\end{tabular}
\subsection{Användarmiljö}
Wikin ämnar användas till nätverk om 5-10 användare, men om möjligt ska programmet gå att skala upp till fler. Användarna kommer jobba via ett webbgränssnitt eller en lokal applikation där de kan läsa och redigera wikin. Detta kommer ske som nuvarande wiki-implementationer där de olika aktiviteterna oftast sträcker sig från några enstaka sekunder (bläddring) till några minuter (redigering). Synkroniseringen mellan användarna sker i bakgrunden och ska till stor del vara osynlig.

Applikationen riktar sig mot linux med grafiskt gränssnitt men bör vara skrivet så att det kan bli plattformsoberoende. Wikin kommer behöva integreras med någon form av git-teknologi samt ett peer-to-peer-protokoll.
\section{Produktöversikt}
\subsection*{Behov och funktoner}
\begin{tabular}{|l|l|l|l|}
\hline
Behov & Prioritet & Funktion & Planerad publikation \\
\hline
Läsa och bläddra i wikin & Hög & Läsare med länkfunktion & 1 \\
\hline
Redigera wikin & Hög & Redigeringsverktyg & 1 \\
\hline
Installera programmet & Hög & Integrera med en paktethanterare & 1 \\
\hline
Spåra ändringar & Normal & Visa skillnader mellan versioner & 2 \\
\hline
Distribuerad teknik & Normal & Peer-to-peer & 2 \\
\hline 
Bilder & Normal & Lägga till bilder i en artikel & 2 \\
\hline
Rättstavning & Låg & Integrera rättstavning i redigeringsverktyget & 3 \\
\hline
Signering & Normal & Signering av artiklar & 3 \\
\hline
\end{tabular}
\section{Övriga produktbehov}
\begin{tabular}{|l|l|l|}
\hline\
Krav & Prioritet & Planerad publikation \\
\hline
Användarvänligt redigeringsgränssnitt & Hög & - \\
\hline
Låga minimikrav på användarens tekniska förmåga & Hög & - \\
\hline
Plattformsoberoende & Normal & - \\
\hline
Hög robusthet i systemet & Normal & - \\
\hline
\end{tabular}
\end{document}
