%
%  untitled
%
%  Created by David Granqvist on 2008-09-08.
%  Modified by Martin Erola
%

% Set document format/class
\documentclass[a4paper,twoside]{article}

%%%%%%%%%%%%%%%%%%%
% Include packages
%
\usepackage[utf8]{inputenc}   % Use utf-8 encoding for foreign characters
\usepackage[swedish]{babel}   % Support for swedish letters
\usepackage{fullpage}         % Setup for fullpage use
\usepackage{fancyhdr}         % Running Headers and footers
\usepackage{boxedminipage}    % Surround parts of graphics with box
\usepackage{listings}         % Package for including code in the document
\usepackage{ifpdf}            % Recommended way for checking for PDFLaTeX:
\usepackage{tabularx}         % Tabeller med automatisk stretch
% \usepackage[nofancy]{svninfo} % Extract Subversion info about the file
% \usepackage{color}          % Color
% \usepackage{lastpage}       % Total page count

% Graphics
\ifpdf
\usepackage[pdftex]{graphicx}
\else
\usepackage{graphicx}
\fi

%%%%%%%%%%%%%%%%%%%%%%%%%%%%%%%%%%%%%%%%%%%%%%%%%%%%%%%%%%
% Uncomment some of the following if you use the features
%

% Multipart figures
%\usepackage{subfigure}

% More symbols
%\usepackage{amsmath}
%\usepackage{amssymb}
%\usepackage{latexsym}

% If you want to generate a toc for each chapter (use with book)
% \usepackage{minitoc}

%%%%%%%%%%%%%%%%%%%%
% Document settings
%

% Header
\pagestyle{fancy}
% Sätter en marginal mellan header och (ovanstående?) text %
\setlength\headsep{10pt}
% Sätter höjden på headern
\setlength{\headheight}{32pt}

% Sätter styckesinställningar
\setlength\parindent{0pt}
\setlength\parskip{10pt}



\ifpdf
  \DeclareGraphicsExtensions{.pdf, .jpg, .tif, .png}
  \pdfinfo{            
    /Title  (Architecture Notebook)
    /Author (PUM-grupp 1)
  }
\else
  \DeclareGraphicsExtensions{.eps, .jpg}
\fi

\title{Distribuerad wiki \\ Design}
\author{PUM-grupp 1}
\date{\today}

\begin{document}

\maketitle

\thispagestyle{empty}

\newpage

{\centering \Large{Dokumenthistorik\\}}

\vspace{10pt}
\begin{tabularx}{\textwidth}{ |l|l|X|l|l| }
  \hline
    \textbf{version} & \textbf{datum} & \textbf{utförda ändringar} & \textbf{utförda av} & \textbf{granskad} \\
	\hline 
  0.1 & 2009-02-12 &  Ett första utkast  & Alla & Alla   \\
  \hline
\end{tabularx}

\newpage

\setcounter{tocdepth}{2}
\tableofcontents
\newpage

\section{Design}

\section{Systemet}
Nedan följer designen för det färdiga systemet.
\subsection{Överblick över systemet}
Systemet består av följande delsystem. 
\begin{itemize}
\item Ett huvudfönster(GUI) som är gränsnittet mot användaren.
\item En filhanterare(Filehandler) som hanterar de filer och objekt systemet i huvudsak jobbar mot.
\item En objekttyp(Object) som är den datatypen filhanteraren och GUI:t jobbar emot.
\item Ett delsystem för att hantera den interna kommunikationen.
\item Ett delsysyem för att hantera den externa kommunikationen. 
\item Ett grupphanteringssystem som administrerar grupperna.
\item En krypterad socket som garanterar säker kommunikation.
\item Ett distruberat filhanteringssystem Bazaar.
\end{itemize}
Nedan följer en överskådlig bild på systemet.
BILDBILDBILDBILDBILDBILDBILDBILDBILDBILDBILDBILDBILDBILDBILDBILDBILDBILDBILDBILD
\subsection{Delsystem}
Nedan följer en mer detaljerad beskrivning över de olika delsystemen.
\subsubsection{GUI}
GUI är systemets gränsnitt mot användaren. Den består av en texteditor och knappar för att interagera med resten av systemet.


\begin{tabular}{|l|l|}
\hline
\multicolumn{2}{|c|}{\textbf{GUI}} \\
\hline
\multicolumn{2}{|c|}{current\_file:Filehandle} \\
\hline
\_\_init\_\_(f:filehandler) &konstruktor\\
OnNew(filename) & Metod för att skapa en ny artikel\\
OnOpen() & Metod för att ladda in och läsa en artikel \\ 
OnSave() & Metod för att spara en artikel \\
OnEdit() & Metod för att editera en artikel \\
OnClose() & Metod för att stänga en artikel \\
OnDelete() & Metod för att ta bort en artikel\\
notify(event:string) & Metod för att uppmärksamma en händelse\\
\hline
\end{tabular}

\subsubsection{Filehandler}
Filehandler är kopplinen mellan artikelfilerna och resten av systemet. Filhanteraren ser till så att nödvändiga parter blir informerade när en ändring eller en uppdatering på en fil har skett. 

\begin{tabular}{|l|l|}
\hline
\multicolumn{2}{|c|}{\textbf{Filehandler}} \\
\hline
\multicolumn{2}{|c|}{Directory:string}
\multicolumn{2}{|c|}{Listeners:list} \\
\multicolumn{2}{|c|}{Objects:list} \\
\hline
\_\_init\_\_(directory) &konstruktor\\
new_object(filename:string) : object & skapar och returnerar ett nytt object\\
new_file(filename:string) & informerar filhanteraren om att det ilkommit en fil\\
update_file(filename:string) & informerar filhanteraren om att en fil blivit uppdaterad\\\\
remove_file(filename:string) & informerar filhanteraren om att en fil blivit borttagen\\
delete(o:object) & tar bort ett object\\
close(o:object) & stänger ett object\\
collition(filename:string\\)
reg\_listener(object) & lägger till en listener \\
unreg\_listener(object) & tar bort en listener \\
notify\_listeners(event:string,argument:string) & uppmärksammar lyssnare på en händelse\\
get_dir_list(): list & returnerar en lista med innehållet i katalogen\\
\hline
\end{tabular}

\subsubsection{Objekt}
Object är den klassen som kommer att representera alla aktiva artiklar. Ett objekt håller koll på statuset av artikeln och kan anropas med operationer på en artikel.

\begin{tabular}{|l|l|}
\hline
\multicolumn{2}{|c|}{\textbf{Object}} \\
\hline
\multicolumn{2}{|c|}{status: string} \\
\multicolumn{2}{|c|}{filename: string} \\
\multicolumn{2}{|c|}{path: string} \\
\multicolumn{2}{|c|}{f:filehandler} \\
\hline
\_\_init\_\_(f: filehandlerdirectory,status:string,path:string,filename:string) &Konstruktor, ställer in filehandler,status,path och filnamn\\
load(): filehandle & Returnerar handtaget till artikeln \\
save() & sparar ner information i filen \\
close() & stänger artikelfilen\\
get\_status(): string & Returnerar status för objektet \\
\hline
\end{tabular}

\subsubsection{Interna kommunikationen}
InternKomm är gränsnittet mellan filhanteraren, Bazaar och den externa kommunikationen. InternKomm låter filhanteraren veta om någon fil blivit uppdaterad av Bazaar. Likaså låter han Bazaar och den externa kommunikationen veta om någon fil uppdaterats i filhanteraren.

\begin{tabular}{|l|l|}
\hline
\multicolumn{2}{|c|}{\textbf{InternKomm}} \\
\hline
\hline
\_\_init\_\_() &Konstruktor\\
notify(event:string)\\
\hline
\end{tabular}
\end{document}
