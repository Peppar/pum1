% Standardinkluderingsfil
%
%  untitled
%
%  Created by David Granqvist on 2008-09-08.
%  Modified by Martin Erola
%

% Set document format/class
\documentclass[a4paper,twoside]{article}

%%%%%%%%%%%%%%%%%%%
% Include packages
%
\usepackage[utf8]{inputenc}   % Use utf-8 encoding for foreign characters
\usepackage[swedish]{babel}   % Support for swedish letters
\usepackage{fullpage}         % Setup for fullpage use
\usepackage{fancyhdr}         % Running Headers and footers
\usepackage{boxedminipage}    % Surround parts of graphics with box
\usepackage{listings}         % Package for including code in the document
\usepackage{ifpdf}            % Recommended way for checking for PDFLaTeX:
\usepackage{tabularx}         % Tabeller med automatisk stretch
% \usepackage[nofancy]{svninfo} % Extract Subversion info about the file
% \usepackage{color}          % Color
% \usepackage{lastpage}       % Total page count

% Graphics
\ifpdf
\usepackage[pdftex]{graphicx}
\else
\usepackage{graphicx}
\fi

%%%%%%%%%%%%%%%%%%%%%%%%%%%%%%%%%%%%%%%%%%%%%%%%%%%%%%%%%%
% Uncomment some of the following if you use the features
%

% Multipart figures
%\usepackage{subfigure}

% More symbols
%\usepackage{amsmath}
%\usepackage{amssymb}
%\usepackage{latexsym}

% If you want to generate a toc for each chapter (use with book)
% \usepackage{minitoc}

%%%%%%%%%%%%%%%%%%%%
% Document settings
%

% Header
\pagestyle{fancy}
% Sätter en marginal mellan header och (ovanstående?) text %
\setlength\headsep{10pt}
% Sätter höjden på headern
\setlength{\headheight}{32pt}

% Sätter styckesinställningar
\setlength\parindent{0pt}
\setlength\parskip{10pt}



\ifpdf
  \DeclareGraphicsExtensions{.pdf, .jpg, .tif, .png}
  \pdfinfo{            
    /Title  (Use-case model)
    /Author (PUM-grupp 1)
  }
\else
  \DeclareGraphicsExtensions{.eps, .jpg}
\fi

\title{Use-case model}
\author{PUM-grupp 1}
\date{\today}

\begin{document}

\maketitle\thispagestyle{empty}

\newpage

\section{Inledning}
För att enklare förstå de olika use-casen i har satt upp och få en överblick över dem har vi skapat en use-case model där vi grafisk visar hur användare och system hänger ihop.
\section{Överblick}
%Skapa en överblick för systemet, dvs beskriv vad systemet gör och vad desss mål är.
Systemet i fråga är ett distribuerat versionshanteringssystem. Meningen med detta system är att man på ett enkelt sätt ska kunna hantera dokument i en wiki-miljö samtidigt som man inte ens behöver vara uppkopplad. När användaren sedan kopplar upp sig mot internet kan denne synkronisera sina ändringar med andra användares ändringar.
\section{Use-case diagram}
%\includegraphics{use-case-diagram.png}
Här ska en bild på ett use-case diagram finnas
\section{Aktörer}
%Beskriv alla aktörer som finns med i use-case diagrammet.
\begin{itemize}
	\item Användare
	\\Denna aktör representerar en normal användare inom systemet.
	\item Git
	\\Denna aktör ska representera systemet som ligger under wiki.
\end{itemize}
\section{Use-cases}
%Rada upp de olika use-cases som finns.
\subsection{Bläddra}
Detta use-case beskriver hur man bläddrar bland dokument.
\subsection{Distribuering}
Detta use-case beskriver hur man distribuerar dokument till andra användare inom wiki-gruppen.
\subsection{Läsa artiklar}
Detta use-case beskriver hur man läser sina egna eller andra användares artiklar.
\subsection{Skriva en artikel}
Detta use-case beskriver hur man skriver en artikel.
\subsection{Ta tillbaka gammal version}
Detta use-case beskriver hur man tar tillbaka en gammal version av en fil.
\subsection{Sammanslagning}
Detta use-case beskriver hur man sammanfogar två eller flera filer med varandra.
\subsection{Rättstavning}
Detta use-case beskriver hur man kan använda rättstavningskontrollen.
\end{document}
