% Standardinkluderingsfil
%
%  untitled
%
%  Created by David Granqvist on 2008-09-08.
%  Modified by Martin Erola
%

% Set document format/class
\documentclass[a4paper,twoside]{article}

%%%%%%%%%%%%%%%%%%%
% Include packages
%
\usepackage[utf8]{inputenc}   % Use utf-8 encoding for foreign characters
\usepackage[swedish]{babel}   % Support for swedish letters
\usepackage{fullpage}         % Setup for fullpage use
\usepackage{fancyhdr}         % Running Headers and footers
\usepackage{boxedminipage}    % Surround parts of graphics with box
\usepackage{listings}         % Package for including code in the document
\usepackage{ifpdf}            % Recommended way for checking for PDFLaTeX:
\usepackage{tabularx}         % Tabeller med automatisk stretch
% \usepackage[nofancy]{svninfo} % Extract Subversion info about the file
% \usepackage{color}          % Color
% \usepackage{lastpage}       % Total page count

% Graphics
\ifpdf
\usepackage[pdftex]{graphicx}
\else
\usepackage{graphicx}
\fi

%%%%%%%%%%%%%%%%%%%%%%%%%%%%%%%%%%%%%%%%%%%%%%%%%%%%%%%%%%
% Uncomment some of the following if you use the features
%

% Multipart figures
%\usepackage{subfigure}

% More symbols
%\usepackage{amsmath}
%\usepackage{amssymb}
%\usepackage{latexsym}

% If you want to generate a toc for each chapter (use with book)
% \usepackage{minitoc}

%%%%%%%%%%%%%%%%%%%%
% Document settings
%

% Header
\pagestyle{fancy}
% Sätter en marginal mellan header och (ovanstående?) text %
\setlength\headsep{10pt}
% Sätter höjden på headern
\setlength{\headheight}{32pt}

% Sätter styckesinställningar
\setlength\parindent{0pt}
\setlength\parskip{10pt}



\ifpdf
  \DeclareGraphicsExtensions{.pdf, .jpg, .tif, .png}
  \pdfinfo{
    /Title  (Kravlista)
    /Author (PUM-grupp 1)
  }
\else
  \DeclareGraphicsExtensions{.eps, .jpg}
\fi

\title{Kravlista}
\author{PUM-grupp 1}
\date{\today}

\begin{document}

\maketitle\thispagestyle{empty}

\newpage

{\centering \Large{Dokumenthistorik\\}}

\vspace{10pt}
\begin{tabularx}{\textwidth}{ |l|l|X|l|l| }
  \hline
    \textbf{version} & \textbf{datum} & \textbf{utförda ändringar} & \textbf{utförda av} & \textbf{granskad} \\
	\hline 
  0.1 & 2009-03-03 &  Ett första utkast  & Mikael & INGEN \\
  \hline
\end{tabularx}

\newpage

\setcounter{tocdepth}{2}
\tableofcontents
\newpage

\section{Inledning}
+ Siteinstallation
    Installera på dator
    Avinstallera från dator
    + Programmet
        Starta programmet (huvudsida visas? välj wiki?)
        Avsluta programmet (fråga om ändringar ska sparas?)
        + Wiki
            Skapa wiki (om man är inblandad i flera projekt så borde man kunna skapa en wiki för varje)
            Ändra wikinamn? (varje wiki bör ju ha ett namn att identifieras med)
            Välj wiki? (om man har flera, kan lösas på annat sätt kanske)
            Ta bort wiki
            Visa artikelindex (lista över alla artiklar på wikin)
            Skriv in artikelnamn (direkt åtkomst till artikel)
            Sök artikel/text?
            + Hjälpfunktionalitet
                Visa hjälp/manual (viktigt för nybörjare)
            + Användare (kompisar/kollegor)
                Tillåt användare att läsa min wiki (lägger till nyckeln)
                Tillåt användare att redigara min wiki (lägger till nyckeln)
                Ta bort användare från min wiki (ta bort nyckeln från databasen)
                ...?
            + Artikel
                Skapa
                Ändra namn
                Ta bort
                Skriva
                    "Outline" (jag är inte helt säker vad detta är, men han ville ju ha det)
                    Formatera text
                        Länka artikel (även länka till rubrik i artikel?)
                        Formatera stycke (fet, kursiv, understruken, genomstruken)
                        Rubriker (antal nivåer?)
                        Skapa/ta bort listor (punkter och ordnade)
                        Öka/minska indrag av text
                        Förformaterad text (HTML: <pre>, MYCKET användbart om vi använder ett eget wikispråk)
                        Kopiera/klipp ut/klistra in (rätt självklart, men gränssnittet borde ha knappar för detta också)
                        Färg/bakgrundsfärg för text?
                    Bilder
                        Lägg in
                        Ta bort
                    Signera
                    Avbryta redigering
                    Förhandsgranska (?, om vi kör med wikispråk utan WYSIWYG)
                    Kommentera/diskutera artikel?
                Läsa
                    Skriva ut?
                    Följa länk
                Historik
                    Läsa
                    Jämföra (?, egentligen bara en enkel "diff")
                    Ändra tillbaka artikel till gammal version

\end{document}
