% Standardinkluderingsfil
%
%  untitled
%
%  Created by David Granqvist on 2008-09-08.
%  Modified by Martin Erola
%

% Set document format/class
\documentclass[a4paper,twoside]{article}

%%%%%%%%%%%%%%%%%%%
% Include packages
%
\usepackage[utf8]{inputenc}   % Use utf-8 encoding for foreign characters
\usepackage[swedish]{babel}   % Support for swedish letters
\usepackage{fullpage}         % Setup for fullpage use
\usepackage{fancyhdr}         % Running Headers and footers
\usepackage{boxedminipage}    % Surround parts of graphics with box
\usepackage{listings}         % Package for including code in the document
\usepackage{ifpdf}            % Recommended way for checking for PDFLaTeX:
\usepackage{tabularx}         % Tabeller med automatisk stretch
% \usepackage[nofancy]{svninfo} % Extract Subversion info about the file
% \usepackage{color}          % Color
% \usepackage{lastpage}       % Total page count

% Graphics
\ifpdf
\usepackage[pdftex]{graphicx}
\else
\usepackage{graphicx}
\fi

%%%%%%%%%%%%%%%%%%%%%%%%%%%%%%%%%%%%%%%%%%%%%%%%%%%%%%%%%%
% Uncomment some of the following if you use the features
%

% Multipart figures
%\usepackage{subfigure}

% More symbols
%\usepackage{amsmath}
%\usepackage{amssymb}
%\usepackage{latexsym}

% If you want to generate a toc for each chapter (use with book)
% \usepackage{minitoc}

%%%%%%%%%%%%%%%%%%%%
% Document settings
%

% Header
\pagestyle{fancy}
% Sätter en marginal mellan header och (ovanstående?) text %
\setlength\headsep{10pt}
% Sätter höjden på headern
\setlength{\headheight}{32pt}

% Sätter styckesinställningar
\setlength\parindent{0pt}
\setlength\parskip{10pt}



\ifpdf
  \DeclareGraphicsExtensions{.pdf, .jpg, .tif, .png}
  \pdfinfo{
    /Title  (Kvalitetsplan)
    /Author (PUM-grupp 1)
  }
\else
  \DeclareGraphicsExtensions{.eps, .jpg}
\fi

\title{Kvalitetsplan}
\author{PUM-grupp 1}
\date{\today}

\begin{document}

\maketitle\thispagestyle{empty}
\newpage

{\centering \Large{Dokumenthistorik\\}}

\vspace{10pt}
\begin{tabularx}{\textwidth}{ |l|l|X|l|l| }
  \hline
    \textbf{version} & \textbf{datum} & \textbf{utförda ändringar} & \textbf{utförda av} & \textbf{granskad} \\
	\hline 
  1.0 & 2009-03-01 &  Första versionen för opposition  & Linus & INGEN \\
  \hline
\end{tabularx}

\newpage

\setcounter{tocdepth}{2}
\tableofcontents
\newpage

\section{Syfte och avgränsning}
Denna kvalitetsplan omfattar utvecklingen av en distribuerad wiki, ett projektarbete i kursen TDDD09 på Linköpings Tekniska Högskola. Kvalitetsplanen innefattar både utvecklingen av själva produkten där utvecklingsmetodik och tester kan nämnas som exempel men också kvalitetskrav och mål för projektgruppens rutiner och arbetssätt där dokumentgranskningsprocess kan ges som exempel. 

\section{Referensdokument}
I texten refereras till ett flertal andra projektdokument vars innehåll och ansvarig roll beskrivs kortfattat under Dokumentation - Dokument och ansvarig. 

\section{Organisation}
Här beskrivs projektets organisation och projektmedlemmarnas roller. Vilka dokument som varje roll ansvarar för finns under Dokumentation - Dokument och ansvarig.

\subsection{Roller}
Gruppmedlemmarna är alla tilldelade minst en roll för vilken de ansvarar för ett specifikt område med uppgifter och dokument.

\subsubsection{Projektledare}
Denna roll finns med i projektmodellen OpenUP och är tilldelad Mikael Waernér. Projektledarens huvudansvar är att driva projektet mot ett mål inom tidsramen för projektet.

\subsubsection{Analytiker}
Denna roll finns med i projektmodellen OpenUP och är tilldelad Erik Thorselius. Analytikerns främsta uppgift är att föra dialog med kund och komma överens om en gemensam vision.

\subsubsection{Arkitekt}
Denna roll finns med i projektmodellen OpenUP och är tilldelad Oskar Holstensson. Arkitektens främsta uppgift är att ta fram en passande arkitektur för utvecklingen av slutprodukten.

\subsubsection{Utvecklare}
Denna roll finns med i projektmodellen OpenUP och är tilldelad Victor Ortman. Utvecklaren ansvarar för att produktutvecklingen genomförs med hög kvalité enligt arkitekturdesign.

\subsubsection{Testare}
Denna roll finns med i projektmodellen OpenUP och är tilldelad Martin Pettersson. Testaren är ansvarig för alla tester som genomförs i de olika faserna av utvecklingen.

\subsubsection{Kvalitetssamordnare}
Denna roll finns inte med i projektmodellen OpenUP utan är ett lokalt tillägg till modellen och är tilldelad Linus Dunkers. Kvalitetssamordnarens uppgift är att se till att projektet håller en hög kvalité både när det gäller framställd produkt och arbetssätt.

\subsubsection{Dokumentansvarig}
Denna roll finns inte med i projektmodellen OpenUP utan är ett lokalt tillägg till modellen och är tilldelad Linus Dunkers. Dokumentansvarig har bland annat som uppgift att förse gruppen med ett versionshanteringssystem och en dokumentgranskningsprocess.

\subsection{Resurser}
Det är inte uppenbart vilka resurser som är och inte är relaterade till kvalitetsarbete. I följande stycke klargörs vilka resurser som finns och hur de relaterar till kvalitetsarbete.
\subsubsection{Totala resurser}
Projektets totala resurser betstår främst av mantimmar. Tidbudgeten är ca 1200 timmar och ska fördelas så jämnt som möjligt mellan gruppmedlemmarna vilket betyder att varje gruppmedlem ska arbeta ungefär 200 timmar med projektet. Då projektet helt saknar ekonomiska resurser så påverkas bland annat verktyg och utvecklingsmiljö där öppen källkod därför blivit en avgörande del i genomförandet av projektet. Projektgruppen har också tillgång till en handledare som vid ett tillfälle i veckan kan hjälpa till med de problem som eventuellt uppstår och ge tips samt granskar och godkänner de dokument som skrivs. Projektgruppen har även tillgång till flera av universitetets datorsalar och grupprum där mötena hålls. Universitetet tillhandahåller också en guide för projektarbete enligt OpenUP-modellen.
\subsubsection{Resurser för kvalitetsarbete}
En av projektgruppens roller, kvalitetssamordnare,  har som främsta uppgift att ansvara för att projektet håller en god kvalité. Detta innebär att ungefär 200 mantimmar är avsatt för direkt kvalitetsarbete. De övriga resurserna ovan syftar alla till att främja en god utvecklingsmiljö och därav också en god kvalité hos slutprodukten och dokumentationen.

\section{Dokumentation}
En stor del av projektet består utav att ta fram de dokument som ska skrivas. För att detta ska bli gjort så har ansvaret varje enskilt dokument blivit tilldelad någon utav projektmedlemmarnas roller.
\subsection{Dokument och ansvarig}
I detta stycke beskrivs samtliga dokument kort samt vilken roll som ansvarar för att dokumentet blir gjort.

\subsubsection{Vision}
Analytikern är ansvarig för framställandet av en projektvision som innehåller en för kund och projektgrupp gemensam och överenskommen vision för slutprodukten.

\subsubsection{Projektplan}
Projektledaren är ansvarig för framställandet av en projektplan innehållande bland annat projektets organisation, uppgifter och milstolpar.

\subsubsection{Use-Case}
Analytikern är ansvarig för framställandet av ett dokument med alla Use-Case för den tänkta slutprodukten. Use-Case eller användarfall definierar krav på bland annat produktens funktioner.

\subsubsection{Use-Case Model}
Analytikern är ansvarig för framställandet av ett Use-Case Model där man överskådligt skall se hur de olika Use-Cases är relaterade till varandra och användarna av systemet.

\subsubsection{System-WideRequirements}
Analytikern är ansvarig för framställandet av System-WideRequirements som innehåller projektets samtliga övergripande mål, såväl funktionella som icke-funktionella.

\subsubsection{Risklista}
Projektledaren är ansvarig för framställandet av en risklista innehållande hela projektgenomförandets olika risker, deras inverkan samt sätt att kringgå dessa eller minska dess inverkan.

\subsubsection{Glossary}
Analytikern är ansvarig för framställandet av en Glossary (ordbok). Denna tar upp och beskriver facktermer och begrepp som används inom projektet.

\subsubsection{Iterationsplan}
Projektledaren är ansvarig för framställandet av en iterationsplan för varje iteration under projektet innehållande bland annat start och slutdatum, milstolpar och mål med iterationen. Kvalitetssamordnaren är ansvarig för dokumentets sista del rörande utvärdering av iterationen.

\subsubsection{Work items list}
Projektledaren är ansvarig för framställandet av en Work items list som beskriver projektets eller iterationens arbetsuppgifter.

\subsubsection{Architecture notebook}
Arkitekten är ansvarig för framställandet av Architecture notebook som är en typ av anteckningsbok där alla arkitektuella beslut dokumenteras.

\subsubsection{Kvalitetsplan}
Kvalitetssamordnaren är ansvarig för framställandet av en Kvalitetsplan innehållande riktlinjer för kvalitetsarbetet under hela projektet.

\subsubsection{Project burndown}
Kvalitetssamordnaren är ansvarig för framställandet av Project burndown som är en graf föreställande projektets utveckling där effektiviteten enkelt kan utläsas som lutningen hos kurvan.

\subsubsection{Iteration burndown}
Kvalitetssamordnaren är ansvarig för framställandet av Iteration burndown som på samma sätt som Project burndown beskriver effektiviteten fast för en enstaka iteration.

\subsubsection{Erfarenhetsrapport}
Kvalitetssamordnaren är ansvarig för framställandet av en Erfarenhetsrapport innehållande viktiga erfarenheter som vunnits under projektets gång.

\subsection{Dokumentgranskningsprocess}
För att säkerställa hög kvalité på de dokument som producerats så skall allt granskas av någon annan än författaren innan det kan sägas vara godkänt för inlämning. Det är därför viktigt att en dokumentgranskningsprocess beskrivs och tas beslut om att följa, så att det blir en rutin att kvalitetsgranska textmaterial.

\subsubsection{Skrivande}
Alla skriver i LaTeX och använder samma mall och "kod-stil". När någon som är ansvarig för ett dokument behöver hjälp så meddelas detta via gruppmejlen. Om projektledaren anser att ansvaret för ett dokument skall delas upp så ska detta också meddelas per gruppmejl med information om exakt vilka avsnitt som tilldelats vem. Gruppmedlemmar som har tid över ska aktivt leta efter arbetsuppgifter som att till exempel skriva en del av ett dokument. Alltid ska dokument som uppdaterats skickas upp på GitHub så att andra ur gruppen kan granska texten löpande under projektets gång. Ansvarig för dokumentet skall ha läst igenom dokumentet och finjusterat innan han kan meddela dokumentet klart för granskning. När ett dokument anses klart så skall detta meddelas per gruppmejl där det ska framgå på vilket GitHub-konto som dokumentet ligger.

\subsubsection{Justering/granskning}
Dokumentet ramlar nedåt i listan Gruppmedlemmar. Om alla skulle redigera ett "färdigt" dokument samtidigt så kommer det generera en hel del konflikter och ändringar som motverkar varandra. Därför skall bara än åt gången justera ett dokument. Så efter att författaren av ett dokument meddelat att dokumentet är klart för granskning så är det den efterföljande gruppmedlemmens ansvar att läsa igenom dokumentet och ändra såväl grammatiska fel, stavfel och enkla tankefel. När det kommer till större fel, där hela stycken är inblandade eller slutsatser som är direkt felaktiga så skall författaren kontaktas och felet utredas innan ändring utförs. När justeringarna är gjorda så skall dokumentet fortsätta ned i listan. När dokumentet "är tillbaka" hos författaren så anses det justerat och godkänt för inlämning om minst två andra gruppmedlemmar då haft tid att granska dokumentet.

\subsubsection{Krav}
För att dokumentgranskningsprocessen skall flyta på så måste alla se till att läsa igenom och justera det dokument som man blivit tilldelad. Om man inte har tid att göra detta så måste man direkt meddela detta till nästa person på listan som istället tar sig an uppgiften. En stor fördel med detta arbetssätt är att alla tvingas läsa samtliga dokument vilket kommer bidra till att alla är bättre insatta i projektet.

\section{Standarder, praxis, konventioner och mått}
I detta stycke beskrivs vilka standarder, praxis, konventioner och mått som används i projektet.

\subsection{Standard - IEEE Std 730}
Kvalitetsplanen är skriven enligt IEEE Std 730 med ett fåtal undantag.

\subsection{Praxis - OpenUP}
Projektet bedrivs i stort enligt OpenUP med ett fåtal undantag och tillägg. En mall för projektarbete enligt OpenUP har tillhandahållits av universitetet samt de lokala tilläggen och undantagen. Mallen innehåller bland annat så kallade Practices som kan vara till stor hjälp för att förstå innebörden och funktionen av olika delmoment i OpenUP.

\subsection{Konventioner - Kodstandard}
Indentering i koden sker enbart med hjälp av tabbar och inte mellanslag. Det är viktigt att vara konsekvent med indenteringen eftersom problem i Python kan uppstå annars. Variabelnamn skrivs med små bokstäver och eventuella mellanslag i variabelnamn skall vara understreck. Klassnamn börjar på stor bokstav och mellanslag tas helt enkelt bort ifrån namnet. Om ett klassnamn består av flera ord, så ska varje ord börja med stor bokstav.

Efter deklaration av klassens namn ska en kort kommentar finnas om dess syfte. Kommentaren skall börja och avslutas med tre stycken citattecken (exempel: ''''''Denna klass används för nätverkskommunikation''''''). Övriga kommentarer görs med hjälp av \#-tecknet (exempel: \# initiering).

\subsection{Mått - Burndowns}
Varje iteration och projektet i sin helhet har burndown-diagram som fungerar som ett kvalitetsmått. 

\section{Granskningsmetoder}
Den metod vi använder för granskning av dokument finns beskriven i Dokumentation - Dokumentgranskningsprocess. Granskning av kod sker enligt Testplan i stycket nedan.

\section{Testplan}
För att säkerställa en hög kvalité hos slutprodukten så kommer tester utföras både i form av daily builds men också med ett antal tester innan en ny release kan släppas.

\subsection{Daily build}
I projektet ska vi använda oss daily builds, dvs att vi varje dag kompilerar för att vara säkra på att inga allvarliga fel har uppstått men även för se till att alla beroenden fortfarande fungerar. Varje daily build kan komma att följas upp med ett Smoke Test. Det innebär enkla tester som säkerställa att de grundläggande funktionerna i systemet fortfarande fungerar. 

\subsection{Misslyckade tester}
Vid misslyckade tester ska orsaken till misslyckandet letas upp. Om det är ett allvarligt fel ska åtgärder för att neutralisera felet genast tas. Om det inte är lika hög prioritet kan felet loggas och åtgärdas vid senare tillfälle. I vissa fall kan felet ignoreras, men detta sker dock endast om det är ett mycket obetydligt fel eller om sannolikheten att det kommer inträffa vid normal användning är minimal.

\subsection{Alfatest}
När projektgruppen anser att man kommit fram till en ny version som är körbar så görs först ett alfatest där programvaran testas skarpt inom projektgruppen. Först när en alfa-release godkänts så kan denna släppas för betatest hos kund. Modifieringar i kod för att lösa de fel och problem som uppstår under alfatest pågår tills dess att alfatestet godkänts.

\subsection{Betatest}
En betaversion av produkten släpps till kund innan produkten är helt färdig. Detta är till för att kunden ska få en känsla av hur systemet kommer att bli när det är färdigt och komma med åsikter om han tycker att produkten ska ändras på något sätt. Om betatestet godkänns av kund så har de funktioner som ännu implementerats godkänts. Om samtliga funktioner finns implementerade så kan denna beta-release gå vidare för sluttest. Om kunden föreslår ändringar eller påpekar brister i programmet så påbörjas en ny iteration.

\subsection{Sluttest}
När produkten är färdig för leverans enligt projektgruppen görs ett sluttest. Målet med sluttestet är att säkerställa att systemet verkligen uppfyller alla de krav som ställs på systemet i dokumenten System-WideRequirements och Use Case. När både betatest och sluttest godkänts så kan produkten släppas som en ny release för acceptanstest.

\subsection{Acceptanstest}
När produkten som har utvecklats av projektgruppen är färdigställd ska acceptanstest ske. Kunden kommer att få pröva systemet i för att se om han är nöjd med det och sedan godkänna systemet för användning eller om kunden inte är nöjd så kommer ett möte med kunden att ske. Kunden tillsammans med projekgruppen bestämmer i så fall vilka åtgärder som ska vidtas och en ny version av produkten produceras för ett nytt acceptanstest.

\section{Problemhantering och ändringsrutiner}
Kvalitetssamordnaren är ansvarig för insamling och dokumentering av begäran om ändringar. Dessa tas upp på kommande möte och beslutas där om de ska drivas vidare. Då möten protokollförs så finns beslut dokumenterade. Projektgruppen har bestämt att i första hand försöka nå enighet genom diskussion, om detta inte är möjligt så får majoritet avgöra och slutligen i de mycket ovanliga fall att rösterna är jämt fördelade mellan alternativen så har den ansvariga för det område som beslutet gäller eller projektledaren beslutanderätt. Vid varje projektmöte så tas tidigare protokoll upp som en stående punkt. Då ges tillfälle att följa upp om ändringar fått positivt eller negativt utslag. Gruppmedlemmarna lämnar också veckorapporter till kvalitetssamordnaren där möjlighet ges att meddela önskan om ändringar utan att behöva ta upp det i helgrupp.

\section{Verktyg}
Efter att beslut fattats angående vilka verktyg som ska användas så är det kvalitetssamordnarens uppgift att se till att samtliga gruppmedlemmar får tillgång till dessa verktyg och utbildning om det behövs. Detta stycke beskriver de verktyg som har använts under projektets gång.

\subsection{OpenUP}
OpenUP är den projektmodell som vi använder oss av. Men det är också ett verktyg på så sätt att universitetet försett oss med ett webb-baserat verktyg med mallar för OpenUP. Det är en databas av information om de olika faserna, rollerna, dokumenten och uppgifterna. Det finns även mallar och exempel att läsa och ladda hem.

\subsection{Redmine}
Redmine är ett kraftfullt webb-baserat verktyg som vi använder för ett flertal viktiga funktioner.

\subsubsection{Wiki}
Vi har en välfylld wiki med projektrelaterad information. Det är bland annat här vi sparar mötesprotokoll.

\subsubsection{Ärendehantering}
Alla uppgifter registreras som ärenden i Redmine. De kan tilldelas en viss person, man anger även uppskattad tidsåtgång och start- och sluttid.

\subsubsection{Tidrapportering}
En annan funktion är möjligheten att logga tid på ärenden eller på projektet i sin helhet. Vi använder denna funktion för tidrapportering vilket har stöd för att automatiskt generera rapporter, gantchart och kalender.

\subsubsection{Burndowns}
Det finns ingen färdig funktion för att göra burndown-diagram i Redmine men vi använder fältet uppskattad tidsåtgång som ett point-system. När ett ärende markerats som avslutat så sjunker kurvan och detta ger oss vårt burndown-diagram.

\subsection{Git och GitHub}
Under de två första faserna (inception och elaboration) så användes Git som versionshanteringssystem. Då inte alla har sina datorer igång dygnet runt så använde vi oss också av konton på GitHub dit vi bänkade upp ändringarna så att alla andra i gruppen kunde ta hem ändringarna.

\subsection{Bazaar}
Under elaboration-fasen så framgick det att vårt tidigare val av distribuerat versionshanterinssystem inte var anpassat efter våra behov och vi valde att gå över till att använda Bazaar. Då vi ändå skulle vara tvungna att sätta oss in i hur Bazaar fungerar så var det lika bra att gå över helt till det systemet och sluta använda Git och GitHub från construction-fasen och vidare.

\subsection{Eclipse och PyDev}
Som programmeringsmiljö testade vi först Code::Blocks under en kort tid men insåg snabbt att valet av språk, Python, gjorde Eclipse med tillägget PyDev till en mycket trevligare programmeringsmiljö och vi beslutade därför under elaboration-fasen att det var detta vi skulle använda.

\section{Media}
Då projektet är tänkt att resultera i en slutprodukt som ska släppas som öppen källkod så kommer vi paketera slutprodukten och distribuera den fritt via SourceForge eller Google Code eller motsvarande. Eftersom programmet samt källkoden kommer att vara fritt att använda för vem som helst så behöver inga åtgärder göras för att skydda källkoden eller förhindra olovlig kopiering.

\section{Dokumenthantering}
Då dokumenthanteringen skötts av versionshanteringssystemen Git och Bazaar så har det hela tiden varit öppet för vem som helst att läsa dokumentation och källkod. Det har också alltid funnits kopior på alla dokument på samtliga gruppmedlemmars privata datorer samt på allas GitHub-konton. Detta har gjort att ingen backuplösning behövts sättas upp.

\section{Utbildning}
I samband med att verktygen bestämts så har tid avsatts för att gruppmedlemmarna ska ha tid att sätta sig in i verktyget och hinna fråga om det är något som är oklart. Programmeringsspråket Python hade alla arbetat med tidigare mer eller mindre detsamma gäller för Eclipse. Git och Redmine är det som var ganska nytt för alla och tog mest tid för utbildning. Under Construction så kommer vissa delar såsom autentisering, kryptering och signering att ta extra lång tid med avseende på utbildning eftersom ingen i projektgruppen arbeta med detta tidigare. Inga gränser är satta på hur mycket tid man får lägga på utbildning, målet är både att få något körbart, väldokumenterat innan deadline men också att gruppmedlemmarna ska ha lärt sig mycket nya kunskaper om att arbeta i projekt samt sakkunskaper om till exempel programmering i Python, distribuerade versionshanteringssystem såsom Git och Bazaar samt säkerhetsteknikerna som nämndes tidigare i detta stycke.

\section{Riskhantering}
Identifikation, bedömning och hantering av risker tas upp i ett eget dokument, Risklist.

\end{document}
