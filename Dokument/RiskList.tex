% Standardinkluderingsfil
%
%  untitled
%
%  Created by David Granqvist on 2008-09-08.
%  Modified by Martin Erola
%

% Set document format/class
\documentclass[a4paper,twoside]{article}

%%%%%%%%%%%%%%%%%%%
% Include packages
%
\usepackage[utf8]{inputenc}   % Use utf-8 encoding for foreign characters
\usepackage[swedish]{babel}   % Support for swedish letters
\usepackage{fullpage}         % Setup for fullpage use
\usepackage{fancyhdr}         % Running Headers and footers
\usepackage{boxedminipage}    % Surround parts of graphics with box
\usepackage{listings}         % Package for including code in the document
\usepackage{ifpdf}            % Recommended way for checking for PDFLaTeX:
\usepackage{tabularx}         % Tabeller med automatisk stretch
% \usepackage[nofancy]{svninfo} % Extract Subversion info about the file
% \usepackage{color}          % Color
% \usepackage{lastpage}       % Total page count

% Graphics
\ifpdf
\usepackage[pdftex]{graphicx}
\else
\usepackage{graphicx}
\fi

%%%%%%%%%%%%%%%%%%%%%%%%%%%%%%%%%%%%%%%%%%%%%%%%%%%%%%%%%%
% Uncomment some of the following if you use the features
%

% Multipart figures
%\usepackage{subfigure}

% More symbols
%\usepackage{amsmath}
%\usepackage{amssymb}
%\usepackage{latexsym}

% If you want to generate a toc for each chapter (use with book)
% \usepackage{minitoc}

%%%%%%%%%%%%%%%%%%%%
% Document settings
%

% Header
\pagestyle{fancy}
% Sätter en marginal mellan header och (ovanstående?) text %
\setlength\headsep{10pt}
% Sätter höjden på headern
\setlength{\headheight}{32pt}

% Sätter styckesinställningar
\setlength\parindent{0pt}
\setlength\parskip{10pt}



\ifpdf
  \DeclareGraphicsExtensions{.pdf, .jpg, .tif, .png}
  \pdfinfo{            
    /Title  (Risklista)
    /Author (PUM-grupp 1)
  }
\else
  \DeclareGraphicsExtensions{.eps, .jpg}
\fi

\title{Ristlista}
\author{PUM-grupp 1}
\date{\today}

\begin{document}

\maketitle\thispagestyle{empty}

\newpage

\setcounter{tocdepth}{2}
\tableofcontents
\newpage

\section{Inledning}
Nedan följer de risker som är aktuella i projektet

\subsection{Maintainability / Underhållbarhet}
\begin{itemize}
\item Risk - Systemet som levereras kan vara svårt att underhålla.
Lösning - Se över koden för att få det mera underhållningsbar.
Alt. Lösning - Koda moduler där vissa delar t.ex. UI är mkt. välkommenterad och lätt att förändra.
\end{itemize}

\subsection{Requirements / Krav}
\begin{itemize}
\item Risk - Kunden ger luddiga krav och blir sedan inte nöjd med produkten då den inte blev som han "hade tänkt sig".
Lösning - Prata med kunden om om strukturering av programvaran. Typ.
\item Risk - Beställarens krav stämmer inte överens med slutanvändarens krav.
Lösning -Intervjuer, prototyp-test eller annat samarbete med slutanvändaren under utvecklingen.
\item Risk - Kraven är ouppnåeliga för oss under detta tidsbegränsade projekt.
Lösning - Förklara för kunden vad vi tror att vi kan och inte kan slutföra under detta projekt. Sätta upp ny kravlista.
\end{itemize}

\subsection{Usability / Användbarhet}
\begin{itemize}
\item Risk - Produkten blir inte lika användbar som var tänkt från början.
Lösning - Iterera fram bättre och bättre GUI och gör användartester mellan varje iteration. Följa guider om att göra bättre GUI etc.
\end{itemize}

\subsection{Portability / Flyttbarhet}
(Osäker på om detta är något vi behöver oroa oss över)
\begin{itemize}
\item Risk - Produkten görs för att fungera bra på olika operativsystem och det slutar med att det inte funkar på något.
Lösning - Se över om det verkligen är nödvändigt att fungera på alla operativsystem.
...ability of a program (or software system) to execute properly on multiple hardware platforms.
(Känns konstigt att under Portability ha en lösning på ett problem som strider mot själva rubriken?)
\item Risk - Produkten görs endast för ett operativsystem men det visar sig sedan att det hade behövts för flera.
Lösning - Använd högnivåspråk vid programmering och undvik plattformsberoende paket etc.
\item Risk - Produkten blir för hårdvarukrävande så att det inte går att använda mjukvaran på den tänkta hårdvaran.
Lösning - Försök att begränsa produkten till det mest väsentliga och fokusera på funktionalitet och användbarhet.
\end{itemize}

\subsection{Testability / Tester}
\begin{itemize}
\item Risk - Tester kan inte ske innan implementation av programvaran.
Lösning - Läs på om gränssnitt, protokoll, dokumentation, använd driver? Testa små delar åt gången?
Compability / Kompatibilitet

\item Risk - Komponenter vi använder oss av fungerar i någon mån inte som tänkt.
Lösning - Byta komponenter eller se till att kolla upp så att samtliga komponenter samarbetar bra.
\end{itemize}

\subsection{Generella risker}
\begin{itemize}
\item Risk - En nyckelperson i projektet blir utbytt eller tvingas sluta.
Lösning - Dokumentera rikligt så någon annan lätt kan ta över personens arbete
\item Risk - Kod blir rörigt och svårläst
Lösning - Sätt upp en kod och kommenteringsstandard som hela gruppen följer.
\item Risk - Vi får ingen kontakt med kunden
Lösning - Jobba på utifrån de krav vi har och försök ta kontakt med kunden
\end{itemize}

\end{document}

