% Standardinkluderingsfil
%
%  untitled
%
%  Created by David Granqvist on 2008-09-08.
%  Modified by Martin Erola
%

% Set document format/class
\documentclass[a4paper,twoside]{article}

%%%%%%%%%%%%%%%%%%%
% Include packages
%
\usepackage[utf8]{inputenc}   % Use utf-8 encoding for foreign characters
\usepackage[swedish]{babel}   % Support for swedish letters
\usepackage{fullpage}         % Setup for fullpage use
\usepackage{fancyhdr}         % Running Headers and footers
\usepackage{boxedminipage}    % Surround parts of graphics with box
\usepackage{listings}         % Package for including code in the document
\usepackage{ifpdf}            % Recommended way for checking for PDFLaTeX:
\usepackage{tabularx}         % Tabeller med automatisk stretch
% \usepackage[nofancy]{svninfo} % Extract Subversion info about the file
% \usepackage{color}          % Color
% \usepackage{lastpage}       % Total page count

% Graphics
\ifpdf
\usepackage[pdftex]{graphicx}
\else
\usepackage{graphicx}
\fi

%%%%%%%%%%%%%%%%%%%%%%%%%%%%%%%%%%%%%%%%%%%%%%%%%%%%%%%%%%
% Uncomment some of the following if you use the features
%

% Multipart figures
%\usepackage{subfigure}

% More symbols
%\usepackage{amsmath}
%\usepackage{amssymb}
%\usepackage{latexsym}

% If you want to generate a toc for each chapter (use with book)
% \usepackage{minitoc}

%%%%%%%%%%%%%%%%%%%%
% Document settings
%

% Header
\pagestyle{fancy}
% Sätter en marginal mellan header och (ovanstående?) text %
\setlength\headsep{10pt}
% Sätter höjden på headern
\setlength{\headheight}{32pt}

% Sätter styckesinställningar
\setlength\parindent{0pt}
\setlength\parskip{10pt}



\ifpdf
  \DeclareGraphicsExtensions{.pdf, .jpg, .tif, .png}
  \pdfinfo{            
    /Title  (Architecture Notebook)
    /Author (PUM-grupp 1)
  }
\else
  \DeclareGraphicsExtensions{.eps, .jpg}
\fi

\title{Distribuerad wiki \\ Architecture Notebook}
\author{PUM-grupp 1}
\date{\today}

\begin{document}

\maketitle

\thispagestyle{empty}

\newpage

{\centering \Large{Dokumenthistorik\\}}

\vspace{10pt}
\begin{tabularx}{\textwidth}{ |l|l|X|l|l| }
  \hline
    \textbf{version} & \textbf{datum} & \textbf{utförda ändringar} & \textbf{utförda av} & \textbf{granskad} \\
	\hline 
  0.1 & 2009-02-12 &  Ett första utkast  & Alla & Alla   \\
  \hline
\end{tabularx}

\newpage

\setcounter{tocdepth}{2}
\tableofcontents
\newpage

\section{Syfte}
Detta dokument syftar till att beskriva de filosofier och motiveringar som står till grund för det arkitekturella ramverket för projektet.
\section{Arkitekturella mål och filosofi}
Arkitekturen i projektet styrs främst av de krav som har stipulerats.
\section{Antaganden och beroenden}
I denna sektion nämns de antaganden och beroenden som finns och som tillsammans bygger upp en bild över hur programvaran kommer byggas.
\begin{itemize}
\item Systemet ska fungera på Linux
\item Systemet ska fungera decentraliserat
\item Systemet ska kunna kommunicera via nätverk
\item Systemet ska lagra data lokalt
\end{itemize}
\section{Arkitekturellt signifikanta krav}
Här kommer referenser till krav i andra dokument att finnas.
\section{Beslut, begränsningar och motiveringar}
Här är en lista med beslut, begränsningar och tillhörande motiveringar.
\begin{itemize}
\item Bazaar används som distributionssystem, eftersom det är lätt att integrera.
\end{itemize}
\section{Arkitekturen}
I den här sektionen beskrivs arkitekturen övergripande, med mer detaljerade beskrivningar i respektive moduls sektion.

Kommunikationsmodulen är central i systemet, och tillhandahåller kommunikationsmöjligheter mellan klienten, distribueringsundersystemet, redigeringsverktyget och andra klienter.

För kommunikation klienter emellan används TLS över TCP/IP. En av klienterna ansluter till den andra och TLS-handskakning och autenticering genomförs.

Det underliggande distributionssystemet som vi använder är Bazaar. För att kunna slå ihop två personers repositorier kommunicerar den ena klientens Bazaar-klient med den andras Bazaar-server över TLS-anslutningen via kommunikationsmodulen.

En filhanterare agerar som lager mellan redigeringsverktyget och kommunikationsmodulen. Denna tillhandahåller en modell för att hålla filhanteraren och kommunikationsmodulen underrättade om händelser i andra delar av systemet. Filhanteraren arbetar också mot arbetsmappen och tillhandahåller metoder för att modifiera filer mot denna. Kommunikationsmodulen via distributionssystemet tillhandahåller metoder för att distribuera modifieringar som sker i arbetsmappen.
\subsection{Kommunikationsmodulen}
Kommunikationsmodulens uppgift består i att upprätta kommunikation mellan klienter och att arbeta som ett mellanled mot distributionstekniken.

För att möjliggöra distribution över TLS-anslutningen tillhandahåller kommunikationsmodulen emulerade portar, som från den lokala datorn går via TLS-anslutningen till mottagarklienten. Detta innebär att distribueringen kan ske dubbelriktat även om den ena av klienterna ligger bakom en brandvägg.


\subsection{Redigeringsmodulen}
Redigeringsmodulen tillhandahåller ett användargränssnitt som kan användas för att visa och redigera sidor. 
\section{Arkitekturella mekanismer}
I denna sektion nämns de olika mekanismerna som driver programvaran. Dessa kommer få mer detaljerade beskrivningar under senare iterationer.
\subsection{Säkerhet}
Som en del av programvarans specifikation nämns säkerhet mellan klienterna. Informationens integritet ska vidbehållas, alltså skyddas mot olovlig ändring vid informationsöverföringar samt vid lagring på hårddisk.

\begin{itemize}
\item Mellan vilka arkitekturella moduler och i vilka instanser är säker kommunikation nödvändig?
\item Vilken data är relevant att kryptera eller verifiera?
\item Hur stort fokus ska läggas på datasäkerhet?
\item Strider åtgärderna för ökad sökerhet mot andra krav såsom tillgänglighet?
\end{itemize}
\subsection{Felsökning}
Programvaran kommer med största sannolikhet behöva felsökas under utvecklingstiden. Felsökning kommer vara relevant i många delar av programvaran.

Felsökning kan ske passivt och aktivt. Aktivt genom att till exempel stega kod under körning. Passivt genom loggning.

Frågor rörande felsökning som kommer utvecklas senare i projektet följer nedan.
\begin{itemize}
\item I vilka instanser kommer felsökning att vara relevant?
\item Vilka felsökningsmetoder och verktyg skall användas?
\item Hur stora datamängder behöver loggas?
\item Hur ska loggning ske utan att användarens integritet kränks?
\end{itemize}
\subsection{Lagring}
Lagringsmodulen är den mekanism som sköter den lokalt lagrade datan hos alla klienter.

För en wiki krävs att en viss mängd data lagras om varje artikel. Artikelhistorik är i vissa fall essentiell men kan vid stora datamängder vara en börda att distribuera.

Lagring av Wiki-data tar upp en viss mängd lokalt utrymme på ett lokalt medium, förslagsvis på en hårddisk. När det blir aktuellt att synkronisera data med andra klienter så måste komminikationsmodulen ha tillgång, läsrättigheter och skrivrättigheter till lokalt lagrad Wiki-data.

Relevanta frågor som rör lagring följer.
\begin{itemize}
\item Hur mycket utrymme krävs på den lokala hårddisken?
\item Vad händer om hårddiskutrymmer tar slut?
\item Vad händer om en delmängd av allt data tas bort eller blir korrupt?
\end{itemize}
\subsection{Distribuering}
Distribueringsmodulen är den som sköter synkroniseringen klienterna emellan. Den ser till att uppdateringar sprider sig över nätverket till alla andra klienter som arbetar online.

För att möjliggöra distribution håller distributionsmodulen koll på alla ändringar som skett, tar emot ändringar från andra klienter och skickar iväg egna ändringar. Om en konflikt sker ser distributionsmodulen till att den ansvarige skribenten blir meddelad om konflikten samtidigt som den gamla, konfliktfria versionen, görs tillgänglig för övriga användare. Detta för att en tredje part inte ska påverkas av en redigeringskonflikt mellan två andra användare.

Relevanta frågor som rör distribuering följer.
\begin{itemize}
\item Hur ofta skall synkronisering ske?
\item Vad händer om en användare bryter en pågående distribution?
\end{itemize}
\subsection{Kommunikation}
Kommunikationsmodulen är en delmängd av distrubueringsmodulen. Den ser till att data felfritt kan kommuniceras mellan klienterna.

\begin{itemize}
\item Vilken kommunikationsmetod skall väljas?
\item Hur skall felfri kommunikation säkerställas?
\item Ska kommunikation ske krypterat?
\end{itemize}
\subsection{Redigering}
Redigeringsmodulen finns lokalt hos klienten och tillåter användaren att redigera wiki-sidor via ett grafiskt gränssnitt.

När användaren redigerat klart sparas ändringen till det lokala versionshanteringssystemet och distributionsmodulen informeras om uppdateringen.
\subsection{Bläddring}
Bläddringsmodulen går ut på att visa wiki-sidor för användaren. Användaren kan använda bläddringsmodulen för att utforska sin lokala kopia av den distribuerade wikin. När ändringar sker i sidor som är i vy ser bläddringsmodulen till att uppdateringarna presenteras till användaren på ett enkelt och tydligt sätt.
\section{Huvudabstraktioner}
Här definieras de abstraktioner som används för att beskriva systemet.
\subsection*{Användare}
En användare är en person i systemet. 
\subsection*{Klient}
En klient är en instans av programvaran som körs aktivt på någon plattform. En användare kan inneha en klient.
\subsection*{Nyckel}
En datamängd som används för signering, verifiering och säker kommunikation.
\section{Arkitekturellt ramverk}
Det arkitekturella ramverket definierar hur systemet är sammansatt. Ramverket definieras bättre under senare delar av projektet.
\end{document}
