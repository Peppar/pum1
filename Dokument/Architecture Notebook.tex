% Standardinkluderingsfil
%
%  untitled
%
%  Created by David Granqvist on 2008-09-08.
%  Modified by Martin Erola
%

% Set document format/class
\documentclass[a4paper,twoside]{article}

%%%%%%%%%%%%%%%%%%%
% Include packages
%
\usepackage[utf8]{inputenc}   % Use utf-8 encoding for foreign characters
\usepackage[swedish]{babel}   % Support for swedish letters
\usepackage{fullpage}         % Setup for fullpage use
\usepackage{fancyhdr}         % Running Headers and footers
\usepackage{boxedminipage}    % Surround parts of graphics with box
\usepackage{listings}         % Package for including code in the document
\usepackage{ifpdf}            % Recommended way for checking for PDFLaTeX:
\usepackage{tabularx}         % Tabeller med automatisk stretch
% \usepackage[nofancy]{svninfo} % Extract Subversion info about the file
% \usepackage{color}          % Color
% \usepackage{lastpage}       % Total page count

% Graphics
\ifpdf
\usepackage[pdftex]{graphicx}
\else
\usepackage{graphicx}
\fi

%%%%%%%%%%%%%%%%%%%%%%%%%%%%%%%%%%%%%%%%%%%%%%%%%%%%%%%%%%
% Uncomment some of the following if you use the features
%

% Multipart figures
%\usepackage{subfigure}

% More symbols
%\usepackage{amsmath}
%\usepackage{amssymb}
%\usepackage{latexsym}

% If you want to generate a toc for each chapter (use with book)
% \usepackage{minitoc}

%%%%%%%%%%%%%%%%%%%%
% Document settings
%

% Header
\pagestyle{fancy}
% Sätter en marginal mellan header och (ovanstående?) text %
\setlength\headsep{10pt}
% Sätter höjden på headern
\setlength{\headheight}{32pt}

% Sätter styckesinställningar
\setlength\parindent{0pt}
\setlength\parskip{10pt}



\ifpdf
  \DeclareGraphicsExtensions{.pdf, .jpg, .tif, .png}
  \pdfinfo{            
    /Title  (Architecture Notebook)
    /Author (PUM-grupp 1)
  }
\else
  \DeclareGraphicsExtensions{.eps, .jpg}
\fi

\title{Distribuerad wiki \\ Architecture Notebook}
\author{PUM-grupp 1}
\date{\today}

\begin{document}

\maketitle

\thispagestyle{empty}
\newpage
\section{Syfte}
Detta dokument syftar till att beskriva de filosofier och motiveringar som står till grund för det arkitekturella ramverket för projektet.
\section{Arkitekturella mål och filosofi}
Arkitekturen i projektet drivs till stor del av de krav som har stipulerats.
\section{Antaganden och beroenden}
\begin{itemize}
\item Systemet skall fungera på Linux
\item Systemet skall fungera decentralisera
\item Systemet skall kommunicera via nätverk
\end{itemize}
\section{Arkitekturellt signifikanta krav}
Här kommer referenser till krav i andra dokument att finnas.
\section{Beslut, begrängningar och motiveringar}
\begin{itemize}
\item Beslut eller begränsning med motivering
\item Beslut eller begränsning med motivering
\item Beslut eller begränsning med motivering
\end{itemize}
\section{Arkitekturella mekanismer}
\subsection{Distribueringsmekanismen}
\subsection{Redigeringsmekanismen}
\subsection{Bläddringsmekanismen}
\section{Huvudabstraktioner}
\section{Arkitekturellt ramverk}
\section{Arkitekturella vyer}
\subsection{Logisk vy}
\subsection{Operationell vy}
\subsection{Användarfall}
\end{document}
