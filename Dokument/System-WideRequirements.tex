% Standardinkluderingsfil
%
%  untitled
%
%  Created by David Granqvist on 2008-09-08.
%  Modified by Martin Erola
%

% Set document format/class
\documentclass[a4paper,twoside]{article}

%%%%%%%%%%%%%%%%%%%
% Include packages
%
\usepackage[utf8]{inputenc}   % Use utf-8 encoding for foreign characters
\usepackage[swedish]{babel}   % Support for swedish letters
\usepackage{fullpage}         % Setup for fullpage use
\usepackage{fancyhdr}         % Running Headers and footers
\usepackage{boxedminipage}    % Surround parts of graphics with box
\usepackage{listings}         % Package for including code in the document
\usepackage{ifpdf}            % Recommended way for checking for PDFLaTeX:
\usepackage{tabularx}         % Tabeller med automatisk stretch
% \usepackage[nofancy]{svninfo} % Extract Subversion info about the file
% \usepackage{color}          % Color
% \usepackage{lastpage}       % Total page count

% Graphics
\ifpdf
\usepackage[pdftex]{graphicx}
\else
\usepackage{graphicx}
\fi

%%%%%%%%%%%%%%%%%%%%%%%%%%%%%%%%%%%%%%%%%%%%%%%%%%%%%%%%%%
% Uncomment some of the following if you use the features
%

% Multipart figures
%\usepackage{subfigure}

% More symbols
%\usepackage{amsmath}
%\usepackage{amssymb}
%\usepackage{latexsym}

% If you want to generate a toc for each chapter (use with book)
% \usepackage{minitoc}

%%%%%%%%%%%%%%%%%%%%
% Document settings
%

% Header
\pagestyle{fancy}
% Sätter en marginal mellan header och (ovanstående?) text %
\setlength\headsep{10pt}
% Sätter höjden på headern
\setlength{\headheight}{32pt}

% Sätter styckesinställningar
\setlength\parindent{0pt}
\setlength\parskip{10pt}



\ifpdf
  \DeclareGraphicsExtensions{.pdf, .jpg, .tif, .png}
  \pdfinfo{            
    /Title  (System Wide Requirements)
    /Author (PUM-grupp 1)
  }
\else
  \DeclareGraphicsExtensions{.eps, .jpg}
\fi

\title{System-WideRequirements}
\author{PUM-grupp 1}
\date{\today}

\begin{document}

\maketitle\thispagestyle{empty}

\newpage

{\centering \Large{Dokumenthistorik\\}}

\vspace{10pt}
\begin{tabularx}{\textwidth}{ |l|l|X|l|l| }
  \hline
    \textbf{version} & \textbf{datum} & \textbf{utförda ändringar} & \textbf{utförda av} & \textbf{granskad} \\
	\hline 
  0.1 & 2009-02-12 &  Dokumentet är påbörjat, all engelsk text kommer från OpenUP & &   \\
  \hline
  1.0 & 2009-03-03 & Dokumentet är klart & & \\
  \hline
\end{tabularx}

\newpage

\setcounter{tocdepth}{2}
\tableofcontents
\newpage

\section{Introduktion}
En stor del av vårt arbetet för gruppen kommer vara integrera och paketera olika programvaror. Så förutom utvecklingen av wiki:n så är det många krav som inte användaren kommer möta direkt.

\section{Systemomfattande funktionskrav}
Här redogörs funktionskrav som är systemomfattande. Krav som inte fångas upp i användarfallen. Det kan röra protokoll och arkitektur 

\section{Systemegenskaper}
\subsection{Användbarhet}
Användaren ska känna att vårt verktyg gör dom effektivare. Det ska mätas och utvärderas med en användargrupp. 

\subsection{Tillförlitlighet} 
Applikationens tillförlitlighet ligger i hur datan hanteras. Användaren ska kunna lita på att den data som användaren lägger till kommer vara detsamma efter den kommer ut. Det kan mätas med tester på skillnaden mellan in och utdata. En stor del ansvaret kommer ligga på den versionshanterare som används så därför bör även den testas och utvärderas.

\subsection{Prestanda}
Systemet är inte prestandakrävande, bandbreddskrävande eller kräver snabba svarstider. Det viktiga är att användaren kan spara till den lokala datorn. Det viktiga är att fel som sker inte stör användarvänligheten. Uppstart av applikationen ska max ta 3 sekunder.

\subsection{Support och påbyggnadsmöjligheter} %Behöver hjälp med översätta rubriken
Applikationen ska byggas modulbaserat. Det är viktigt att olika delar kan uppdateras försig just för att vårt applikation bygger på flera olika program. Systeminstallationen ska gå via en pakethanterar i linux.  Hur det sker på andra plattformar är en senare fråga.

%This section indicates any requirements that will enhance the supportability or maintainability of the system being built, including adaptability and upgrading, compatibility, configurability, scalability and requirements regarding system installation, level of support and maintenance.

\section{Systemgränssnitt}

\subsection{Användargränsnitt}

I detta avsnitt behandlas krav rörande användargränssnittet.

\subsubsection{Utseende och känsla}
Applikationen ska i redigeringsläget kännas som en ordbehandlare. Gränssnittet ska upplevas som effektivt och enkelt. Hämtning av data och synkronisering mot andra användare ska ske i bakgrunden utan att påverka användarens arbete.

\subsubsection{Layout och navigering}
Användaren ska kunna nå stora delar av systemet med endast ett fåtal navigeringar. Navigeringen ska vara intuitiv och efterlikna operativsystemet som programmet körs på. Beteendemönster ska kännas igen från andra textredigeringsprogram och Webbläsare. Applikationen har främst två lägen, ett för visning och ett för redigering och från varje visningssida som man har rätt att redigera ska man enkelt kunna byta till redigeringsläge. När man redigerat klart och godkänner ändringarna så återvänder man till visningssläget.

%\subsubsection{Följdriktighet} %Consistency
%Consistency in the user interface enables users to predict what will happen. This section states requirements on the use of mechanisms to be employed in the user interface. This applies both within the system and with other systems and can be applied at different levels: navigation controls, screen areas sizes and shapes, placements for entering / presenting data, terminology.

\subsubsection{Användaranpassning och kundanpassning}
Vi kommer använda det bästa verktyget för kundanpassning, open source. Koden är öppen så att det är fritt för kunden att göra förändringar och förbättringar.

\subsection{Gränssnitt mot externa system}
Inga externa system finns som kräver gränssnitt, förutom kommunikationen två klienter emellan.

\subsubsection{Mjukvarugränssnitt}
För extern kommunikation distributionssystemet Bazaar att användas. Bazaar tillhandahåller en komplett uppsättning funktioner för styrning av systemet. Bazaar är skrivet i Python och är därför särskilt enkel att integrera.

För användargränssnitt kommer wxPython att användas. wxPython är ett bibliotek för att tillhandahålla grafiska användargränssnitt till Pythonprogram och kan anpassa sig till flera operativsystem, bland annat det som vi utvecklar till.

\subsubsection{Hårdvarugränsnitt}
I begränsningen av systemet ingår att vi ska kunna använda det operativsystemet som vi utvecklar mot, varför vi antar full funktionalitet mot systemfunktioner så som filhantering. Python tillhandahåller abstraktioner för den grundläggande operativsystemsfunktionaliteten.

\subsubsection{Kommunikationsgräsnitt}
%Här kan jag tänka mig en mer avancerad förklaring hur vi har tänkt oss p2p, krypetering osv
Kommunikation mellan klienter sker med hjälp av TLS över TCP/IP. För detta krävs tillgång till ett nätverksgrässnitt på det lokala systemet samt en Python-specifik kommunikationsmodul som tillhandahåller TLS över TCP/IP. All form av binär data kan skickas som en ström över TLS.

För att underlätta kommunikationen mellan två klienters distributionssystem kommer vi tunnla datatrafiken dem emellan genom vår TLS-länk. Detta är särskilt lämpligt om en användare ligger bakom en brandvägg, där distributionssystemet i normala fall endast skulle kunna verka enkelriktat.

Klienter kommunicerar över nätverket med ett gränssnitt som möjliggör automatisk och komplett propagering av ändringar, administration av medlemsskap i nätverket och autenticering av medlemmar.

%Describe any communications interfaces to other systems or devices such as local area networks, remote serial devices, and so on.

\section{System begränsningar}
Applikationen kommer skrivas i python. Det finns flera tredjeparts delar bland annat Bazaar och wxPython. Applikationen kommer släppas på linux till att börja med men troligtvis kommer en konvertering till flera system vara möjligt.

\section{Övrig systeminformation} %Vet inte om det finns en bätte översättning

\subsection{Licenskrav}
Projektet ska släppas som open source. Vilka licenser vi väljer beror på om vi utnyttjar redan befintlig kod med restriktiva licenser.

\subsection{Rättsligt, upphovsrätt och andra meddelanden}
Det ligger som förslag att lägga programmet på en idell förening. Detta för att förenkla upphovsrättsliga frågor när programmet ska vidareutvecklas. Annars kommer det hamna på oss som utvecklare och vår kund. 

%\subsection{Applicable Standards}
%This section describes by reference any applicable standards and the specific sections of any such standards that apply to the system being described. For example, this could include legal, quality and regulatory standards, industry standards for usability, interoperability, internationalization, operating system compliance, and so forth.

\section{System Dokumentation}
System dokumentationen kommer finnas på projektets hemsida i form av en wiki där användarna kan medverka. Dokumentering av kod och övrig dokumentation för utvecklarna bör följa med koden. Användarmanuler kommer följa med när man installerar programmet via en paktethanterare. % Vem är ansvarig för punken?

\end{document}
