% Standardinkluderingsfil
%
%  untitled
%
%  Created by David Granqvist on 2008-09-08.
%  Modified by Martin Erola
%

% Set document format/class
\documentclass[a4paper,twoside]{article}

%%%%%%%%%%%%%%%%%%%
% Include packages
%
\usepackage[utf8]{inputenc}   % Use utf-8 encoding for foreign characters
\usepackage[swedish]{babel}   % Support for swedish letters
\usepackage{fullpage}         % Setup for fullpage use
\usepackage{fancyhdr}         % Running Headers and footers
\usepackage{boxedminipage}    % Surround parts of graphics with box
\usepackage{listings}         % Package for including code in the document
\usepackage{ifpdf}            % Recommended way for checking for PDFLaTeX:
\usepackage{tabularx}         % Tabeller med automatisk stretch
% \usepackage[nofancy]{svninfo} % Extract Subversion info about the file
% \usepackage{color}          % Color
% \usepackage{lastpage}       % Total page count

% Graphics
\ifpdf
\usepackage[pdftex]{graphicx}
\else
\usepackage{graphicx}
\fi

%%%%%%%%%%%%%%%%%%%%%%%%%%%%%%%%%%%%%%%%%%%%%%%%%%%%%%%%%%
% Uncomment some of the following if you use the features
%

% Multipart figures
%\usepackage{subfigure}

% More symbols
%\usepackage{amsmath}
%\usepackage{amssymb}
%\usepackage{latexsym}

% If you want to generate a toc for each chapter (use with book)
% \usepackage{minitoc}

%%%%%%%%%%%%%%%%%%%%
% Document settings
%

% Header
\pagestyle{fancy}
% Sätter en marginal mellan header och (ovanstående?) text %
\setlength\headsep{10pt}
% Sätter höjden på headern
\setlength{\headheight}{32pt}

% Sätter styckesinställningar
\setlength\parindent{0pt}
\setlength\parskip{10pt}



\ifpdf
  \DeclareGraphicsExtensions{.pdf, .jpg, .tif, .png}
  \pdfinfo{            
    /Title  (System Wide Requirements)
    /Author (PUM-grupp 1)
  }
\else
  \DeclareGraphicsExtensions{.eps, .jpg}
\fi

\title{System Wide Requirements}
\author{PUM-grupp 1}
\date{2009-02-12}

\begin{document}

\maketitle\thispagestyle{empty}

\newpage

{\centering \Large{Dokumenthistorik\\}}

\vspace{10pt}
\begin{tabularx}{\textwidth}{ |l|l|X|l|l| }
  \hline
    \textbf{version} & \textbf{datum} & \textbf{utförda ändringar} & \textbf{utförda av} & \textbf{granskad} \\
	\hline 
  1.0 & 2009-02-12 &  Första versionen klar för inlämning  & Alla & Alla   \\
  \hline
\end{tabularx}

\newpage

\setcounter{tocdepth}{2}
\tableofcontents
\newpage

\section{Introduktion}

\section{Systemomfattande Funktionskrav}

Statement of system-wide functional requirements, not expressed as use cases. Examples include auditing, authentication, printing, reporting.

\section{System Kvaliteter}
\subsection{Användbarhet}
Användaren ska nå stora delar av systemet på ett fåtal
\subsection{Tillförlitlighet}

Reliability includes the product and/or system's ability to keep running under stress and adverse conditions. Specify requirements for reliability acceptance levels, and how they will be measured and evaluated. Suggested topics are availability, frequency of severity of failures and recoverability.

\subsection{Prestanda}

The performance characteristics of the system should be outlined in this section. Examples are response time, throughput, capacity and startup or shutdown times.

\subsection{Supportability}

This section indicates any requirements that will enhance the supportability or maintainability of the system being built, including adaptability and upgrading, compatibility, configurability, scalability and requirements regarding system installation, level of support and maintenance.

\section{Systemgränssnitt}

Interface Requirements are part of the + in the FURPS+ classification of supporting requirements. Define the interfaces that must be supported by the application. It should contain adequate specificity, protocols, ports and logical addresses, and so forth, so that the software can be developed and verified against the interface requirements.

\subsection{Användargränsnitt}

Describe the user interfaces that are to be implemented by the software. The intention of this section is to state requirements relating to the interface. Interface design may overlap the requirements gathering process.

\subsubsection{Utseende och känsla}
Applikationen ska kännas som en lättare ordbehandlare. Gränsnittet ska förmedla effektivitet och enkelhet.

\subsubsection{Layout och navigeringskrav}
Användaren ska nå stora delar av systemet via ett fåtal navigeringar. 

\subsubsection{Konsekvens}

Consistency in the user interface enables users to predict what will happen. This section states requirements on the use of mechanisms to be employed in the user interface. This applies both within the system and with other systems and can be applied at different levels: navigation controls, screen areas sizes and shapes, placements for entering / presenting data, terminology.

\subsubsection{Användaranpassning och and kundanpassning krav}
y
Requirements on content that should automatically displayed to users or available based on user attributes. Sometimes users allowed to customize the content displayed or to personalize displayed content.

\subsection{Interfaces to External Systems or Devices}

Are there any external systems with which this system must interface? Are there any constraints on the nature of the interface between this system and any external system, such as the format of data passed between these systems, and any particular protocol used? Consider both provided and required interfaces.

\subsubsection{Software Interfaces}

This section describes software interfaces to other components of the software system. These may be purchased components, components reused from another application or components being developed for subsystems outside of the scope of this SRS, but with which this software application must interact.

\subsubsection{Hardware Interfaces}

This section defines any hardware interfaces that are to be supported by the software, including logical structure, physical addresses, expected behavior, and so on.

\subsubsection{Communications Interfaces}

Describe any communications interfaces to other systems or devices such as local area networks, remote serial devices, and so on.

\section{Business Rules}

Business rules are statements that define or constrain some aspect of the business. Business rules are often represented as production rules when they are meant to be directly executed by an IT System: a production rule is an independent statement of programming logic that specifies the execution of one or more actions in the case that its conditions are satisfied. Production Rules define the operation semantic for the system in a technologic independent way. They constrain the behavior expressed in system use cases.
Organize this document on rule classes, a high level grouping of candidate or actual rules about one business concept with a specific kind of logic processing, example: Driver Risk Assessment Rules or Customer Validation Rules.


\subsection{Rule class name}

\subsubsection{Rule name and ID}

The description defines the rule. It can be made in natural language typically following a decision table or a pattern like:  if [condition-list] then [action-list], example: 
If there are at least 3 items of the same type in the customer shopping cart and each item’s value is greater than 30 USD then give to the customer a voucher whose value is 10 percent of the cheapest item.


\section{System Constraints}

Constraints are part of the + in the FURPS+ classification of supporting requirements. Describe any design; implementation or deployment constraints on the system being built that have been mandated and must be adhered to. Examples include software implementation languages, prescribed use of developmental tools, third-party components or class libraries, platform support, resource limits and requirements on the shape, size or weight of the resulting hardware housing the system.

\section{System Compliance}

\subsection{Licensing Requirements}

Define any licensing enforcement requirements or other usage restriction requirements that are to be exhibited by the software.

\subsection{Legal, Copyright, and Other Notices}

This section describes any necessary legal disclaimers, warranties, copyright notices, patent notice, wordmark, trademark, or logo compliance issues for the software.

\subsection{Applicable Standards}

This section describes by reference any applicable standards and the specific sections of any such standards that apply to the system being described. For example, this could include legal, quality and regulatory standards, industry standards for usability, interoperability, internationalization, operating system compliance, and so forth.

\section{System Documentation}

Describes the requirements, for on-line user documentation, help systems, help about notices, and so on. Set expectations for the documentation and to identify who will be responsible for creating it.

\end{document}
