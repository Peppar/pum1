% Standardinkluderingsfil
%
%  untitled
%
%  Created by David Granqvist on 2008-09-08.
%  Modified by Martin Erola
%

% Set document format/class
\documentclass[a4paper,twoside]{article}

%%%%%%%%%%%%%%%%%%%
% Include packages
%
\usepackage[utf8]{inputenc}   % Use utf-8 encoding for foreign characters
\usepackage[swedish]{babel}   % Support for swedish letters
\usepackage{fullpage}         % Setup for fullpage use
\usepackage{fancyhdr}         % Running Headers and footers
\usepackage{boxedminipage}    % Surround parts of graphics with box
\usepackage{listings}         % Package for including code in the document
\usepackage{ifpdf}            % Recommended way for checking for PDFLaTeX:
\usepackage{tabularx}         % Tabeller med automatisk stretch
% \usepackage[nofancy]{svninfo} % Extract Subversion info about the file
% \usepackage{color}          % Color
% \usepackage{lastpage}       % Total page count

% Graphics
\ifpdf
\usepackage[pdftex]{graphicx}
\else
\usepackage{graphicx}
\fi

%%%%%%%%%%%%%%%%%%%%%%%%%%%%%%%%%%%%%%%%%%%%%%%%%%%%%%%%%%
% Uncomment some of the following if you use the features
%

% Multipart figures
%\usepackage{subfigure}

% More symbols
%\usepackage{amsmath}
%\usepackage{amssymb}
%\usepackage{latexsym}

% If you want to generate a toc for each chapter (use with book)
% \usepackage{minitoc}

%%%%%%%%%%%%%%%%%%%%
% Document settings
%

% Header
\pagestyle{fancy}
% Sätter en marginal mellan header och (ovanstående?) text %
\setlength\headsep{10pt}
% Sätter höjden på headern
\setlength{\headheight}{32pt}

% Sätter styckesinställningar
\setlength\parindent{0pt}
\setlength\parskip{10pt}



\ifpdf
  \DeclareGraphicsExtensions{.pdf, .jpg, .tif, .png}
  \pdfinfo{            
    /Title  (Projektplan)
    /Author (PUM-grupp 1)
  }
\else
  \DeclareGraphicsExtensions{.eps, .jpg}
\fi

\title{Projektplan}
\author{PUM-grupp 1}
\date{\today}

\begin{document}

\maketitle\thispagestyle{empty}

\newpage

\section{Inledning}
Innehållet i det här dokumentet beskriver hur projektet är organiserat, hur det skall genomföras och vilka personer som är delaktiga i projektet. 
\section{Bakgrund}
Detta projekt genomförs i projektkursen TDDD09 Programutvecklingsprojekt i ett helhetsperspektiv på Linköpings Universitet. Projektet genomförs under vårterminen 2009.
\section{Syfte och mål}
Syftet med projektet är att lära oss mjukvaruframställning på projektform samt att få en inblick i hur det går till att framställa mjukvara i arbetslivet. Målet med projektet är att genomföra ett lyckat projekt och ge kunden en mjukvara som denne är nöjd med och kan dra nytta av.
\section{Projektöversikt}
Nedan följer en översiktlig beskrivning av projektet, samt de personer som har intresse i projektet.
Inledande beskrivning utav projektet
[kort översiktlig beskrivning av vad vårt projekt går ut på]
\section{Projektmedlemmar}
Projektgruppen består av följande medlemmar:
\begin{itemize}
\item Mikael Waernér - Projektledare
\item Erik Thorselius - Analytiker
\item Oskar Holstensson - Arkitekt 
\item Victor Ortman - Utvecklare
\item Martin Pettersson - Testare
\item Linus Dunkers – Kvalitetssamordnare
\end{itemize}
\section{Beskrivning av rollerna}
Nedan följer en beskrivning utav de roller som innehas av medlemmar i projektgruppen
\subsection*{Projektledare}
Projektledaren är den person som har huvudansvaret för planeringen i projektet. Han ser till så att projektgruppen är fokuserad på målen i projektet. Projektledaren ser till så att projektgruppen genomför ett projekt med ett resultat som kunden är nöjd med.
\subsection*{Analytiker}
Analytikerns jobb är att fånga in kunden och målgruppens behov och förstå de problem som måste lösas. Analytiker bör också se de möjligheter som eventuella problem kan medföra.
\subsection*{Arkitekt}
Arkitekten ansvarar för mjukvarans arkitektur, i det ingår att ta tekniska beslut om den överläggande designen och implementationen av systemet. Arkitekten måste balansera kundens behov med tekniska risker för att skapa en design som är genomförbar och effektiv att följa och validera.
\subsection*{Utvecklare}
Utvecklaren ansvarar för att systemet blir utvecklat enligt arkitekturen. I det ingår bland annat att implementera enhetstestning och integrera olika komponenter.
\subsection*{Testare}
Testaren ansvarar för att idientifiera de tester som är nödvändiga för produkten. Han ansvarar också för implementationen och ledandet av dessa tester. Testaren skall också logga testresultaten och förstå innebörden av dessa för systemet.
\subsection*{Kvalitetssamordnare}
[beskrivning]
\subsection*{Kunden}
[beskrivning]
\section{Liknande projekt}
[kanske är onödigt]
\section{Projektadministration}
[alt Projektgenomförande eller någonting annat ]
Denna del handlar om hur projektet skall genomföras.
\section{Projektprocessen OpenUP}
[översätt och lägg till]
OpenUP is a lean Unified Process that applies iterative and incremental approaches within a structured lifecycle. OpenUP embraces a pragmatic, agile philosophy that focuses on the collaborative nature of software development. It is a tools-agnostic, low-ceremony process that can be extended to address a broad variety of project types.
\section{Fasplan}
Nedan är en beskrivning av de faser som ingår i projektet.
\subsection*{Inception}
Inception fasen är den första fasen i projektet. I denna fas så startas projektet och kontakten med kunden inleds. Kunden och projektgruppen kommer överens om projektets omfattning och en projektplan för projektets genomförande skrivs. Iden om den tekniska lösningen skrivs också ner i en vision. Utöver dessa två dokument så skall en work item list och en Risklista skrivas. I work item list bryter man ner och listar det som skall göras i projektet och i Risklistan listar man de största riskerna i projektet så man kan hantera dessa i ett tidigt stadie.
\subsection*{Elaboration}
\subsection*{Construction}
\subsection*{Transition}
[Describe or reference which management and technical practices will be used in the project, such as iterative development, continuous integration, independent testing and list any changes or particular configuration to the project. Specify how you will track progress in each practice. As an example, for iterative development the team may decide to use iteration assessments and iteration burndown reports and collect metrics such as velocity (completed work item points/ iteration).]
\section{Milstolpar}
[Define and describe the high-level objectives for the iterations and define milestones. For example, use the following table to lay out the schedule. If needed you may group the iterations into phases and use a separate table for each phase]
\begin{center}
    \begin{tabular}{ | c | m{6cm}  | c | c |}
    \hline
    Iteration & Primary objectives (risks and use case scenarios) & Scheduled start or milestone & Target velocity \\
    \hline
    I1 & Objectives \begin{enumerate} \item Mitigate Risk \item Develop Use Case 3, Scenario 2 \end{enumerate} & Date from/Date to & 15 \\
    \hline
	I2 & Objectives \begin{enumerate} \item Mitigate Risk 2 \item Develop Use Case 1, Scenario 2 \end{enumerate} & Date from/Date to & 16 \\
\hline
    \end{tabular}
\end{center}

\section{Deployment}
[Outline the strategy for deploying the software (and its updates) into the production environment.]
\section{Lessons learned}
[List lessons learned from the retrospective, with special emphasis on actions to be taken to improve, for example: the development environment, the process, or team collaboration.]
\end{document}
