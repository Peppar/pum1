% Standardinkluderingsfil
%
%  untitled
%
%  Created by David Granqvist on 2008-09-08.
%  Modified by Martin Erola
%

% Set document format/class
\documentclass[a4paper,twoside]{article}

%%%%%%%%%%%%%%%%%%%
% Include packages
%
\usepackage[utf8]{inputenc}   % Use utf-8 encoding for foreign characters
\usepackage[swedish]{babel}   % Support for swedish letters
\usepackage{fullpage}         % Setup for fullpage use
\usepackage{fancyhdr}         % Running Headers and footers
\usepackage{boxedminipage}    % Surround parts of graphics with box
\usepackage{listings}         % Package for including code in the document
\usepackage{ifpdf}            % Recommended way for checking for PDFLaTeX:
\usepackage{tabularx}         % Tabeller med automatisk stretch
% \usepackage[nofancy]{svninfo} % Extract Subversion info about the file
% \usepackage{color}          % Color
% \usepackage{lastpage}       % Total page count

% Graphics
\ifpdf
\usepackage[pdftex]{graphicx}
\else
\usepackage{graphicx}
\fi

%%%%%%%%%%%%%%%%%%%%%%%%%%%%%%%%%%%%%%%%%%%%%%%%%%%%%%%%%%
% Uncomment some of the following if you use the features
%

% Multipart figures
%\usepackage{subfigure}

% More symbols
%\usepackage{amsmath}
%\usepackage{amssymb}
%\usepackage{latexsym}

% If you want to generate a toc for each chapter (use with book)
% \usepackage{minitoc}

%%%%%%%%%%%%%%%%%%%%
% Document settings
%

% Header
\pagestyle{fancy}
% Sätter en marginal mellan header och (ovanstående?) text %
\setlength\headsep{10pt}
% Sätter höjden på headern
\setlength{\headheight}{32pt}

% Sätter styckesinställningar
\setlength\parindent{0pt}
\setlength\parskip{10pt}



\ifpdf
  \DeclareGraphicsExtensions{.pdf, .jpg, .tif, .png}
  \pdfinfo{
    /Title  (Iterationsplan för inception)
    /Author (PUM-grupp 1)
  }
\else
  \DeclareGraphicsExtensions{.eps, .jpg}
\fi

\title{Iterationsplan för inception}
\author{PUM-grupp 1}
\date{\today}

\begin{document}

\maketitle\thispagestyle{empty}
\newpage

{\centering \Large{Dokumenthistorik\\}}

\vspace{10pt}
\begin{tabularx}{\textwidth}{ |l|l|X|l|l| }
  \hline
    \textbf{version} & \textbf{datum} & \textbf{utförda ändringar} & \textbf{utförda av} & \textbf{granskad} \\
	\hline 
  0.1 & 2009-02-12 &  Bedömning av iteration (kapitel 7) ej gjord  & Alla & Alla   \\
  \hline
\end{tabularx}

\newpage

\setcounter{tocdepth}{2}
\tableofcontents
\newpage

\section{Inledning}
Det här dokumentet beskriver vad som ska göras i projektet under fasen inception. Inledningsvis beskrivs fasens milstolpar och övergripande mål. Därefter beskrivs arbetsfördelningen under fasen och de viktigaste ärendena som måste behandlas. Dokumentet avslutas med underlag för utvärdering av fasen samt själva utvärderingen, som fylls i precis innan fasen är över.

\section{Milstolpar}
I tabellen nedan beskrivs de milstolpar som har satts upp för denna fas, inklusive datum för varje milstolpe.

\begin{center}
	\begin{tabular}{| l | c |}
	\hline \textbf{Milstolpe} & \textbf{Datum} \\
	\hline Start & 2009-01-19 \\
	\hline Upprätta verktyg & 2009-01-30 \\
	\hline Utbildning inom OpenUP och verktyg & 2009-02-06 \\
	\hline Seminarium för inception & 2009-02-10 \\
	\hline Kund och projektgrupp överens om vision & 2009-02-12 \\
	\hline Dokumentinlämning & 2009-02-12 \\
	\hline
	\end{tabular}
\end{center}

\section{Övergripande mål}
De övergripande målen med iterationen är följande:

\begin{itemize}
	\item Anordna ett möte med kunden. Informationen från mötet ligger till grund för den tekniska visionen.
	\item Upprätta verktyg och anordna utbildning kring dessa verktyg för projektmedlemmarna.
	\item Skriva en teknisk vision som kunden sedan godkänner.
	\item Skriva en projektplan som hela projektgruppen är överens om.
	\item Presentera och sälja in projektet på det seminarium som hålls i slutet av fasen.
\end{itemize}

\section{Arbetsfördelning}
Uppgifterna fördelas jämnt mellan gruppmedlemmarna och nedlagd tid kan enkelt kontrolleras via tidrapport i Redmine.

\section{Ärenden som måste lösas}
Den här delen av dokumentet listar de ärenden som fortfarande är oklara och som måste lösas snarast.

\begin{center}
	\begin{tabular}{| l | l | l |}
		\hline Ärende & Status & Anteckningar \\
		\hline \#40 - Linux på lånedator & Öppet & Vänta med att installera Linux till Elaboration. \\
		\hline
	\end{tabular}
\end{center}

\section{Kriterier för utvärdering}
Här listas de kriterier som ska användas för att utvärdera om de övergripande målen med fasen har uppnåtts.

\begin{itemize}
	\item Kunden har godkänt den tekniska visionen
	\item Alla större problem med verktygen som används har utretts
	\item Projektmedlemmarna anser att de har tillräcklig kunskap för att hantera de vertkyg som upprättats
	\item Projektmedlemmarna anser att de har tillräcklig kunskap om OpenUP för att använda processen
	\item Projektplanen har godkänts av hela projektgruppen
	\item Projektgruppen anser att det gick att sälja in produkten under seminariet
	\item Alla dokument som ska lämnas in har granskats av projektgruppen och godkänts av alla projektmedlemmar
	\item Projektet har kommit igång och projektgruppen är redo att gå vidare till elaboration
\end{itemize}

\section{Bedömning}
Den här delen av dokumentet kommer att skrivas precis innan det att fasen är slut.

\begin{center}
	\begin{tabular}{| l | l |}
		\hline Föremål för bedömning & Inception-fasen \\
		\hline Datum för bedömning & 2009-02-13 \\
		\hline Deltagare & Alla gruppmedlemmar \\
		\hline Projektstatus & Grön \\
		\hline
	\end{tabular}
\end{center}

\begin{itemize}
	\item \textbf{Bedömning gentemot mål}
	\\ Vi har uppnått alla mål i god tid!
	\item \textbf{Work items: planerade jämfört med avklarade}
	\\ Alla planerade ärenden avklarades under denna fas.
	\item \textbf{Bedömning gentemot \textit{kriterier för utvärdering} i iterationsplanen}
	
	Vid utvärderingen av inception-fasen användes delen \textit{kriterier för utvärdering} av iterationsplanen som underlag för diskussion. En summering av vad vi kom fram till för respektive kriterium listas nedan:
	
	\begin{itemize}
		\item Kunden har godkänt den tekniska visionen
		\\ Ja, den blev godkänd den 13/2-09, vissa ändringar har skett så det är viktigt att alla har läst visionen.

		\item Alla större problem med verktygen som används har utretts
		\\ Ja, utan märkbar fördröjning av arbetet i övrigt.

		\item Projektmedlemmarna anser att de har tillräcklig kunskap för att hantera de verktyg som upprättats
		\\ Ja, men vi kan behöva öva lite mer på git merge och branch vilket blir lättare när det börjar komma in kod.

		\item Projektmedlemmarna anser att de har tillräcklig kunskap om OpenUP för att använda processen
		\\ Ja, även om det är svårt att få grepp om helheten från hemsidan.
		Handledaren tipsade om att använda Practices. Alla ska läsa på om OpenUP, om artifacts, tasks och om sin egen roll och dess funktion i de kommande faserna.

		\item Projektplanen har godkänts av hela projektgruppen
		\\ Ja, de som inte läst projektplanen skall göra det tills nästa möte.

		\item Projektgruppen anser att det gick att sälja in produkten under seminariet
		\\ Så där, seminariet var inte riktigt anpassat för ett projekt med öppen källkod. Handledaren som höll i seminariet gjorde några misstag vilket förargade vissa. Informationen inför seminariet var mycket bristfälligt jämfört med hur det genomfördes.

		\item Alla dokument som ska lämnas in har granskats av projektgruppen och godkänts av alla projektmedlemmar
		\\ Nej. Vissa dokument granskades inte av samtliga gruppmedlemmar.
		Handledaren påpekade att detta inte heller är ett nåbart mål när textmängden ökar i kommande faser.

		\item Projektet har kommit igång och projektgruppen är redo att gå vidare till elaboration
		\\ Ja, med många nyttiga lärdomar från den första etappen.
	
	\item \textbf{Övriga synpunkter och avvikelser}
		\begin{itemize}
		\item Vi konstaterade att vi rapporterat in all tid sanningsenligt och inte struntat i någon tid för att det ska stämma bättre med RUP/OpenUP-modellen. Vi har därför lagt ner ungefär 20 procent av den totala tidbudgeten på Inception istället för 10 procent vilket rekommenderas enligt RUP/OpenUP. Under de kommande faserna måste vi därför tänka mer på hur mycket tid vi lägger på möten och försöka effektivisera arbetet där det är möjligt.
		\item Sen var det en diskussion hur vi löser Iteration- och Project Burndown Report. Vi beslutade att vi skulle lösa detta med hjälp av fältet estimerade timmar i ärendesystemet i Redmine. De estimerade timmarna kommer då ungefär motsvara ett points-system där en point är en estimerad timme. X-axeln på burndown-grafen kommer representera kalenderdagag. Y-axeln kommer visa det totala antalet återstående estimerade timmar för iterationen respektive projektet.
		\end{itemize}
	\end{itemize}
	
\end{itemize}

\end{document}

