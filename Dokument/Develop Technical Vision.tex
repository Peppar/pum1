% Standardinkluderingsfil
%
%  untitled
%
%  Created by David Granqvist on 2008-09-08.
%  Modified by Martin Erola
%

% Set document format/class
\documentclass[a4paper,twoside]{article}

%%%%%%%%%%%%%%%%%%%
% Include packages
%
\usepackage[utf8]{inputenc}   % Use utf-8 encoding for foreign characters
\usepackage[swedish]{babel}   % Support for swedish letters
\usepackage{fullpage}         % Setup for fullpage use
\usepackage{fancyhdr}         % Running Headers and footers
\usepackage{boxedminipage}    % Surround parts of graphics with box
\usepackage{listings}         % Package for including code in the document
\usepackage{ifpdf}            % Recommended way for checking for PDFLaTeX:
\usepackage{tabularx}         % Tabeller med automatisk stretch
% \usepackage[nofancy]{svninfo} % Extract Subversion info about the file
% \usepackage{color}          % Color
% \usepackage{lastpage}       % Total page count

% Graphics
\ifpdf
\usepackage[pdftex]{graphicx}
\else
\usepackage{graphicx}
\fi

%%%%%%%%%%%%%%%%%%%%%%%%%%%%%%%%%%%%%%%%%%%%%%%%%%%%%%%%%%
% Uncomment some of the following if you use the features
%

% Multipart figures
%\usepackage{subfigure}

% More symbols
%\usepackage{amsmath}
%\usepackage{amssymb}
%\usepackage{latexsym}

% If you want to generate a toc for each chapter (use with book)
% \usepackage{minitoc}

%%%%%%%%%%%%%%%%%%%%
% Document settings
%

% Header
\pagestyle{fancy}
% Sätter en marginal mellan header och (ovanstående?) text %
\setlength\headsep{10pt}
% Sätter höjden på headern
\setlength{\headheight}{32pt}

% Sätter styckesinställningar
\setlength\parindent{0pt}
\setlength\parskip{10pt}



\ifpdf
  \DeclareGraphicsExtensions{.pdf, .jpg, .tif, .png}
  \pdfinfo{            
    /Title  (Develop Technical Vision)
    /Author (PUM-grupp 1)
  }
\else
  \DeclareGraphicsExtensions{.eps, .jpg}
\fi

\title{Develop Technical Vision}
\author{PUM-grupp 1}
\date{\today}

\begin{document}

\maketitle\thispagestyle{empty}
\newpage
\section{Intressenter}
Dom olika intressenterna vi har identiferat.
\begin{itemize}
\item PUM-gruppen
\item Kunden, VISIARC AB
\item Användare
\begin{itemize}
\item Användare som ej har tillgång till en traditionell klient-server lösning
\item Användare som väljer en distribuerad lösning
\end{itemize}
\item Andra utvecklare
\end{itemize}
En beskrivning av varje part för att klargöra ansvar  
\begin{itemize}
\item PUM-gruppen: Dom som kommer utveckla systemet, gör projektnära beslut och implementera funktionerna kunden formulerar.  PUM-gruppen bör ha nära kontakt med kunden.
\item Kunden, VISIARC AB är den som formulerar krav och funktioner till systemet. 
\item Användare som ej har tillgång till en traditonell klient-server lösning. Vi ser att det finns två sorters användare i den här gruppen. Dom som av ekonomiska skäl väljer bort en serverlösning. Serverar är konstsamma i el, underhåll och lokal. Sen dom användarna utan en större datorvana. För en oerfaren datoranvändare kan det vara svårt att få tillgänglighet till servermöjligheter. 
\item Användare som väljer en distribuerad lösning på grund av strukuren och rubustheten på systemet. Om användarna sitter mycket i en offlinemiljöer eller i otillförlitliga nätverk. Robustheten blir intressant då användaren ska publicera material som är kontroversiellt eller populärt så informationen blir redudant och har hög tillgänglighet.
\item Utvecklare som kommer in i projeket i framtiden. Hur tar man tillvara deras engagemang?
\end{itemize}
\section{Intresseters önskan}
PUM-gruppen och Kunden har träffats och diskuterat. Den absolut enskillt viktigaste egenskapen för projektet är att programmet ska vara lätt att använda och lätt att installera men fortfarande vara användbart.  Vi riktar oss mot användare utan större datorvana som sammarbetar med runt 5-10 personer på samma wiki. 
\end{document}