% Standardinkluderingsfil
%
%  untitled
%
%  Created by David Granqvist on 2008-09-08.
%  Modified by Martin Erola
%

% Set document format/class
\documentclass[a4paper,twoside]{article}

%%%%%%%%%%%%%%%%%%%
% Include packages
%
\usepackage[utf8]{inputenc}   % Use utf-8 encoding for foreign characters
\usepackage[swedish]{babel}   % Support for swedish letters
\usepackage{fullpage}         % Setup for fullpage use
\usepackage{fancyhdr}         % Running Headers and footers
\usepackage{boxedminipage}    % Surround parts of graphics with box
\usepackage{listings}         % Package for including code in the document
\usepackage{ifpdf}            % Recommended way for checking for PDFLaTeX:
\usepackage{tabularx}         % Tabeller med automatisk stretch
% \usepackage[nofancy]{svninfo} % Extract Subversion info about the file
% \usepackage{color}          % Color
% \usepackage{lastpage}       % Total page count

% Graphics
\ifpdf
\usepackage[pdftex]{graphicx}
\else
\usepackage{graphicx}
\fi

%%%%%%%%%%%%%%%%%%%%%%%%%%%%%%%%%%%%%%%%%%%%%%%%%%%%%%%%%%
% Uncomment some of the following if you use the features
%

% Multipart figures
%\usepackage{subfigure}

% More symbols
%\usepackage{amsmath}
%\usepackage{amssymb}
%\usepackage{latexsym}

% If you want to generate a toc for each chapter (use with book)
% \usepackage{minitoc}

%%%%%%%%%%%%%%%%%%%%
% Document settings
%

% Header
\pagestyle{fancy}
% Sätter en marginal mellan header och (ovanstående?) text %
\setlength\headsep{10pt}
% Sätter höjden på headern
\setlength{\headheight}{32pt}

% Sätter styckesinställningar
\setlength\parindent{0pt}
\setlength\parskip{10pt}



\ifpdf
  \DeclareGraphicsExtensions{.pdf, .jpg, .tif, .png}
  \pdfinfo{            
    /Title  (Projektplan)
    /Author (PUM-grupp 1)
  }
\else
  \DeclareGraphicsExtensions{.eps, .jpg}
\fi

\title{Projektplan}
\author{PUM-grupp 1}
\date{\today}

\begin{document}

\maketitle\thispagestyle{empty}

\newpage

\section{Introduktion}
Det här dokumentet beskriver hur projektet ska genomföras. Inledningsvis beskrivs projektets bakgrund, dess syfte och mål samt vilka personer som är delaktiga i projektet. Därefter ges en översiktlig beskrivning av projektprocessen OpenUp som används i projektet samt vilka roller som är aktuella för detta projekt. Slutligen beskrivs hur projektet ska levereras och de lärdomar dragits från projektet.

\subsection{Bakgrund}
Detta projekt genomförs som huvudmoment i projektkursen \textit{Programutvecklingsprojekt i ett helhetsperspektiv} (kurskod TDDD09) vid Linköpings universitet. Projektet genomförs under vårterminen 2009.

\subsection{Syfte och mål}
Syftet med projektet är att visa hur mjukvaruframställning på projektfrom går till samt att ge en inblick i hur mjukvara framställs inom industrin.

Målet med projektet är att genomföra projektet på ett lyckat sätt och samtidigt ge kunden en mjukvara som denne är nöjd med och kan dra nytta av.

\section{Projektöversikt}
[kort översiktlig beskrivning av vad vårt projekt går ut på]

\section{Projektorganisation}
[Projektorganisation]

\subsection{Projektmedlemmar}
[Introduce the project team, team members, and roles that they play during this project. If applicable, introduce work areas, domains, or technical work packages that are assigned to team members.]
Nedan följer en beskrivning utav de roller som innehas av medlemmar i projektgruppen.

\subsubsection*{Projektledare - Mikael Waernér}
Projektledaren är den person som har huvudansvaret för planeringen i projektet. Han ser till så att projektgruppen är fokuserad på målen i projektet. Projektledaren ser till så att projektgruppen genomför ett projekt med ett resultat som kunden är nöjd med.

\subsubsection*{Analytiker - Erik Thorselius}
Analytikerns jobb är att fånga in kunden och målgruppens behov och förstå de problem som måste lösas. Analytiker bör också se de möjligheter som eventuella problem kan medföra.

\subsubsection*{Arkitekt - Oskar Holstensson}
Arkitekten ansvarar för mjukvarans arkitektur, i det ingår att ta tekniska beslut om den överläggande designen och implementationen av systemet. Arkitekten måste balansera kundens behov med tekniska risker för att skapa en design som är genomförbar och effektiv att följa och validera.

\subsubsection*{Utvecklare - Victor Ortman}
Utvecklaren ansvarar för att systemet blir utvecklat enligt arkitekturen. I det ingår bland annat att implementera enhetstestning och integrera olika komponenter.

\subsubsection*{Testare - Martin Pettersson}
Testaren ansvarar för att idientifiera de tester som är nödvändiga för produkten. Han ansvarar också för implementationen och ledandet av dessa tester. Testaren skall också logga testresultaten och förstå innebörden av dessa för systemet.

\subsubsection*{Kvalitetssamordnare - Linus Dunkers}
[Kvalitetssamordnaren]

\subsection{Kunden}
[Kunden]

\section{Projektgenomförande}
[Describe or reference which management and technical practices will be used in the project, such as iterative development, continuous integration, independent testing and list any changes or particular configuration to the project.

\subsection{Fasplan}
Nedan beskrivs de faser som ingår i projektet.

\subsubsection*{Inception}
Inception är den första fasen i projektet. I denna fas så startas projektet och kontakten med kunden inleds. Kunden och projektgruppen kommer överens om projektets omfattning och en projektplan för projektets genomförande skrivs. Idén om den tekniska lösningen skrivs också ner i en vision. Utöver dessa två dokument så skall en work items list och en risklista skrivas. I work items list bryts det som ska göras i projektet ned och i risklistan listas de största riskerna i projektet så dessa kan hanteras i ett tidigt stadie.

\subsubsection*{Elaboration}
I elaboration identifieras och utvecklas de krav som kunden har på produkten. 

\subsubsection*{Construction}
Construction är den tredje fasen i projektet. Här påbörjas konstruktionen av programvaran.

\subsubsection*{Transition}
[Transition]

\subsection{Status- och tidrapportering}
[Specify how you will track progress in each practice. As an example, for iterative development the team may decide to use iteration assessments and iteration burndown reports and collect metrics such as velocity (completed work item points/ iteration).]

\section{Milstolpar och mål}
[Define and describe the high-level objectives for the iterations and define milestones. For example, use the following table to lay out the schedule. If needed you may group the iterations into phases and use a separate table for each phase]

\begin{center}
	\begin{tabular}{ | c | m{1.3cm} | m{5cm} | m{2.5cm} | m{2cm} |}
		\hline
		\textbf{Fas} & \textbf{Iteration} & \textbf{Primära mål (risker och use case-scenarier)} & \textbf{Start och slut (milstolpe)} & \textbf{Uppskattad tid (arbetsdagar)} \\
		\hline
		Inception & M0 & \begin{enumerate} \item Mitigate Risk \item Develop Use Case 3, Scenario 2 \end{enumerate} & 2009-01-26 2009-02-12 & X \\
		\hline
		Elaboration & M1 & \begin{enumerate} \item Mitigate Risk 2 \item Develop Use Case 1, Scenario 2 \end{enumerate} & 2009-02-13 2009-03-06 & X \\
		\hline
		Construction & M2 & \begin{enumerate} \item Mitigate Risk \item Develop Use Case 3, Scenario 2 \end{enumerate} & 2009-03-07 2009-04-10 & X \\
		\hline
		Construction & M3 & \begin{enumerate} \item Mitigate Risk \item Develop Use Case 3, Scenario 2 \end{enumerate} & 2009-04-11 2008-05-12 & X \\
		\hline
		Transition & M4 & \begin{enumerate} \item Mitigate Risk \item Develop Use Case 3, Scenario 2 \end{enumerate} & 2008-05-13 2008-05-21 & X \\
		\hline
	\end{tabular}
\end{center}

\section{Leverans}
Mjukvaran kommer paketeras för enkel installation hos klientdatorerna. Om möjlighet ges för slutproduken att läggas på en central produktserver kommer denna att utnyttjas för enklast möjliga distribution och uppdatering av mjukvaran.

\section{Lärdomar}
Den här delen av dokumentet kommer att fyllas på senare med de lärdomar som dragits från projektet.

\end{document}

