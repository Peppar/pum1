% Standardinkluderingsfil
%
%  untitled
%
%  Created by David Granqvist on 2008-09-08.
%  Modified by Martin Erola
%

% Set document format/class
\documentclass[a4paper,twoside]{article}

%%%%%%%%%%%%%%%%%%%
% Include packages
%
\usepackage[utf8]{inputenc}   % Use utf-8 encoding for foreign characters
\usepackage[swedish]{babel}   % Support for swedish letters
\usepackage{fullpage}         % Setup for fullpage use
\usepackage{fancyhdr}         % Running Headers and footers
\usepackage{boxedminipage}    % Surround parts of graphics with box
\usepackage{listings}         % Package for including code in the document
\usepackage{ifpdf}            % Recommended way for checking for PDFLaTeX:
\usepackage{tabularx}         % Tabeller med automatisk stretch
% \usepackage[nofancy]{svninfo} % Extract Subversion info about the file
% \usepackage{color}          % Color
% \usepackage{lastpage}       % Total page count

% Graphics
\ifpdf
\usepackage[pdftex]{graphicx}
\else
\usepackage{graphicx}
\fi

%%%%%%%%%%%%%%%%%%%%%%%%%%%%%%%%%%%%%%%%%%%%%%%%%%%%%%%%%%
% Uncomment some of the following if you use the features
%

% Multipart figures
%\usepackage{subfigure}

% More symbols
%\usepackage{amsmath}
%\usepackage{amssymb}
%\usepackage{latexsym}

% If you want to generate a toc for each chapter (use with book)
% \usepackage{minitoc}

%%%%%%%%%%%%%%%%%%%%
% Document settings
%

% Header
\pagestyle{fancy}
% Sätter en marginal mellan header och (ovanstående?) text %
\setlength\headsep{10pt}
% Sätter höjden på headern
\setlength{\headheight}{32pt}

% Sätter styckesinställningar
\setlength\parindent{0pt}
\setlength\parskip{10pt}



\ifpdf
  \DeclareGraphicsExtensions{.pdf, .jpg, .tif, .png}
  \pdfinfo{
    /Title  (Iterationsplan för elaboration)
    /Author (PUM-grupp 1)
  }
\else
  \DeclareGraphicsExtensions{.eps, .jpg}
\fi

\title{Iterationsplan för elaboration}
\author{PUM-grupp 1}
\date{\today}

\begin{document}

\maketitle\thispagestyle{empty}
\newpage

{\centering \Large{Dokumenthistorik\\}}

\vspace{10pt}
\begin{tabularx}{\textwidth}{ |l|l|X|l|l| }
  \hline
    \textbf{version} & \textbf{datum} & \textbf{utförda ändringar} & \textbf{utförda av} & \textbf{granskad} \\
	\hline 0.1 & 2009-02-17 & Ett första utkast & Alla & Alla \\
  \hline
\end{tabularx}

\newpage

\setcounter{tocdepth}{2}
\tableofcontents
\newpage

\section{Inledning}
Det här dokumentet beskriver vad som ska göras i projektet under fasen elaboration. Inledningsvis beskrivs fasens milstolpar och övergripande mål. Därefter beskrivs arbetsfördelningen under fasen och de viktigaste ärendena som måste behandlas. Dokumentet avslutas med underlag för utvärdering av fasen samt själva utvärderingen, som fylls i precis innan det att fasen är slut.

\section{Milstolpar}
I tabellen nedan beskrivs de milstolpar som har satts upp för denna fas, inklusive datum för varje milstolpe.

\begin{center}
	\begin{tabular}{| l | c |}
	\hline \textbf{Milstolpe} & \textbf{Datum} \\
	\hline Start & 2009-02-13 \\
	\hline Inlämning av dokument för opposition & 2009-02-27 \\
	\hline Opposition & 2009-03-05 \\
	\hline Seminarium för elaboration & 2009-03-05 \\
	\hline Dokumentinlämning & 2009-03-05 \\
	\hline
	\end{tabular}
\end{center}

\section{Övergripande mål}
De övergripande målen med iterationen är följande:

\begin{itemize}
	\item Samla in och detaljera alla krav som kunden har på mjukvaran
	\item Skapa en hållbar arkitektur för programmet
	\item Skapa en enkel prototyp som visar att programmet är möjligt att bygga
	\item Samla potentiella risker inom projektet i en risklista
	\item Skriva en mer detaljerad plan för hur fasen construction ska utföras
	\item Skapa en detaljerad design för programmet
	\item Presentera processen och arbetssättet inom projektet på det seminarium som hålls i slutet av fasen
\end{itemize}

\section{Arbetsfördelning}
Uppgifterna fördelas jämnt mellan gruppmedlemmarna och nedlagd tid kan enkelt kontrolleras via tidrapport i Redmine.

\section{Ärenden som måste lösas}
Den här delen av dokumentet listar de ärenden som fortfarande är oklara och som måste lösas snarast.

\begin{center}
	\begin{tabular}{| l | l | l |}
		\hline Ärende & Status & Anteckningar \\
		\hline \#40 - Linux på lånedator & Öppet & Ej påbörjat, måste göras innan construction. \\
		\hline
	\end{tabular}
\end{center}

\section{Kriterier för utvärdering}
Här listas de kriterier som ska användas för att utvärdera om de övergripande målen med fasen har uppnåtts.

\begin{itemize}
	\item Kunden har skriftligen godkänt att projektgruppen uppfattat alla krav som kunden har på mjukvaran och vilka krav som kommer implementeras 
	\item Det finns en arkitektur för programmet som projektgruppen är säkra på kommer att fungera som underlag för det slutgiltiga programmet
	\item Det finns en detaljerad design för hur programmet ska byggas
	\item En prototyp har skapats med tillräcklig funktionalitet för att visa att mjukvaran kommer att fungera
	\item En risklista med potentiella risker inom projektet har skrivits
	\item Det finns en detaljerad plan för hur fasen construction ska utföras
	\item De fiktiva åhörarna under seminariet blev nöjda med presentationen av vårt arbetssätt och processen
	\item Alla dokument som ska lämnas in har granskats av åtminstone två andra projektmedlemmar förutom den som ursprungligen skrev dokumentet
	\item Projektgruppen är redo att gå vidare till construction
\end{itemize}

\section{Bedömning}
Den här delen av dokumentet kommer att skrivas precis innan det att fasen är slut.

\begin{center}
	\begin{tabular}{| l | l |}
		\hline Föremål för bedömning & Elaboration-fasen \\
		\hline Datum för bedömning & 2009-03-06 \\
		\hline Deltagare & Alla gruppmedlemmar förutom Erik \\
		\hline Projektstatus & Grön \\
		\hline
	\end{tabular}
\end{center}

\begin{itemize}
	\item \textbf{Bedömning gentemot mål}
		\begin{itemize}
		\item Samla in och detaljera alla krav som kunden har på mjukvaran
		Kunden har endast ett fåtal främst icke-funktionella krav. Vi får specificera fler mätbara krav och få dessa godkända av kund.
		\item Skapa en hållbar arkitektur för programmet
		I och med bytet till Bazaar så känns upplägget med python och wxPython stabilt. Något oklart hur det går att implementera säker kommuniktation med t.ex. TLS och hur bra wxPythons HTML-stöd är.
		\item Skapa en enkel prototyp som visar att programmet är möjligt att bygga
		En väldigt simpel prototyp kommer byggas klart tills dokumentinlämningen samt dokumentet prototyp-design.
		\item Samla potentiella risker inom projektet i en risklista
		Risklistan skall utökas innan dokumentinlämningen.
		\item Skriva en mer detaljerad plan för hur fasen construction ska utföras
		Construction-planen ska påbörjas så att en preliminär version finns klar innan dokumentinlämningen.
		\item Skapa en detaljerad design för programmet
		Designen är ca 70\% klar men ska hinna färdigställas innan dokumentinlämningen.
		\item Presentera processen och arbetssättet inom projektet på det seminarium som hålls i slutet av fasen
		Presentationen gick bra.
		\end{itemize}
	
	\item \textbf{Work items: planerade jämfört med avklarade}
		Som det ser ut just nu är vi långt från klara men detta beror till stor del på att gruppmedlemmarna inte loggat tid och uppdaterat ärenden i den takt de jobbat. En hel del tid återstår 			också som är relaterat till prototypen som ännu inte är färdigställd.	
	
	\item \textbf{Bedömning gentemot \textit{kriterier för utvärdering} i iterationsplanen}
		\begin{itemize}
		\item Kunden har skriftligen godkänt att projektgruppen uppfattat alla krav som kunden har på mjukvaran och vilka krav som kommer implementeras
		Ska ske under måndagen eller tisdagen i samband med dokumentinlämningen
		\item Det finns en arkitektur för programmet som projektgruppen är säkra på kommer att fungera som underlag för det slutgiltiga programmet
		Projektgruppen är säker på att arkitekturen ska klara kraven då prototypen testar de allra flesta grundkraven.
		\item Det finns en detaljerad design för hur programmet ska byggas
		Det finns en design och de småsaker som är oklart är för närvarande oväsentligt och kan beslutas om vid ett senare skeende.
		\item En prototyp har skapats med tillräcklig funktionalitet för att visa att mjukvaran kommer att fungera
		Ja, prototypen färdigställs helt tills dokumentinlämningen.
		\item En risklista med potentiella risker inom projektet har skrivits
		Ja, kommer uppdateras innan dokumentinlämning.
		\item Det finns en detaljerad plan för hur fasen construction ska utföras
		Inte än, men förhoppningsvis helt klar vid dokumentinlämning.
		\item De fiktiva åhörarna under seminariet blev nöjda med presentationen av vårt arbetssätt och processen
		De var nöjda med presentationen. Det var tydligt och bra.
		\item Alla dokument som ska lämnas in har granskats av åtminstone två andra projektmedlemmar förutom den som ursprungligen skrev dokumentet
		Granskningsprocessen har inte kommit igång ordentligt än men måste göra det under helgen.
		\item Projektgruppen är redo att gå vidare till construction
		Ja, det är inget av stor betydelse som förhindrar att vi går vidare till construction.
		\end{itemize}
\end{itemize}

\end{document}

