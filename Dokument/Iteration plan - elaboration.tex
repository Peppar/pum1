% Standardinkluderingsfil
%
%  untitled
%
%  Created by David Granqvist on 2008-09-08.
%  Modified by Martin Erola
%

% Set document format/class
\documentclass[a4paper,twoside]{article}

%%%%%%%%%%%%%%%%%%%
% Include packages
%
\usepackage[utf8]{inputenc}   % Use utf-8 encoding for foreign characters
\usepackage[swedish]{babel}   % Support for swedish letters
\usepackage{fullpage}         % Setup for fullpage use
\usepackage{fancyhdr}         % Running Headers and footers
\usepackage{boxedminipage}    % Surround parts of graphics with box
\usepackage{listings}         % Package for including code in the document
\usepackage{ifpdf}            % Recommended way for checking for PDFLaTeX:
\usepackage{tabularx}         % Tabeller med automatisk stretch
% \usepackage[nofancy]{svninfo} % Extract Subversion info about the file
% \usepackage{color}          % Color
% \usepackage{lastpage}       % Total page count

% Graphics
\ifpdf
\usepackage[pdftex]{graphicx}
\else
\usepackage{graphicx}
\fi

%%%%%%%%%%%%%%%%%%%%%%%%%%%%%%%%%%%%%%%%%%%%%%%%%%%%%%%%%%
% Uncomment some of the following if you use the features
%

% Multipart figures
%\usepackage{subfigure}

% More symbols
%\usepackage{amsmath}
%\usepackage{amssymb}
%\usepackage{latexsym}

% If you want to generate a toc for each chapter (use with book)
% \usepackage{minitoc}

%%%%%%%%%%%%%%%%%%%%
% Document settings
%

% Header
\pagestyle{fancy}
% Sätter en marginal mellan header och (ovanstående?) text %
\setlength\headsep{10pt}
% Sätter höjden på headern
\setlength{\headheight}{32pt}

% Sätter styckesinställningar
\setlength\parindent{0pt}
\setlength\parskip{10pt}



\ifpdf
  \DeclareGraphicsExtensions{.pdf, .jpg, .tif, .png}
  \pdfinfo{
    /Title  (Iterationsplan för elaboration)
    /Author (PUM-grupp 1)
  }
\else
  \DeclareGraphicsExtensions{.eps, .jpg}
\fi

\title{Iterationsplan för elaboration}
\author{PUM-grupp 1}
\date{\today}

\begin{document}

\maketitle\thispagestyle{empty}
\newpage

{\centering \Large{Dokumenthistorik\\}}

\vspace{10pt}
\begin{tabularx}{\textwidth}{ |l|l|X|l|l| }
  \hline
    \textbf{version} & \textbf{datum} & \textbf{utförda ändringar} & \textbf{utförda av} & \textbf{granskad} \\
	\hline 0.1 & 2009-02-17 & Ett första utkast & Alla & Alla \\
  \hline
\end{tabularx}

\newpage

\setcounter{tocdepth}{2}
\tableofcontents
\newpage

\section{Inledning}
Det här dokumentet beskriver vad som ska göras i projektet under fasen elaboration. Inledningsvis beskrivs fasens milstolpar och övergripande mål. Därefter beskrivs arbetsfördelningen under fasen och de viktigaste ärendena som måste behandlas. Dokumentet avslutas med underlag för utvärdering av fasen samt själva utvärderingen, som fylls i precis innan det att fasen är slut.

\section{Milstolpar}
I tabellen nedan beskrivs de milstolpar som har satts upp för denna fas, inklusive datum för varje milstolpe.

\begin{center}
	\begin{tabular}{| l | c |}
	\hline \textbf{Milstolpe} & \textbf{Datum} \\
	\hline Start & 2009-02-13 \\
	\hline Inlämning av dokument för opposition & 2009-02-27 \\
	\hline Opposition & 2009-03-05 \\
	\hline Seminarium för elaboration & 2009-03-05 \\
	\hline Dokumentinlämning & 2009-03-05 \\
	\hline
	\end{tabular}
\end{center}

\section{Övergripande mål}
De övergripande målen med iterationen är följande:

\begin{itemize}
	\item Samla in och detaljera alla krav som kunden har på mjukvaran
	\item Skapa en hållbar arkitektur för programmet
	\item Skapa en enkel prototyp som visar att programmet är möjligt att bygga
	\item Samla potentiella risker inom projektet i en risklista
	\item Skriva en mer detaljerad plan för hur fasen construction ska utföras
	\item Skapa en detaljerad design för programmet
	\item Presentera processen och arbetssättet inom projektet på det seminarium som hålls i slutet av fasen
\end{itemize}

\section{Arbetsfördelning}
Uppgifterna fördelas jämnt mellan gruppmedlemmarna och nedlagd tid kan enkelt kontrolleras via tidrapport i Redmine.

\section{Ärenden som måste lösas}
Den här delen av dokumentet listar de ärenden som fortfarande är oklara och som måste lösas snarast.

\begin{center}
	\begin{tabular}{| l | l | l |}
		\hline Ärende & Status & Anteckningar \\
		\hline \#40 - Linux på lånedator & Öppet & Ej påbörjat, måste göras innan construction. \\
		\hline
	\end{tabular}
\end{center}

\section{Kriterier för utvärdering}
Här listas de kriterier som ska användas för att utvärdera om de övergripande målen med fasen har uppnåtts.

\begin{itemize}
	\item Kunden har skriftligen godkänt att projektgruppen uppfattat alla krav som kunden har på mjukvaran och vilka krav som kommer implementeras 
	\item Det finns en arkitektur för programmet som projektgruppen är säkra på kommer att fungera som underlag för det slutgiltiga programmet
	\item Det finns en detaljerad design för hur programmet ska byggas
	\item En prototyp har skapats med tillräcklig funktionalitet för att visa att mjukvaran kommer att fungera
	\item En risklista med potentiella risker inom projektet har skrivits
	\item Det finns en detaljerad plan för hur fasen construction ska utföras
	\item De fiktiva åhörarna under seminariet blev nöjda med presentationen av vårt arbetssätt och processen
	\item Alla dokument som ska lämnas in har granskats av åtminstone två andra projektmedlemmar förutom den som ursprungligen skrev dokumentet
	\item Projektgruppen är redo att gå vidare till construction
\end{itemize}

\section{Bedömning}
Den här delen av dokumentet kommer att skrivas precis innan det att fasen är slut.

\begin{center}
	\begin{tabular}{| l | l |}
		\hline Föremål för bedömning & Elaboration-fasen \\
		\hline Datum för bedömning & [datum] \\
		\hline Deltagare & Alla gruppmedlemmar \\
		\hline Projektstatus & [röd, gul, grön] \\
		\hline
	\end{tabular}
\end{center}

\begin{itemize}
	\item \textbf{Bedömning gentemot mål}
	\item \textbf{Work items: planerade jämfört med avklarade}
	\item \textbf{Bedömning gentemot \textit{kriterier för utvärdering} i iterationsplanen}	
	\item \textbf{Övriga synpunkter och avvikelser}
\end{itemize}

\end{document}

