% Standardinkluderingsfil
%
%  untitled
%
%  Created by David Granqvist on 2008-09-08.
%  Modified by Martin Erola
%

% Set document format/class
\documentclass[a4paper,twoside]{article}

%%%%%%%%%%%%%%%%%%%
% Include packages
%
\usepackage[utf8]{inputenc}   % Use utf-8 encoding for foreign characters
\usepackage[swedish]{babel}   % Support for swedish letters
\usepackage{fullpage}         % Setup for fullpage use
\usepackage{fancyhdr}         % Running Headers and footers
\usepackage{boxedminipage}    % Surround parts of graphics with box
\usepackage{listings}         % Package for including code in the document
\usepackage{ifpdf}            % Recommended way for checking for PDFLaTeX:
\usepackage{tabularx}         % Tabeller med automatisk stretch
% \usepackage[nofancy]{svninfo} % Extract Subversion info about the file
% \usepackage{color}          % Color
% \usepackage{lastpage}       % Total page count

% Graphics
\ifpdf
\usepackage[pdftex]{graphicx}
\else
\usepackage{graphicx}
\fi

%%%%%%%%%%%%%%%%%%%%%%%%%%%%%%%%%%%%%%%%%%%%%%%%%%%%%%%%%%
% Uncomment some of the following if you use the features
%

% Multipart figures
%\usepackage{subfigure}

% More symbols
%\usepackage{amsmath}
%\usepackage{amssymb}
%\usepackage{latexsym}

% If you want to generate a toc for each chapter (use with book)
% \usepackage{minitoc}

%%%%%%%%%%%%%%%%%%%%
% Document settings
%

% Header
\pagestyle{fancy}
% Sätter en marginal mellan header och (ovanstående?) text %
\setlength\headsep{10pt}
% Sätter höjden på headern
\setlength{\headheight}{32pt}

% Sätter styckesinställningar
\setlength\parindent{0pt}
\setlength\parskip{10pt}



\ifpdf
  \DeclareGraphicsExtensions{.pdf, .jpg, .tif, .png}
  \pdfinfo{            
    /Title  (Risklista)
    /Author (PUM-grupp 1)
  }
\else
  \DeclareGraphicsExtensions{.eps, .jpg}
\fi

\title{Risklista}
\author{PUM-grupp 1}
\date{\today}

\begin{document}

\maketitle\thispagestyle{empty}

\newpage

\setcounter{tocdepth}{2}
\newpage

\section{Inledning}
Det här dokumentet beskriver tänkbara risker som är aktuella i projektet. Det beskriver hur stor chans att något inträffar, vilken inverkan det har och hur risken ska hanteras.

\section{Risker}
Nedan följer de risker som är aktuella i projektet. Riskernas sannolikhet att inträffa samt inverkan har graderats på en skala 1-4, där ett lägre värde innebär mindre sannolikhet att risken slår in respektive mindre inverkan på projektet. Magnituden beräknas som produkten av sannolikheten och inverkan och får därför en skala på 1-16, även här är ett lägre värde bättre. Tänkbara metoder för att undvika en risk eller mildra dess effekter kan läsas under punkten lösning för varje risk.

\subsection{Requirements / Krav}
\begin{itemize}
\item \textbf{Risk}: Kraven på projektet är ouppnåeliga på grund av tidsbrist.

\begin{tabular}{ | l | l | l |}
	\hline
	Sannolikhet: 3 & Inverkan: 2 & Magnitud: 6 \\ \hline
\end{tabular}

\textbf{Lösning}: Förklara för kunden vad vi tror att vi kan och inte kan slutföra under detta projekt samt att vid behov skriva om kravlistan.
\end{itemize}

\subsection{Usability / Användbarhet}
\begin{itemize}
\item \textbf{Risk}: Produkten blir inte så användbar som det var tänkt från början.

\begin{tabular}{ | l | l | l |}
	\hline
	Sannolikhet: 3 & Inverkan: 3 & Magnitud: 9 \\ \hline
\end{tabular}

\textbf{Lösning}: Iterera fram bättre och bättre grafiska användargränssnitt och gör användartester mellan varje iteration. Arbeta med en erkänd metod för att förbättra det grafiska gränsnittet.
\end{itemize}

\subsection{Testability / Tester}
\begin{itemize}
\item \textbf{Risk}: Tester kan inte ske innan implementation av programvaran.

\begin{tabular}{ | l | l | l |}
	\hline
	Sannolikhet: 2 & Inverkan: 3 & Magnitud: 6 \\ \hline
\end{tabular}

\textbf{Lösning}: Läs på om gränssnitt, protokoll och dokumentation. Pröva att använda drivers eller att testa små delar åt gången.
\end{itemize}

\subsection{Generella risker}
\begin{itemize}
\item \textbf{Risk}: En nyckelperson i projektet blir utbytt eller tvingas sluta.

\begin{tabular}{ | l | l | l |}
	\hline
	Sannolikhet: 4 & Inverkan: 3 & Magnitud: 12 \\ \hline
\end{tabular}

\textbf{Lösning}: Dokumentera rikligt så någon annan snabbt och enkelt kan ta över personens arbete.
\item \textbf{Risk}: Kod blir rörig och svårläst.

\begin{tabular}{ | l | l | l |}
	\hline
	Sannolikhet: 4 & Inverkan: 2 & Magnitud: 8 \\ \hline
\end{tabular}

\textbf{Lösning}: Sätt upp en kod- och kommenteringsstandard som hela gruppen följer.
\item \textbf{Risk}: Vi lyckas inte få kontakt med kunden under en lång tidsperiod.

\begin{tabular}{ | l | l | l |}
	\hline
	Sannolikhet: 1 & Inverkan: 3 & Magnitud: 3 \\ \hline
\end{tabular}

\textbf{Lösning}: Arbeta vidare mot kraven så gott som det går och fortsätt söka kunden.
\end{itemize}

\end{document}

