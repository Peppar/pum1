% Standardinkluderingsfil
%
%  untitled
%
%  Created by David Granqvist on 2008-09-08.
%  Modified by Martin Erola
%

% Set document format/class
\documentclass[a4paper,twoside]{article}

%%%%%%%%%%%%%%%%%%%
% Include packages
%
\usepackage[utf8]{inputenc}   % Use utf-8 encoding for foreign characters
\usepackage[swedish]{babel}   % Support for swedish letters
\usepackage{fullpage}         % Setup for fullpage use
\usepackage{fancyhdr}         % Running Headers and footers
\usepackage{boxedminipage}    % Surround parts of graphics with box
\usepackage{listings}         % Package for including code in the document
\usepackage{ifpdf}            % Recommended way for checking for PDFLaTeX:
\usepackage{tabularx}         % Tabeller med automatisk stretch
% \usepackage[nofancy]{svninfo} % Extract Subversion info about the file
% \usepackage{color}          % Color
% \usepackage{lastpage}       % Total page count

% Graphics
\ifpdf
\usepackage[pdftex]{graphicx}
\else
\usepackage{graphicx}
\fi

%%%%%%%%%%%%%%%%%%%%%%%%%%%%%%%%%%%%%%%%%%%%%%%%%%%%%%%%%%
% Uncomment some of the following if you use the features
%

% Multipart figures
%\usepackage{subfigure}

% More symbols
%\usepackage{amsmath}
%\usepackage{amssymb}
%\usepackage{latexsym}

% If you want to generate a toc for each chapter (use with book)
% \usepackage{minitoc}

%%%%%%%%%%%%%%%%%%%%
% Document settings
%

% Header
\pagestyle{fancy}
% Sätter en marginal mellan header och (ovanstående?) text %
\setlength\headsep{10pt}
% Sätter höjden på headern
\setlength{\headheight}{32pt}

% Sätter styckesinställningar
\setlength\parindent{0pt}
\setlength\parskip{10pt}



\ifpdf
  \DeclareGraphicsExtensions{.pdf, .jpg, .tif, .png}
  \pdfinfo{            
    /Title  (Use-case)
    /Author (PUM-grupp 1)
  }
\else
  \DeclareGraphicsExtensions{.eps, .jpg}
\fi

\title{Use-case}
\author{PUM-grupp 1}
\date{\today}

\begin{document}

\maketitle\thispagestyle{empty}

\newpage

\section{Inledning}

%\textsc{\LARGE Use-case}

\subsection{Exempel}
\begin{itemize}
	\item Kortfattad beskrivning - Kort beskrivning av use-case.
	\item Aktörer - Beskrivning av de aktörer som medverkar.
	\item Förhandsvillkor - Antaganden för att fallet ska kunna genomföras.
	\item Händelseförlopp - Stegvis genomgång av händelseförloppet.
	\item Alternativt händelseförlopp - Andra sätt fallet kan utspela sig på, till exempel att något går fel.
	\item Huvudscenario - Det viktigaste scenariot i fallet.
	\item Efterhandsvillkor - Möjliga utgångar av fallet.
	\item Speciella krav - Speciella krav för fallet.
\end{itemize}

\section{Use-cases} 

För att lättare förstå vad användarna ska kunna åstadkomma med vårt system har gruppen satt upp ett antal use-cases för att försöka täcka de olika funktioner som kommer att finnas.

\subsection{Bläddra}

\begin{itemize}
	\item Kortfattad beskrivning
	\\En person vill finna en speciell artikel.
	\item Aktörer
	\\En användare, person A, utan större datorvana.
	\item Förhandsvillkor
	\\Person A har programmet installerat på sin dator.
	\item Händelseförlopp
	\begin{enumerate}
		\item Person A startar programmet.
		\item Person A skriver in sökord som matchar artikeln han vill söka efter i en sökruta.
		\item En lista med möjliga artiklar dyker upp och person A väljer en av dem.
		\item Artikeln visas i ett fönster och person A kan läsa den.
	\end{enumerate}
	\item Alternativt händelseförlopp
	\begin{enumerate}
		\item Person A startar programmet.
		\item Person A navigerar till sin önskade artikel genom länkar från andra artiklar.	
		\item Artikeln visas i ett fönster och person A kan läsa den.
	\end{enumerate}	
	\item Huvudscenario
	\\Den viktigaste delen i detta use-case är att person A kan välja mellan två sätt att finna sin artikel. Antingen med hjälp av länkar från andra artiklar eller genom en sökfunktion.
	\item Efterhandsvillkor
	\begin{itemize}
	\item Fann önskad artikel
	\\Användaren fick tag på den artikel som han/hon önskade
	\item Fann inte önskad artikel
	\\Användaren fick inte tag på den artikel han/hon önskade, antingen på grund av bristande sökord, felaktiga länkningar eller så kanske inte användarens wiki fått hem alla artiklar från övriga användare i gruppen.
	\end{itemize}
	\item Speciella krav
	\\En startsida på användarens wiki med länkar till andra wiki-sidor krävs för att det alternativa händelseförloppet ska kunna genomföras.
\end{itemize}

\subsection{Läsa en artikel}
\begin{itemize}
	\item Kortfattad beskrivning
	\\En person skall läsa en artikel i sin wiki online.
	\item Aktörer
	\\En användare, person A, utan större datorvana.
	\item Förhandsvillkor
	\\Person A är online och har programmet installerat på sin dator.
	\item Händelseförlopp
	\begin{enumerate}
		\item Person A startar programmet.
		\item Programmet synkroniserar ändringar i artiklar med andra medlemmar som är online i wikigruppen.
		\item Person A väljer en uppdaterad artikel han/hon vill läsa.
		\item Artikeln visas och person A läser den.
	\end{enumerate}
	\item Alternativt händelseförlopp
	\\Efter steg 1:
	\begin{enumerate}	
		\item Programmet söker efter uppdateringar från andra medlemmar i wikigruppen men finner inga.
		\item Person A låter programmet stå på i bakgrunden i väntan på nya uppdateringar.	
	\end{enumerate}	
	\item Huvudscenario
	\\Den viktigaste delen i use-caset är när programmet synkroniserar artiklarna med de andra medlemmarna
	\item Efterhandsvillkor
	\begin{itemize}
		\item Avändaren lyckades hitta uppateringar.
		\\Person A hitta nya uppdateringar från andra användare och synkroniserade automatiskt sina dokument med denna användares.
		\item Användaren lyckades inte hitta nya uppdateringar.
		\\Person A lyckades inte hitta nya uppdateringar från några andra användare.
	\end{itemize}	
	\item Speciella krav
	\\Användaren skall kunna vara medlem i en speciell wikigrupp.
	\\Användaren skall ha läsrättigheter i wikigruppen.
\end{itemize}

\subsection{Skriva en artikel}
\begin{itemize}
	\item Kortfattad beskrivning
	\\En person skall skriva en artikel i sin wiki online.
	\item Aktörer
	\\En användare, person A, utan större datorvana.
	\item Förhandsvillkor
	\\Person A är online och har programmet installerat på sin dator.
	\item Händelseförlopp
	\begin{enumerate}
		\item Person A startar programmet.
		\item Person A väljer att skapa en ny artikel.
		\item Ett redigeringsfönster visas och person A skriver sin artikel.
		\item När person A är färdig väljer han att publicera sin artikel som nu kan hämtas och läsas av de andra medlemmarna i gruppen.
	\end{enumerate}
	\item Alternativt händelseförlopp
	\\Efter steg 1:
	\begin{enumerate}	
		\item Programmet synkroniserar ändringar i artiklar med andra medlemmar som är online i wikigruppen.
		\item Person A läser en uppdaterad artikel och bestämmer sig för att ändra på den.
		\item Person A väljer att redigera artikeln och ett redigeringsfönster med artikeln i visas.
		\item Person A gör de önskvärda ändringarna	och publicerar den nya artikeln	som nu kan hämtas och läsas av de andra medlemmarna i gruppen.
	\end{enumerate}
	\item Huvudscenario
	\\De viktigaste delarna i detta use-case är när person A redigerar en artikel i det redigerbara fönstret samt när person A är färdig och publicerar artikeln.
	\item Efterhandsvillkor
	\begin{itemize}
		\item Användaren sparar en artikel
		\\Person A skriver klart en artikel och anser att den är färdig och sparar dennna på wiki:n så att andra användaren kan komma åt den.
		\item Användaren avbryter skapandet av en artikel
		\\Person A börjar skriva en ny artikel men tycker inte att den är bra så han avbryter skapandet och alla ändringar han har gjort försvinner.
	\end{itemize}	
	\item Speciella krav
	\\Användaren skall kunna vara medlem i en speciell wikigrupp.
	\\Användaren skall ha läsrättigheter i wikigruppen.
	\\Användaren skall ha skrivrättigheter i wikigruppen.
\end{itemize}

\subsection{Ta tillbaks en gammal version av en artikel}
\begin{itemize}
	\item Kortfattad beskrivning
	\\En person finner att en artikel blivit felaktigt uppdaterad och vill återställa sin gamla kopia lokalt.
	\item Aktörer
	\\En användare av systemet, person A, utan större datorvana.
	\item Förhandsvillkor
	\\Person A är online och har programmet installerat på sin dator.
	\item Händelseförlopp
	\begin{enumerate}
		\item Person A startar programmet.
		\item Programmet synkroniserar ändringar i artiklar med andra medlemmar som är online i wikigruppen.
		\item Person A läser en av de uppdaterade artiklarna och ogillar de nya ändringarna.
		\item Person A väljer att återskapa en äldre version av artikeln och ett fönster med möjliga äldre versioner visas.
		\item Person A väljer den version av artikeln han vill ha och artikeln återskapas lokalt.
	\end{enumerate}	
	\item Huvudscenario
	\\Den viktigaste delen i detta use-case är att person A kan välja vilken version han vill gå tillbaks till.
	\item Efterhandsvillkor	
	\begin{itemize}
		\item Lyckades med att återskapa en artikel
		\\Användaren lyckades att återskapa en gammal version av en artikel.
		\item Lyckades inte att återskapa en artikel.
		\\Något gick fel på vägen och artikeln återskapades inte. Den nuvarande versionen finns kvar som innan.
	\end{itemize}
	\item Speciella krav
	\\Användaren skall kunna vara medlem i en speciell wikigrupp.
	\\Användaren skall ha läsrättigheter i wikigruppen(behövs endast för att ta emot nya uppdateringar).
	\\Användaren har ändringar av den aktuella artikeln sparade lokalt.
\end{itemize}


\subsection{Sammanslagning}
\begin{itemize}
	\item Kortfattad beskrivning
	\\En användare ska sammanfoga sina dokument eller annat arbete med andra personers arbeten.
	\item Aktörer
	\\Tre användare A, B och C som inte har någon större datorvana.
	\\Administratör för wiki-gruppen med lite mer datorvana.	
	\item Förhandsvillkor
	\\Ett distribuerat wiki-system är konfigurerat, satt igång och används.
	\item Händelseförlopp
	\begin{enumerate}
		\item Person A har arbetat med dokument på sin dator och vill sätta ihop detta med person B:s och person C:s dokument.
		\item Person A vet inte vad person B och person C har arbetat med men öppnar ändå sin wiki.
		\item Han går in på sidan för person B samt C:s dokument och därmed synkroniseras hans dokument automatiskt.
		\item Systemet försöker att sammanfoga dokumenten.
		\item Eftersom olika stycken är ändrade i dokumenten sker en konflikt.
		\item Användaren får en förfrågan om han vill lösa konflikten.
		\item Användaren får upp dokumenten bredvid varandra och kan sedan ändra det han behagar.
		\item Till sist sparar användaren dokumentet och känner sig nöjd.
	\end{enumerate}
	\item Alternativt händelseförlopp
	\\Efter steg 5:
	\begin{enumerate}	
		\item Användaren inte har rättigheter att ändra på filerna.
		\item Användaren kontaktar administratören för att få hjälp.
	\end{enumerate}	
	\item Alternativt händelseförlopp
	\\Efter steg 5: 
	\begin{enumerate}	
		\item Användaren inte känner sig tillräckligt säker att ändra på filerna.
		\item Användaren kontaktar då den/de personer som har skrivit filerna och ber om hjälp.
	\end{enumerate}
	\item Huvudscenario
	
	Det viktiga i detta use-case är att visa att man kan sammanfoga olika filer med varandra trots att rättigheter samt kompetens kan ställa till det.
	\item Efterhandsvillkor
		\begin{itemize}	
		\item Lyckad sammanfogning
		\\Användaren lyckades med att sammanfoga (merge) två eller fler filer.
		\item Icke lyckad sammanfogning
		\\Användaren lyckades ej med en sammanfogning.
		\end{itemize}
	\item Speciella krav
	\\Användaren ska vara medlem i en speciell wiki-grupp.
\end{itemize}

\subsection{Rättstavning}
\begin{itemize}
	\item Kortfattad beskrivning
	\\En användare orkar eller vill inte sitta och korrekturläsa hela artikeln för att hitta stavfel utan vill använda en automatisk rättstavare.
	\item Aktörer
	\\En användare, person A, utan direkt datorvana.
	\item Förhandvillkor
	\\Person A har programmet installerat på datorn
	\item Händelseförlopp
	\begin{enumerate}
		\item En användare vill eller har inte tid att läsa en artikel som han vill göra tillgängligt för andra användare.
		\item Användaren redigerar artikeln.
		\item Användaren väljer sedan att rättstava artikeln.
		\item För varje fel som rättstavningskontrollen hittar får användaren en fråga om han vill ändra det befintliga ordet till det som rättstavningskontrollen föreslår.
		\item När användaren är klar sparar han artikeln.
	\end{enumerate}
	\item Huvudscenario
	\\Det viktiga delen i det här use-caset är att en användare skall kunna använda automatisk rättstavning för att granska sin artikel efter stavfel.
	\item Post-condition
	\begin{itemize}
	\item Inga fel
	\\Rättstavningskontrollen hittar inga fel och godkänner artikeln direkt.
	\item Fel hittades
	\\Rättstavningskontrollen hittar något som den uppfattar som felstavat och användaren får en rättningsförfrågan.
	\end{itemize}	
	\item Speciella krav
	\\Användaren måste ha stavningskontrollen installerad.
\end{itemize}
\end{document}
