



<!DOCTYPE html PUBLIC "-//W3C//DTD XHTML 1.0 Transitional//EN"
  "http://www.w3.org/TR/xhtml1/DTD/xhtml1-transitional.dtd">

<html xmlns="http://www.w3.org/1999/xhtml" xml:lang="en" lang="en">
  <head>
    <meta http-equiv="content-type" content="text/html;charset=UTF-8" />
    <title>Dokument/Use-case.tex at 33b6a349addfd210501656690e7912f48236da11 from linusdunkers's pum1 - GitHub</title>
    <link rel="search" type="application/opensearchdescription+xml" href="/opensearch.xml" title="GitHub" />
    <link rel="fluid-icon" href="http://github.com/fluidicon.png" title="GitHub" />

    
      <link href="http://assets0.github.com/stylesheets/bundle.css?26bb592b1b97858ec56790eb481385b979cb6f72" media="screen" rel="stylesheet" type="text/css" />
    

    
      
        <script type="text/javascript" src="http://ajax.googleapis.com/ajax/libs/jquery/1.2.6/jquery.min.js"></script>
        <script src="http://assets3.github.com/javascripts/bundle.js?26bb592b1b97858ec56790eb481385b979cb6f72" type="text/javascript"></script>
      
    
    
  
    
  

  <link href="http://github.com/feeds/linusdunkers/commits/pum1/33b6a349addfd210501656690e7912f48236da11" rel="alternate" title="Recent Commits to pum1:33b6a349addfd210501656690e7912f48236da11" type="application/atom+xml" />

  <meta name="description" content="Distribuerad wiki" />


    
  </head>

  

  <body>
    

    <div id="main">
      <div id="header" class="">
        <div class="site">
          <div class="logo">
            <a href="http://github.com"><img src="/images/modules/header/logov3.png" alt="github" /></a>
          </div>
          
            <div class="topsearch">
  <form action="/search" id="top_search_form" method="get">
    <input type="search" class="search" name="q" /> <input type="submit" value="Search" />
    <input type="hidden" name="type" value="Everything" />
    <input type="hidden" name="repo" value="" />
    <input type="hidden" name="langOverride" value="">
    <input type="hidden" name="start_value" value="1" />
  </form>
  <div class="links">
    <a href="/repositories">Browse</a> | <a href="/guides">Guides</a> | <a href="/search">Advanced</a>
  </div>
</div>
            
  <div class="corner userbox">
    <div class="box">
      <div class="gravatar">
        <a href="/"><img alt="" height="40" src="http://www.gravatar.com/avatar/f73260242f57114d30039dc970de8482?s=40&amp;d=http%3A%2F%2Fgithub.com%2Fimages%2Fgravatars%2Fgravatar-40.png" width="40" /></a>
      </div>

      <div class="top">
        <div class="name">
          <a href="/">linusdunkers</a>
        </div>
        <div class="links">
          <a href="/account">account</a> |
          <a href="/linusdunkers">profile</a> |
          <a href="/logout">log out</a>
        </div>
      </div>

      <div class="bottom">
        <div class="select">
          <div class="site_links">
            <a href="/">dashboard</a> | <a href="http://gist.github.com/mine">gists</a>
          </div>

          <form action="/search" class="search_repos" method="get" style="display:none;">
          <input id="q" name="q" size="18" type="search" /> 
          <input type="submit" value="Search" />
          <a href="#" class="cancel_search_link">x</a>
          </form>
        </div>
        
        <div class="inbox"> <span><a href="/inbox">0</a></span> </div>
      </div>
    </div>
  </div>

          
        </div>
      </div>
      
      
        
  
  
    <div id="repo_menu">
      <div class="site">
        <ul>
          
            <li class="active"><a href="http://github.com/linusdunkers/pum1/tree/">Source</a></li>

            <li class=""><a href="http://github.com/linusdunkers/pum1/commits/">Commits</a></li>

            <li class=""><a href="/linusdunkers/pum1/network">Network (7)</a></li>

            
              <li class=""><a href="/linusdunkers/pum1/forkqueue">Fork Queue</a></li>
            

            <li class=""><a href="/linusdunkers/pum1/downloads">Downloads (0)</a></li>

            <li class=""><a href="http://wiki.github.com/linusdunkers/pum1">Wiki (1)</a></li>

            <li class=""><a href="/linusdunkers/pum1/graphs">Graphs</a></li>

            
              <li class=""><a href="/linusdunkers/pum1/edit">Admin</a></li>
            

          
        </ul>
      </div>
    </div>
  

  <div id="repo_sub_menu">
    <div class="site">
      <div class="joiner"></div>
      

      

      

      
    </div>
  </div>

  <div class="site">
    





<div id="repos">
  




  <div class="repo public">
    <div class="title">
      <div class="path">
        <a href="/linusdunkers">linusdunkers</a> / <b><a href="http://github.com/linusdunkers/pum1/tree">pum1</a></b>

        

          
            <a href="/linusdunkers/pum1/edit"><img alt="edit" class="button" src="http://assets3.github.com/images/modules/repos/edit_button.png?26bb592b1b97858ec56790eb481385b979cb6f72" /></a>
          

          
            
              
                <a href="/linusdunkers/pum1/pull_request/" class="pull_request_button"><img alt="pull request" class="button" src="http://assets3.github.com/images/modules/repos/pull_request_button.png?26bb592b1b97858ec56790eb481385b979cb6f72" /></a>
              
            

            
          

          <a href="/linusdunkers/pum1/toggle_watch" class="toggle_watch" style="display:none;"><img alt="watch" class="button" src="http://assets3.github.com/images/modules/repos/watch_button.png?26bb592b1b97858ec56790eb481385b979cb6f72" /></a><a href="/linusdunkers/pum1/toggle_watch" class="toggle_watch"><img alt="watch" class="button" src="http://assets2.github.com/images/modules/repos/unwatch_button.png?26bb592b1b97858ec56790eb481385b979cb6f72" /></a>

          
            <a href="#" id="download_button" rel="http://github.com/linusdunkers/pum1/archives/master"><img alt="download tarball" class="button" src="http://assets1.github.com/images/modules/repos/download_button.png?26bb592b1b97858ec56790eb481385b979cb6f72" /></a>
          
        
      </div>

      <div class="security private_security" style="display:none">
        <a href="#private_repo" rel="facebox"><img src="/images/icons/private.png" alt="private" /></a>
      </div>

      <div id="private_repo" class="hidden">
        This repository is private.
        All pages are served over SSL and all pushing and pulling is done over SSH.
        No one may fork, clone, or view it unless they are added as a <a href="/linusdunkers/pum1/edit">member</a>.

        <br/>
        <br/>
        Every repository with this icon (<img src="/images/icons/private.png" alt="private" />) is private.
      </div>

      <div class="security public_security" style="">
        <a href="#public_repo" rel="facebox"><img src="/images/icons/public.png" alt="public" /></a>
      </div>

      <div id="public_repo" class="hidden">
        This repository is public.
        Anyone may fork, clone, or view it.

        <br/>
        <br/>
        Every repository with this icon (<img src="/images/icons/public.png" alt="public" />) is public.
      </div>

      

      <div class="flexipill">
        <a href="/linusdunkers/pum1/network">
        <table cellpadding="0" cellspacing="0">
          <tr><td><img alt="Forks" src="http://assets0.github.com/images/modules/repos/pills/forks.png?26bb592b1b97858ec56790eb481385b979cb6f72" /></td><td class="middle"><span>7</span></td><td><img alt="Right" src="http://assets0.github.com/images/modules/repos/pills/right.png?26bb592b1b97858ec56790eb481385b979cb6f72" /></td></tr>
        </table>
        </a>
      </div>

      <div class="flexipill">
        <a href="/linusdunkers/pum1/watchers">
        <table cellpadding="0" cellspacing="0">
          <tr><td><img alt="Watchers" src="http://assets3.github.com/images/modules/repos/pills/watchers.png?26bb592b1b97858ec56790eb481385b979cb6f72" /></td><td class="middle"><span>3</span></td><td><img alt="Right" src="http://assets0.github.com/images/modules/repos/pills/right.png?26bb592b1b97858ec56790eb481385b979cb6f72" /></td></tr>
        </table>
        </a>
      </div>
    </div>
    <div class="meta">
      <table>
        
          <tr>
            <td class="label" colspan="2">
              <em>Fork of <a href="/pum1/pum1/tree">pum1/pum1</a></em>
            </td>
          </tr>
        
        
          <tr>
            <td class="label">Description:</td>
            <td>
              <span id="repository_description" rel="/linusdunkers/pum1/edit/update" class="edit">Distribuerad wiki</span>
              <a href="#description" class="edit_link action">edit</a>
            </td>
          </tr>
        

        
          
            <tr>
              <td class="label">Homepage:</td>
              <td>
                
                  <span id="repository_homepage" rel="/linusdunkers/pum1/edit/update" class="edit">http://tp-server.dlinkddns.com:8406</span>
                  <a href="#homepage" class="edit_link action">edit</a>
                
              </td>
            </tr>
          

          
            <tr>
              <td class="label">Public&nbsp;Clone&nbsp;URL:</td>
              
              <td>
                <a href="git://github.com/linusdunkers/pum1.git" class="git_url_facebox" rel="#git-clone">git://github.com/linusdunkers/pum1.git</a>
                      <object classid="clsid:d27cdb6e-ae6d-11cf-96b8-444553540000"
              width="110"
              height="14"
              id="clippy" >
      <param name="movie" value="/flash/clippy.swf"/>
      <param name="allowScriptAccess" value="always" />
      <param name="quality" value="high" />
      <param name="scale" value="noscale" />
      <param NAME="FlashVars" value="text=git://github.com/linusdunkers/pum1.git">
      <param name="bgcolor" value="#F0F0F0">
      <embed src="/flash/clippy.swf"
             width="110"
             height="14"
             name="clippy"
             quality="high"
             allowScriptAccess="always"
             type="application/x-shockwave-flash"
             pluginspage="http://www.macromedia.com/go/getflashplayer"
             FlashVars="text=git://github.com/linusdunkers/pum1.git"
             bgcolor="#F0F0F0"
      />
      </object>

                <div id="git-clone" style="display:none;">
                  Give this clone URL to anyone.
                  <br/>
                  <code>git clone git://github.com/linusdunkers/pum1.git </code>
                </div>
              </td>
            </tr>
          
          
          <tr>
            <td class="label">Your Clone URL:</td>
            
            <td>

              <div id="private-clone-url">
                <a href="git@github.com:linusdunkers/pum1.git" class="git_url_facebox" rel="#your-git-clone">git@github.com:linusdunkers/pum1.git</a>
                <input type="text" value="git@github.com:linusdunkers/pum1.git" style="display: none;" />
                      <object classid="clsid:d27cdb6e-ae6d-11cf-96b8-444553540000"
              width="110"
              height="14"
              id="clippy" >
      <param name="movie" value="/flash/clippy.swf"/>
      <param name="allowScriptAccess" value="always" />
      <param name="quality" value="high" />
      <param name="scale" value="noscale" />
      <param NAME="FlashVars" value="text=git@github.com:linusdunkers/pum1.git">
      <param name="bgcolor" value="#F0F0F0">
      <embed src="/flash/clippy.swf"
             width="110"
             height="14"
             name="clippy"
             quality="high"
             allowScriptAccess="always"
             type="application/x-shockwave-flash"
             pluginspage="http://www.macromedia.com/go/getflashplayer"
             FlashVars="text=git@github.com:linusdunkers/pum1.git"
             bgcolor="#F0F0F0"
      />
      </object>

              </div>

              <div id="your-git-clone" style="display:none;">
                Use this clone URL yourself.
                <br/>
                <code>git clone git@github.com:linusdunkers/pum1.git </code>
              </div>
            </td>
          </tr>
          
          

          

          
      </table>

          </div>
  </div>




</div>


  <div id="commit">
    <div class="group">
        
  <div class="envelope commit">
    <div class="human">
      
        <div class="message"><pre><a href="/linusdunkers/pum1/commit/33b6a349addfd210501656690e7912f48236da11">delar av dokumentet use-case.tex granskat</a> </pre></div>
      

      <div class="actor">
        <div class="gravatar">
          
          <img alt="" height="30" src="http://www.gravatar.com/avatar/f73260242f57114d30039dc970de8482?s=30&amp;d=http%3A%2F%2Fgithub.com%2Fimages%2Fgravatars%2Fgravatar-30.png" width="30" />
        </div>
        <div class="name"><a href="/linusdunkers">linusdunkers</a> <span>(author)</span></div>
          <div class="date">
            <abbr class="relatize" title="2009-02-11 08:47:01">Wed Feb 11 08:47:01 -0800 2009</abbr> 
          </div>
      </div>
  
      
  
    </div>
    <div class="machine">
      <span>c</span>ommit&nbsp;&nbsp;<a href="/linusdunkers/pum1/commit/33b6a349addfd210501656690e7912f48236da11" hotkey="c">33b6a349addfd210501656690e7912f48236da11</a><br />
      <span>t</span>ree&nbsp;&nbsp;&nbsp;&nbsp;<a href="/linusdunkers/pum1/tree/33b6a349addfd210501656690e7912f48236da11" hotkey="t">293822789e8e71b9e62764d816f89f55c6df6fba</a><br />
  
      
        <span>p</span>arent&nbsp;
        
        <a href="/linusdunkers/pum1/tree/1cdaab812e7ae81023979ed68550057629291572" hotkey="p">1cdaab812e7ae81023979ed68550057629291572</a>
      
  
    </div>
  </div>

    </div>
  </div>





  
    <div id="path">
      <b><a href="/linusdunkers/pum1/tree">pum1</a></b> / <a href="/linusdunkers/pum1/tree/33b6a349addfd210501656690e7912f48236da11/Dokument">Dokument</a> / Use-case.tex
    </div>

    <div id="files">
      <div class="file">
        <div class="meta">
          <div class="info">
            <span>100644</span>
            <span>276 lines (253 sloc)</span>
            <span>12.392 kb</span>
          </div>
          <div class="actions">
            
              <a id="file-edit-link" href="#" rel="/linusdunkers/pum1/file-edit/33b6a349addfd210501656690e7912f48236da11/Dokument/Use-case.tex">edit</a>
            
            <a href="/linusdunkers/pum1/raw/33b6a349addfd210501656690e7912f48236da11/Dokument/Use-case.tex" id="raw-url">raw</a>
            
              <a href="/linusdunkers/pum1/blame/33b6a349addfd210501656690e7912f48236da11/Dokument/Use-case.tex">blame</a>
            
            <a href="/linusdunkers/pum1/commits/master/Dokument/Use-case.tex">history</a>
          </div>
        </div>
        
  <div class="data syntax">
    
      <table cellpadding="0" cellspacing="0">
        <tr>
          <td>
            
            <pre class="line_numbers">
<span id="LID1" rel="#L1">1</span>
<span id="LID2" rel="#L2">2</span>
<span id="LID3" rel="#L3">3</span>
<span id="LID4" rel="#L4">4</span>
<span id="LID5" rel="#L5">5</span>
<span id="LID6" rel="#L6">6</span>
<span id="LID7" rel="#L7">7</span>
<span id="LID8" rel="#L8">8</span>
<span id="LID9" rel="#L9">9</span>
<span id="LID10" rel="#L10">10</span>
<span id="LID11" rel="#L11">11</span>
<span id="LID12" rel="#L12">12</span>
<span id="LID13" rel="#L13">13</span>
<span id="LID14" rel="#L14">14</span>
<span id="LID15" rel="#L15">15</span>
<span id="LID16" rel="#L16">16</span>
<span id="LID17" rel="#L17">17</span>
<span id="LID18" rel="#L18">18</span>
<span id="LID19" rel="#L19">19</span>
<span id="LID20" rel="#L20">20</span>
<span id="LID21" rel="#L21">21</span>
<span id="LID22" rel="#L22">22</span>
<span id="LID23" rel="#L23">23</span>
<span id="LID24" rel="#L24">24</span>
<span id="LID25" rel="#L25">25</span>
<span id="LID26" rel="#L26">26</span>
<span id="LID27" rel="#L27">27</span>
<span id="LID28" rel="#L28">28</span>
<span id="LID29" rel="#L29">29</span>
<span id="LID30" rel="#L30">30</span>
<span id="LID31" rel="#L31">31</span>
<span id="LID32" rel="#L32">32</span>
<span id="LID33" rel="#L33">33</span>
<span id="LID34" rel="#L34">34</span>
<span id="LID35" rel="#L35">35</span>
<span id="LID36" rel="#L36">36</span>
<span id="LID37" rel="#L37">37</span>
<span id="LID38" rel="#L38">38</span>
<span id="LID39" rel="#L39">39</span>
<span id="LID40" rel="#L40">40</span>
<span id="LID41" rel="#L41">41</span>
<span id="LID42" rel="#L42">42</span>
<span id="LID43" rel="#L43">43</span>
<span id="LID44" rel="#L44">44</span>
<span id="LID45" rel="#L45">45</span>
<span id="LID46" rel="#L46">46</span>
<span id="LID47" rel="#L47">47</span>
<span id="LID48" rel="#L48">48</span>
<span id="LID49" rel="#L49">49</span>
<span id="LID50" rel="#L50">50</span>
<span id="LID51" rel="#L51">51</span>
<span id="LID52" rel="#L52">52</span>
<span id="LID53" rel="#L53">53</span>
<span id="LID54" rel="#L54">54</span>
<span id="LID55" rel="#L55">55</span>
<span id="LID56" rel="#L56">56</span>
<span id="LID57" rel="#L57">57</span>
<span id="LID58" rel="#L58">58</span>
<span id="LID59" rel="#L59">59</span>
<span id="LID60" rel="#L60">60</span>
<span id="LID61" rel="#L61">61</span>
<span id="LID62" rel="#L62">62</span>
<span id="LID63" rel="#L63">63</span>
<span id="LID64" rel="#L64">64</span>
<span id="LID65" rel="#L65">65</span>
<span id="LID66" rel="#L66">66</span>
<span id="LID67" rel="#L67">67</span>
<span id="LID68" rel="#L68">68</span>
<span id="LID69" rel="#L69">69</span>
<span id="LID70" rel="#L70">70</span>
<span id="LID71" rel="#L71">71</span>
<span id="LID72" rel="#L72">72</span>
<span id="LID73" rel="#L73">73</span>
<span id="LID74" rel="#L74">74</span>
<span id="LID75" rel="#L75">75</span>
<span id="LID76" rel="#L76">76</span>
<span id="LID77" rel="#L77">77</span>
<span id="LID78" rel="#L78">78</span>
<span id="LID79" rel="#L79">79</span>
<span id="LID80" rel="#L80">80</span>
<span id="LID81" rel="#L81">81</span>
<span id="LID82" rel="#L82">82</span>
<span id="LID83" rel="#L83">83</span>
<span id="LID84" rel="#L84">84</span>
<span id="LID85" rel="#L85">85</span>
<span id="LID86" rel="#L86">86</span>
<span id="LID87" rel="#L87">87</span>
<span id="LID88" rel="#L88">88</span>
<span id="LID89" rel="#L89">89</span>
<span id="LID90" rel="#L90">90</span>
<span id="LID91" rel="#L91">91</span>
<span id="LID92" rel="#L92">92</span>
<span id="LID93" rel="#L93">93</span>
<span id="LID94" rel="#L94">94</span>
<span id="LID95" rel="#L95">95</span>
<span id="LID96" rel="#L96">96</span>
<span id="LID97" rel="#L97">97</span>
<span id="LID98" rel="#L98">98</span>
<span id="LID99" rel="#L99">99</span>
<span id="LID100" rel="#L100">100</span>
<span id="LID101" rel="#L101">101</span>
<span id="LID102" rel="#L102">102</span>
<span id="LID103" rel="#L103">103</span>
<span id="LID104" rel="#L104">104</span>
<span id="LID105" rel="#L105">105</span>
<span id="LID106" rel="#L106">106</span>
<span id="LID107" rel="#L107">107</span>
<span id="LID108" rel="#L108">108</span>
<span id="LID109" rel="#L109">109</span>
<span id="LID110" rel="#L110">110</span>
<span id="LID111" rel="#L111">111</span>
<span id="LID112" rel="#L112">112</span>
<span id="LID113" rel="#L113">113</span>
<span id="LID114" rel="#L114">114</span>
<span id="LID115" rel="#L115">115</span>
<span id="LID116" rel="#L116">116</span>
<span id="LID117" rel="#L117">117</span>
<span id="LID118" rel="#L118">118</span>
<span id="LID119" rel="#L119">119</span>
<span id="LID120" rel="#L120">120</span>
<span id="LID121" rel="#L121">121</span>
<span id="LID122" rel="#L122">122</span>
<span id="LID123" rel="#L123">123</span>
<span id="LID124" rel="#L124">124</span>
<span id="LID125" rel="#L125">125</span>
<span id="LID126" rel="#L126">126</span>
<span id="LID127" rel="#L127">127</span>
<span id="LID128" rel="#L128">128</span>
<span id="LID129" rel="#L129">129</span>
<span id="LID130" rel="#L130">130</span>
<span id="LID131" rel="#L131">131</span>
<span id="LID132" rel="#L132">132</span>
<span id="LID133" rel="#L133">133</span>
<span id="LID134" rel="#L134">134</span>
<span id="LID135" rel="#L135">135</span>
<span id="LID136" rel="#L136">136</span>
<span id="LID137" rel="#L137">137</span>
<span id="LID138" rel="#L138">138</span>
<span id="LID139" rel="#L139">139</span>
<span id="LID140" rel="#L140">140</span>
<span id="LID141" rel="#L141">141</span>
<span id="LID142" rel="#L142">142</span>
<span id="LID143" rel="#L143">143</span>
<span id="LID144" rel="#L144">144</span>
<span id="LID145" rel="#L145">145</span>
<span id="LID146" rel="#L146">146</span>
<span id="LID147" rel="#L147">147</span>
<span id="LID148" rel="#L148">148</span>
<span id="LID149" rel="#L149">149</span>
<span id="LID150" rel="#L150">150</span>
<span id="LID151" rel="#L151">151</span>
<span id="LID152" rel="#L152">152</span>
<span id="LID153" rel="#L153">153</span>
<span id="LID154" rel="#L154">154</span>
<span id="LID155" rel="#L155">155</span>
<span id="LID156" rel="#L156">156</span>
<span id="LID157" rel="#L157">157</span>
<span id="LID158" rel="#L158">158</span>
<span id="LID159" rel="#L159">159</span>
<span id="LID160" rel="#L160">160</span>
<span id="LID161" rel="#L161">161</span>
<span id="LID162" rel="#L162">162</span>
<span id="LID163" rel="#L163">163</span>
<span id="LID164" rel="#L164">164</span>
<span id="LID165" rel="#L165">165</span>
<span id="LID166" rel="#L166">166</span>
<span id="LID167" rel="#L167">167</span>
<span id="LID168" rel="#L168">168</span>
<span id="LID169" rel="#L169">169</span>
<span id="LID170" rel="#L170">170</span>
<span id="LID171" rel="#L171">171</span>
<span id="LID172" rel="#L172">172</span>
<span id="LID173" rel="#L173">173</span>
<span id="LID174" rel="#L174">174</span>
<span id="LID175" rel="#L175">175</span>
<span id="LID176" rel="#L176">176</span>
<span id="LID177" rel="#L177">177</span>
<span id="LID178" rel="#L178">178</span>
<span id="LID179" rel="#L179">179</span>
<span id="LID180" rel="#L180">180</span>
<span id="LID181" rel="#L181">181</span>
<span id="LID182" rel="#L182">182</span>
<span id="LID183" rel="#L183">183</span>
<span id="LID184" rel="#L184">184</span>
<span id="LID185" rel="#L185">185</span>
<span id="LID186" rel="#L186">186</span>
<span id="LID187" rel="#L187">187</span>
<span id="LID188" rel="#L188">188</span>
<span id="LID189" rel="#L189">189</span>
<span id="LID190" rel="#L190">190</span>
<span id="LID191" rel="#L191">191</span>
<span id="LID192" rel="#L192">192</span>
<span id="LID193" rel="#L193">193</span>
<span id="LID194" rel="#L194">194</span>
<span id="LID195" rel="#L195">195</span>
<span id="LID196" rel="#L196">196</span>
<span id="LID197" rel="#L197">197</span>
<span id="LID198" rel="#L198">198</span>
<span id="LID199" rel="#L199">199</span>
<span id="LID200" rel="#L200">200</span>
<span id="LID201" rel="#L201">201</span>
<span id="LID202" rel="#L202">202</span>
<span id="LID203" rel="#L203">203</span>
<span id="LID204" rel="#L204">204</span>
<span id="LID205" rel="#L205">205</span>
<span id="LID206" rel="#L206">206</span>
<span id="LID207" rel="#L207">207</span>
<span id="LID208" rel="#L208">208</span>
<span id="LID209" rel="#L209">209</span>
<span id="LID210" rel="#L210">210</span>
<span id="LID211" rel="#L211">211</span>
<span id="LID212" rel="#L212">212</span>
<span id="LID213" rel="#L213">213</span>
<span id="LID214" rel="#L214">214</span>
<span id="LID215" rel="#L215">215</span>
<span id="LID216" rel="#L216">216</span>
<span id="LID217" rel="#L217">217</span>
<span id="LID218" rel="#L218">218</span>
<span id="LID219" rel="#L219">219</span>
<span id="LID220" rel="#L220">220</span>
<span id="LID221" rel="#L221">221</span>
<span id="LID222" rel="#L222">222</span>
<span id="LID223" rel="#L223">223</span>
<span id="LID224" rel="#L224">224</span>
<span id="LID225" rel="#L225">225</span>
<span id="LID226" rel="#L226">226</span>
<span id="LID227" rel="#L227">227</span>
<span id="LID228" rel="#L228">228</span>
<span id="LID229" rel="#L229">229</span>
<span id="LID230" rel="#L230">230</span>
<span id="LID231" rel="#L231">231</span>
<span id="LID232" rel="#L232">232</span>
<span id="LID233" rel="#L233">233</span>
<span id="LID234" rel="#L234">234</span>
<span id="LID235" rel="#L235">235</span>
<span id="LID236" rel="#L236">236</span>
<span id="LID237" rel="#L237">237</span>
<span id="LID238" rel="#L238">238</span>
<span id="LID239" rel="#L239">239</span>
<span id="LID240" rel="#L240">240</span>
<span id="LID241" rel="#L241">241</span>
<span id="LID242" rel="#L242">242</span>
<span id="LID243" rel="#L243">243</span>
<span id="LID244" rel="#L244">244</span>
<span id="LID245" rel="#L245">245</span>
<span id="LID246" rel="#L246">246</span>
<span id="LID247" rel="#L247">247</span>
<span id="LID248" rel="#L248">248</span>
<span id="LID249" rel="#L249">249</span>
<span id="LID250" rel="#L250">250</span>
<span id="LID251" rel="#L251">251</span>
<span id="LID252" rel="#L252">252</span>
<span id="LID253" rel="#L253">253</span>
<span id="LID254" rel="#L254">254</span>
<span id="LID255" rel="#L255">255</span>
<span id="LID256" rel="#L256">256</span>
<span id="LID257" rel="#L257">257</span>
<span id="LID258" rel="#L258">258</span>
<span id="LID259" rel="#L259">259</span>
<span id="LID260" rel="#L260">260</span>
<span id="LID261" rel="#L261">261</span>
<span id="LID262" rel="#L262">262</span>
<span id="LID263" rel="#L263">263</span>
<span id="LID264" rel="#L264">264</span>
<span id="LID265" rel="#L265">265</span>
<span id="LID266" rel="#L266">266</span>
<span id="LID267" rel="#L267">267</span>
<span id="LID268" rel="#L268">268</span>
<span id="LID269" rel="#L269">269</span>
<span id="LID270" rel="#L270">270</span>
<span id="LID271" rel="#L271">271</span>
<span id="LID272" rel="#L272">272</span>
<span id="LID273" rel="#L273">273</span>
<span id="LID274" rel="#L274">274</span>
<span id="LID275" rel="#L275">275</span>
<span id="LID276" rel="#L276">276</span>
</pre>
          </td>
          <td width="100%">
            
            
              <div class="highlight"><pre><div class="line" id="LC1"><span class="c">% Standardinkluderingsfil</span></div><div class="line" id="LC2"><span class="k">\input</span><span class="nb">{</span>standard<span class="nb">}</span></div><div class="line" id="LC3">&nbsp;</div><div class="line" id="LC4"><span class="k">\ifpdf</span></div><div class="line" id="LC5">&nbsp;&nbsp;<span class="k">\DeclareGraphicsExtensions</span><span class="nb">{</span>.pdf, .jpg, .tif, .png<span class="nb">}</span></div><div class="line" id="LC6">&nbsp;&nbsp;<span class="k">\pdfinfo</span><span class="nb">{</span>            </div><div class="line" id="LC7">&nbsp;&nbsp;&nbsp;&nbsp;/Title  (Use-case)</div><div class="line" id="LC8">&nbsp;&nbsp;&nbsp;&nbsp;/Author (PUM-grupp 1)</div><div class="line" id="LC9">&nbsp;&nbsp;<span class="nb">}</span></div><div class="line" id="LC10"><span class="k">\else</span></div><div class="line" id="LC11">&nbsp;&nbsp;<span class="k">\DeclareGraphicsExtensions</span><span class="nb">{</span>.eps, .jpg<span class="nb">}</span></div><div class="line" id="LC12"><span class="k">\fi</span></div><div class="line" id="LC13">&nbsp;</div><div class="line" id="LC14"><span class="k">\title</span><span class="nb">{</span>Use-case<span class="nb">}</span></div><div class="line" id="LC15"><span class="k">\author</span><span class="nb">{</span>PUM-grupp 1<span class="nb">}</span></div><div class="line" id="LC16"><span class="k">\date</span><span class="nb">{</span><span class="k">\today</span><span class="nb">}</span></div><div class="line" id="LC17">&nbsp;</div><div class="line" id="LC18"><span class="k">\begin</span><span class="nb">{</span>document<span class="nb">}</span></div><div class="line" id="LC19">&nbsp;</div><div class="line" id="LC20"><span class="k">\maketitle\thispagestyle</span><span class="nb">{</span>empty<span class="nb">}</span></div><div class="line" id="LC21">&nbsp;</div><div class="line" id="LC22"><span class="k">\newpage</span></div><div class="line" id="LC23">&nbsp;</div><div class="line" id="LC24"><span class="k">\section</span><span class="nb">{</span>Inledning<span class="nb">}</span></div><div class="line" id="LC25">&nbsp;</div><div class="line" id="LC26"><span class="k">\textsc</span><span class="nb">{</span><span class="k">\LARGE</span> Use-case<span class="nb">}</span></div><div class="line" id="LC27">&nbsp;</div><div class="line" id="LC28"><span class="k">\subsection</span><span class="nb">{</span>Exempel<span class="nb">}</span></div><div class="line" id="LC29"><span class="k">\begin</span><span class="nb">{</span>itemize<span class="nb">}</span></div><div class="line" id="LC30">&nbsp;&nbsp;<span class="k">\item</span> Kortfattad beskrivning</div><div class="line" id="LC31">&nbsp;&nbsp;Kort beskrivning av use-case.</div><div class="line" id="LC32">&nbsp;&nbsp;<span class="k">\item</span> Aktörer</div><div class="line" id="LC33">&nbsp;&nbsp;Beskrivning av de aktörer som medverkar.</div><div class="line" id="LC34">&nbsp;&nbsp;<span class="k">\item</span> Förhandsvillkor</div><div class="line" id="LC35">&nbsp;&nbsp;Antaganden för att fallet ska kunna genomföras.</div><div class="line" id="LC36">&nbsp;&nbsp;<span class="k">\item</span> Händelseförlopp</div><div class="line" id="LC37">&nbsp;&nbsp;Stegvis genomgång av händelseförloppet.</div><div class="line" id="LC38">&nbsp;&nbsp;<span class="k">\item</span> Alternativt händelseförlopp</div><div class="line" id="LC39">&nbsp;&nbsp;Andra sätt fallet kan utspela sig på, till exempel att något går fel.</div><div class="line" id="LC40">&nbsp;&nbsp;<span class="k">\item</span> Huvudscenario</div><div class="line" id="LC41">&nbsp;&nbsp;Det viktigaste scenariot i fallet.</div><div class="line" id="LC42">&nbsp;&nbsp;<span class="k">\item</span> Efterhandsvillkor</div><div class="line" id="LC43">&nbsp;&nbsp;Möjliga utgångar av fallet.</div><div class="line" id="LC44">&nbsp;&nbsp;<span class="k">\item</span> Speciella krav</div><div class="line" id="LC45">&nbsp;&nbsp;Speciella krav för fallet.</div><div class="line" id="LC46"><span class="k">\end</span><span class="nb">{</span>itemize<span class="nb">}</span></div><div class="line" id="LC47">&nbsp;</div><div class="line" id="LC48"><span class="k">\section</span><span class="nb">{</span>Use-cases<span class="nb">}</span> </div><div class="line" id="LC49">&nbsp;</div><div class="line" id="LC50">För att lättare förstå vad användarna ska kunna åstadkomma med vårt system har gruppen satt upp ett antal use-cases för att försöka täcka de olika funktioner som kommer att finnas.</div><div class="line" id="LC51">&nbsp;</div><div class="line" id="LC52"><span class="k">\subsection</span><span class="nb">{</span>Bläddra<span class="nb">}</span></div><div class="line" id="LC53">&nbsp;</div><div class="line" id="LC54"><span class="k">\begin</span><span class="nb">{</span>itemize<span class="nb">}</span></div><div class="line" id="LC55">&nbsp;&nbsp;<span class="k">\item</span> Kortfattad beskrivning</div><div class="line" id="LC56">&nbsp;&nbsp;<span class="k">\\</span>En person vill finna en speciell artikel.</div><div class="line" id="LC57">&nbsp;&nbsp;<span class="k">\item</span> Aktörer</div><div class="line" id="LC58">&nbsp;&nbsp;<span class="k">\\</span>En användare, person A, utan större datorvana.</div><div class="line" id="LC59">&nbsp;&nbsp;<span class="k">\item</span> Förhandsvillkor</div><div class="line" id="LC60">&nbsp;&nbsp;<span class="k">\\</span>Person A har programmet installerat på sin dator.</div><div class="line" id="LC61">&nbsp;&nbsp;<span class="k">\item</span> Händelseförlopp</div><div class="line" id="LC62">&nbsp;&nbsp;<span class="k">\begin</span><span class="nb">{</span>enumerate<span class="nb">}</span></div><div class="line" id="LC63">&nbsp;&nbsp;&nbsp;&nbsp;<span class="k">\item</span> Person A startar programmet.</div><div class="line" id="LC64">&nbsp;&nbsp;&nbsp;&nbsp;<span class="k">\item</span> Person A skriver in sökord som matchar artikeln han vill söka efter i en sökruta.</div><div class="line" id="LC65">&nbsp;&nbsp;&nbsp;&nbsp;<span class="k">\item</span> En lista med möjliga artiklar dyker upp och person A väljer en av dem.</div><div class="line" id="LC66">&nbsp;&nbsp;&nbsp;&nbsp;<span class="k">\item</span> Artikeln visas i ett fönster och person A kan läsa den.</div><div class="line" id="LC67">&nbsp;&nbsp;<span class="k">\end</span><span class="nb">{</span>enumerate<span class="nb">}</span></div><div class="line" id="LC68">&nbsp;&nbsp;<span class="k">\item</span> Alternativt händelseförlopp</div><div class="line" id="LC69">&nbsp;&nbsp;<span class="k">\begin</span><span class="nb">{</span>enumerate<span class="nb">}</span></div><div class="line" id="LC70">&nbsp;&nbsp;&nbsp;&nbsp;<span class="k">\item</span> Person A startar programmet.</div><div class="line" id="LC71">&nbsp;&nbsp;&nbsp;&nbsp;<span class="k">\item</span> Person A navigerar till sin önskade artikel genom länkar från andra artiklar.&nbsp;&nbsp;</div><div class="line" id="LC72">&nbsp;&nbsp;&nbsp;&nbsp;<span class="k">\item</span> Artikeln visas i ett fönster och person A kan läsa den.</div><div class="line" id="LC73">&nbsp;&nbsp;<span class="k">\end</span><span class="nb">{</span>enumerate<span class="nb">}</span>&nbsp;&nbsp;</div><div class="line" id="LC74">&nbsp;&nbsp;<span class="k">\item</span> Huvudscenario</div><div class="line" id="LC75">&nbsp;&nbsp;<span class="k">\\</span>Den viktigaste delen i detta use-case är att person A kan välja mellan två sätt att finna sin artikel. Antingen med hjälp av länkar från andra artiklar eller genom en sökfunktion.</div><div class="line" id="LC76">&nbsp;&nbsp;<span class="k">\item</span> Efterhandsvillkor</div><div class="line" id="LC77">&nbsp;&nbsp;<span class="k">\begin</span><span class="nb">{</span>itemize<span class="nb">}</span></div><div class="line" id="LC78">&nbsp;&nbsp;<span class="k">\item</span> Fann önskad artikel</div><div class="line" id="LC79">&nbsp;&nbsp;<span class="k">\\</span>Användaren fick tag på den artikel som han/hon önskade</div><div class="line" id="LC80">&nbsp;&nbsp;<span class="k">\item</span> Fann inte önskad artikel</div><div class="line" id="LC81">&nbsp;&nbsp;<span class="k">\\</span>Användaren fick inte tag på den artikel han/hon önskade, antingen på grund av bristande sökord, felaktiga länkningar eller så kanske inte användarens wiki fått hem alla artiklar från övriga användare i gruppen.</div><div class="line" id="LC82">&nbsp;&nbsp;<span class="k">\end</span><span class="nb">{</span>itemize<span class="nb">}</span></div><div class="line" id="LC83">&nbsp;&nbsp;<span class="k">\item</span> Speciella krav</div><div class="line" id="LC84">&nbsp;&nbsp;<span class="k">\\</span>En startsida på användarens wiki med länkar till andra wiki-sidor krävs för det alternativa händelseförloppet.</div><div class="line" id="LC85"><span class="k">\end</span><span class="nb">{</span>itemize<span class="nb">}</span></div><div class="line" id="LC86">&nbsp;</div><div class="line" id="LC87"><span class="k">\subsection</span><span class="nb">{</span>Allmän distribuering<span class="nb">}</span></div><div class="line" id="LC88"><span class="k">\begin</span><span class="nb">{</span>itemize<span class="nb">}</span></div><div class="line" id="LC89">&nbsp;&nbsp;<span class="k">\item</span> Kortfattad beskrivning</div><div class="line" id="LC90">&nbsp;&nbsp;<span class="k">\\</span>En person skriver en artikel, en annan person läser artikeln och en tredje redigerar den.</div><div class="line" id="LC91">&nbsp;&nbsp;<span class="k">\item</span> Aktörer</div><div class="line" id="LC92">&nbsp;&nbsp;<span class="k">\\</span>Tre användare A, B och C som alla är utan någon större datorvana.</div><div class="line" id="LC93">&nbsp;&nbsp;<span class="k">\item</span> Förhandsvillkor</div><div class="line" id="LC94">&nbsp;&nbsp;<span class="k">\\</span>Personerna A, B och C har programmet installerat på sin dator. De har oskcå programmet igång och är anslutna till Internet.</div><div class="line" id="LC95">&nbsp;&nbsp;<span class="k">\item</span> Händelseförlopp</div><div class="line" id="LC96">&nbsp;&nbsp;<span class="k">\begin</span><span class="nb">{</span>enumerate<span class="nb">}</span></div><div class="line" id="LC97">&nbsp;&nbsp;&nbsp;&nbsp;<span class="k">\item</span> Person A skriver en ny artikel.</div><div class="line" id="LC98">&nbsp;&nbsp;&nbsp;&nbsp;<span class="k">\item</span> Person B och C får automatiskt artikeln från person A.</div><div class="line" id="LC99">&nbsp;&nbsp;&nbsp;&nbsp;<span class="k">\item</span> Person B läser artikeln.</div><div class="line" id="LC100">&nbsp;&nbsp;&nbsp;&nbsp;<span class="k">\item</span> Person C väljer att redigera artikeln.</div><div class="line" id="LC101">&nbsp;&nbsp;&nbsp;&nbsp;<span class="k">\item</span> Person A och B får automatiskt ändringarna i artikeln från person C.</div><div class="line" id="LC102">&nbsp;&nbsp;<span class="k">\end</span><span class="nb">{</span>enumerate<span class="nb">}</span>&nbsp;&nbsp;</div><div class="line" id="LC103">&nbsp;&nbsp;<span class="k">\item</span> Huvudscenario</div><div class="line" id="LC104">&nbsp;&nbsp;<span class="k">\\</span>Den viktigaste delen i detta use-case är att visa på vilket sätt ändringar sprids i ett distribuerat system.</div><div class="line" id="LC105">&nbsp;&nbsp;<span class="k">\item</span> Efterhandsvillkor</div><div class="line" id="LC106">&nbsp;&nbsp;<span class="k">\\</span>Detta use-case utgår från att flera användare har programmet och är medlemmar i samma wikigrupp.</div><div class="line" id="LC107">&nbsp;&nbsp;<span class="k">\item</span> Speciella krav</div><div class="line" id="LC108">&nbsp;&nbsp;<span class="k">\begin</span><span class="nb">{</span>enumerate<span class="nb">}</span>&nbsp;&nbsp;</div><div class="line" id="LC109">&nbsp;&nbsp;&nbsp;&nbsp;<span class="k">\item</span> Användarna skall kunna vara medlem i en speciell wikigrupp.</div><div class="line" id="LC110">&nbsp;&nbsp;&nbsp;&nbsp;<span class="k">\item</span> Användare A skall ha skriv och läs-rättigheter i wikigruppen.</div><div class="line" id="LC111">&nbsp;&nbsp;&nbsp;&nbsp;<span class="k">\item</span> Användare B skall ha läsrättigheter i wikigruppen.</div><div class="line" id="LC112">&nbsp;&nbsp;&nbsp;&nbsp;<span class="k">\item</span> Användare C skall ha skriv och läs-rättigheter i wikigruppen.</div><div class="line" id="LC113">&nbsp;&nbsp;<span class="k">\end</span><span class="nb">{</span>enumerate<span class="nb">}</span></div><div class="line" id="LC114"><span class="k">\end</span><span class="nb">{</span>itemize<span class="nb">}</span></div><div class="line" id="LC115">&nbsp;</div><div class="line" id="LC116">&nbsp;</div><div class="line" id="LC117"><span class="k">\subsection</span><span class="nb">{</span>Läsa en artikel<span class="nb">}</span></div><div class="line" id="LC118"><span class="k">\begin</span><span class="nb">{</span>itemize<span class="nb">}</span></div><div class="line" id="LC119">&nbsp;&nbsp;<span class="k">\item</span> Kortfattad beskrivning</div><div class="line" id="LC120">&nbsp;&nbsp;<span class="k">\\</span>En person skall läsa en artikel i sin wiki online.</div><div class="line" id="LC121">&nbsp;&nbsp;<span class="k">\item</span> Aktörer</div><div class="line" id="LC122">&nbsp;&nbsp;<span class="k">\\</span>En användare person A utan större datorvana.</div><div class="line" id="LC123">&nbsp;&nbsp;<span class="k">\item</span> Förhandsvillkor</div><div class="line" id="LC124">&nbsp;&nbsp;<span class="k">\\</span>Person A är online och har programmet installerat på sin dator.</div><div class="line" id="LC125">&nbsp;&nbsp;<span class="k">\item</span> Händelseförlopp</div><div class="line" id="LC126">&nbsp;&nbsp;<span class="k">\begin</span><span class="nb">{</span>enumerate<span class="nb">}</span></div><div class="line" id="LC127">&nbsp;&nbsp;&nbsp;&nbsp;<span class="k">\item</span> Person A startar programmet.</div><div class="line" id="LC128">&nbsp;&nbsp;&nbsp;&nbsp;<span class="k">\item</span> Programmet synkroniserar ändringar i artiklar med andra medlemmar som är online i wikigruppen.</div><div class="line" id="LC129">&nbsp;&nbsp;&nbsp;&nbsp;<span class="k">\item</span> Person A väljer en uppdaterad artikel han/hon vill läsa.</div><div class="line" id="LC130">&nbsp;&nbsp;&nbsp;&nbsp;<span class="k">\item</span> Artikeln visas och person A läser den.</div><div class="line" id="LC131">&nbsp;&nbsp;<span class="k">\end</span><span class="nb">{</span>enumerate<span class="nb">}</span></div><div class="line" id="LC132">&nbsp;&nbsp;<span class="k">\item</span> Alternativt händelseförlopp</div><div class="line" id="LC133">&nbsp;&nbsp;<span class="k">\\</span>Efter steg 1:</div><div class="line" id="LC134">&nbsp;&nbsp;<span class="k">\begin</span><span class="nb">{</span>enumerate<span class="nb">}</span>&nbsp;&nbsp;</div><div class="line" id="LC135">&nbsp;&nbsp;&nbsp;&nbsp;<span class="k">\item</span> Programmet söker efter uppdateringar från andra medlemmar i wikigruppen men finner inga.</div><div class="line" id="LC136">&nbsp;&nbsp;&nbsp;&nbsp;<span class="k">\item</span> Person A låter programmet stå på i bakgrunden i väntan på nya uppdateringar.&nbsp;&nbsp;</div><div class="line" id="LC137">&nbsp;&nbsp;<span class="k">\end</span><span class="nb">{</span>enumerate<span class="nb">}</span>&nbsp;&nbsp;</div><div class="line" id="LC138">&nbsp;&nbsp;<span class="k">\item</span> Huvudscenario</div><div class="line" id="LC139">&nbsp;&nbsp;<span class="k">\\</span>Den viktigaste delen i use-caset är när programmet synkroniserar artiklarna med de andra medlemmarna</div><div class="line" id="LC140">&nbsp;&nbsp;<span class="k">\item</span> Efterhandsvillkor</div><div class="line" id="LC141">&nbsp;&nbsp;<span class="k">\\</span>Detta use-case utgår från att användaren är online. Användaren kan lika gärna läsa artiklar offline men inga nya uppdateringar kan då hämtas.</div><div class="line" id="LC142">&nbsp;&nbsp;<span class="k">\item</span> Speciella krav</div><div class="line" id="LC143">&nbsp;&nbsp;<span class="k">\\</span>Användaren skall kunna vara medlem i en speciell wikigrupp.</div><div class="line" id="LC144">&nbsp;&nbsp;<span class="k">\\</span>Användaren skall ha läsrättigheter i wikigruppen.</div><div class="line" id="LC145"><span class="k">\end</span><span class="nb">{</span>itemize<span class="nb">}</span></div><div class="line" id="LC146">&nbsp;</div><div class="line" id="LC147"><span class="k">\subsection</span><span class="nb">{</span>Skriva en artikel<span class="nb">}</span></div><div class="line" id="LC148"><span class="k">\begin</span><span class="nb">{</span>itemize<span class="nb">}</span></div><div class="line" id="LC149">&nbsp;&nbsp;<span class="k">\item</span> Kortfattad beskrivning</div><div class="line" id="LC150">&nbsp;&nbsp;<span class="k">\\</span>En person skall skriva en artikel i sin wiki online.</div><div class="line" id="LC151">&nbsp;&nbsp;<span class="k">\item</span> Aktörer</div><div class="line" id="LC152">&nbsp;&nbsp;<span class="k">\\</span>En användare person A utan större datorvana.</div><div class="line" id="LC153">&nbsp;&nbsp;<span class="k">\item</span> Förhandsvillkor</div><div class="line" id="LC154">&nbsp;&nbsp;<span class="k">\\</span>Person A är online och har programmet installerat på sin dator.</div><div class="line" id="LC155">&nbsp;&nbsp;<span class="k">\item</span> Händelseförlopp</div><div class="line" id="LC156">&nbsp;&nbsp;<span class="k">\begin</span><span class="nb">{</span>enumerate<span class="nb">}</span></div><div class="line" id="LC157">&nbsp;&nbsp;&nbsp;&nbsp;<span class="k">\item</span> Person A startar programmet.</div><div class="line" id="LC158">&nbsp;&nbsp;&nbsp;&nbsp;<span class="k">\item</span> Person A väljer att skapa en ny artikel.</div><div class="line" id="LC159">&nbsp;&nbsp;&nbsp;&nbsp;<span class="k">\item</span> Ett redigeringsfönster visas och person A skriver sin artikel.</div><div class="line" id="LC160">&nbsp;&nbsp;&nbsp;&nbsp;<span class="k">\item</span> När person A är färdig väljer han att publicera sin artikel som nu kan hämtas och läsas av de andra medlemmarna i gruppen.</div><div class="line" id="LC161">&nbsp;&nbsp;<span class="k">\end</span><span class="nb">{</span>enumerate<span class="nb">}</span></div><div class="line" id="LC162">&nbsp;&nbsp;<span class="k">\item</span> Alternativt händelseförlopp</div><div class="line" id="LC163">&nbsp;&nbsp;<span class="k">\\</span>Efter steg 1:</div><div class="line" id="LC164">&nbsp;&nbsp;<span class="k">\begin</span><span class="nb">{</span>enumerate<span class="nb">}</span>&nbsp;&nbsp;</div><div class="line" id="LC165">&nbsp;&nbsp;&nbsp;&nbsp;<span class="k">\item</span> Programmet synkroniserar ändringar i artiklar med andra medlemmar som är online i wikigruppen.</div><div class="line" id="LC166">&nbsp;&nbsp;&nbsp;&nbsp;<span class="k">\item</span> Person A läser en uppdaterad artikel och bestämmer sig för att ändra på den.</div><div class="line" id="LC167">&nbsp;&nbsp;&nbsp;&nbsp;<span class="k">\item</span> Person A väljer att redigera artikeln och ett redigeringsfönster med artikeln i visas.</div><div class="line" id="LC168">&nbsp;&nbsp;&nbsp;&nbsp;<span class="k">\item</span> Person A gör de önskvärda ändringarna&nbsp;&nbsp;och publicerar den nya artikeln&nbsp;&nbsp;som nu kan hämtas och läsas av de andra medlemmarna i gruppen.</div><div class="line" id="LC169">&nbsp;&nbsp;<span class="k">\end</span><span class="nb">{</span>enumerate<span class="nb">}</span></div><div class="line" id="LC170">&nbsp;&nbsp;<span class="k">\item</span> Huvudscenario</div><div class="line" id="LC171">&nbsp;&nbsp;<span class="k">\\</span>De viktigaste delarna i detta use-case är när person A redigerar en artikel i det redigerbara fönstret samt när person A är färdig och publicerar artikeln.</div><div class="line" id="LC172">&nbsp;&nbsp;<span class="k">\item</span> Efterhandsvillkor</div><div class="line" id="LC173">&nbsp;&nbsp;<span class="k">\\</span>Detta use-case utgår från att användaren är online. Användaren kan lika gärna skriva artiklar offline men inga  andra användare kan då läsa dem innan användaren kopplar upp sig.</div><div class="line" id="LC174">&nbsp;&nbsp;<span class="k">\item</span> Speciella krav</div><div class="line" id="LC175">&nbsp;&nbsp;<span class="k">\\</span>Användaren skall kunna vara medlem i en speciell wikigrupp.</div><div class="line" id="LC176">&nbsp;&nbsp;<span class="k">\\</span>Användaren skall ha läsrättigheter i wikigruppen.</div><div class="line" id="LC177">&nbsp;&nbsp;<span class="k">\\</span>Användaren skall ha skrivrättigheter i wikigruppen.</div><div class="line" id="LC178"><span class="k">\end</span><span class="nb">{</span>itemize<span class="nb">}</span></div><div class="line" id="LC179">&nbsp;</div><div class="line" id="LC180"><span class="k">\subsection</span><span class="nb">{</span>Ta tillbaks en gammal version av en artikel<span class="nb">}</span></div><div class="line" id="LC181"><span class="k">\begin</span><span class="nb">{</span>itemize<span class="nb">}</span></div><div class="line" id="LC182">&nbsp;&nbsp;<span class="k">\item</span> Kortfattad beskrivning</div><div class="line" id="LC183">&nbsp;&nbsp;<span class="k">\\</span>En person finner att en artikel blivit felaktigt uppdaterad och vill återställa sin gamla kopia lokalt.</div><div class="line" id="LC184">&nbsp;&nbsp;<span class="k">\item</span> Aktörer</div><div class="line" id="LC185">&nbsp;&nbsp;<span class="k">\\</span>En användare av systemet, person A, utan större datorvana.</div><div class="line" id="LC186">&nbsp;&nbsp;<span class="k">\item</span> Förhandsvillkor</div><div class="line" id="LC187">&nbsp;&nbsp;<span class="k">\\</span>Person A är online och har programmet installerat på sin dator.</div><div class="line" id="LC188">&nbsp;&nbsp;<span class="k">\item</span> Händelseförlopp</div><div class="line" id="LC189">&nbsp;&nbsp;<span class="k">\begin</span><span class="nb">{</span>enumerate<span class="nb">}</span></div><div class="line" id="LC190">&nbsp;&nbsp;&nbsp;&nbsp;<span class="k">\item</span> Person A startar programmet.</div><div class="line" id="LC191">&nbsp;&nbsp;&nbsp;&nbsp;<span class="k">\item</span> Programmet synkroniserar ändringar i artiklar med andra medlemmar som är online i wikigruppen.</div><div class="line" id="LC192">&nbsp;&nbsp;&nbsp;&nbsp;<span class="k">\item</span> Person A läser en av de uppdaterade artiklarna och ogillar de nya ändringarna.</div><div class="line" id="LC193">&nbsp;&nbsp;&nbsp;&nbsp;<span class="k">\item</span> Person A väljer att återskapa en äldre version av artikeln och ett fönster med möjliga äldre versioner visas.</div><div class="line" id="LC194">&nbsp;&nbsp;&nbsp;&nbsp;<span class="k">\item</span> Person A väljer den version av artikeln han vill ha och artikeln återskapas lokalt.</div><div class="line" id="LC195">&nbsp;&nbsp;<span class="k">\end</span><span class="nb">{</span>enumerate<span class="nb">}</span>&nbsp;&nbsp;</div><div class="line" id="LC196">&nbsp;&nbsp;<span class="k">\item</span> Huvudscenario</div><div class="line" id="LC197">&nbsp;&nbsp;<span class="k">\\</span>Den viktigaste delen i detta use-case är att person A kan välja vilken version han vill gå tillbaks till.</div><div class="line" id="LC198">&nbsp;&nbsp;<span class="k">\item</span> Efterhandsvillkor</div><div class="line" id="LC199">&nbsp;&nbsp;<span class="k">\\</span>Detta use-case utgår från att användaren är online och får en uppdatering på en artikel. Dock kan användaren även återskapa äldre versioner av artiklar offline.</div><div class="line" id="LC200">&nbsp;&nbsp;<span class="k">\item</span> Speciella krav</div><div class="line" id="LC201">&nbsp;&nbsp;<span class="k">\\</span>Användaren skall kunna vara medlem i en speciell wikigrupp.</div><div class="line" id="LC202">&nbsp;&nbsp;<span class="k">\\</span>Användaren skall ha läsrättigheter i wikigruppen(behövs endast för att ta emot nya uppdateringar).</div><div class="line" id="LC203">&nbsp;&nbsp;<span class="k">\\</span>Användaren har ändringar av den aktuella artikeln sparade lokalt.</div><div class="line" id="LC204"><span class="k">\end</span><span class="nb">{</span>itemize<span class="nb">}</span></div><div class="line" id="LC205">&nbsp;</div><div class="line" id="LC206">&nbsp;</div><div class="line" id="LC207"><span class="k">\subsection</span><span class="nb">{</span>Sammanslagning<span class="nb">}</span></div><div class="line" id="LC208"><span class="k">\begin</span><span class="nb">{</span>itemize<span class="nb">}</span></div><div class="line" id="LC209">&nbsp;&nbsp;<span class="k">\item</span> Kortfattad beskrivning</div><div class="line" id="LC210">&nbsp;&nbsp;<span class="k">\\</span>En användare ska sammanfoga sina dokument eller annat arbete med andra personers arbeten.</div><div class="line" id="LC211">&nbsp;&nbsp;<span class="k">\item</span> Aktörer</div><div class="line" id="LC212">&nbsp;&nbsp;<span class="k">\\</span>Tre användare A, B och C som inte har någon större datorvana.</div><div class="line" id="LC213">&nbsp;&nbsp;<span class="k">\\</span>Administratör för wiki-gruppen med lite mer datorvana.&nbsp;&nbsp;</div><div class="line" id="LC214">&nbsp;&nbsp;<span class="k">\item</span> Förhandsvillkor</div><div class="line" id="LC215">&nbsp;&nbsp;<span class="k">\\</span>Ett distribuerat wiki-system är konfigurerat, satt igång och används.</div><div class="line" id="LC216">&nbsp;&nbsp;<span class="k">\item</span> Händelseförlopp</div><div class="line" id="LC217">&nbsp;&nbsp;<span class="k">\begin</span><span class="nb">{</span>enumerate<span class="nb">}</span></div><div class="line" id="LC218">&nbsp;&nbsp;&nbsp;&nbsp;<span class="k">\item</span> Person A har arbetat med dokument på sin dator och vill sätta ihop detta med person B:s och person C:s dokument.</div><div class="line" id="LC219">&nbsp;&nbsp;&nbsp;&nbsp;<span class="k">\item</span> Person A vet inte vad person B och person C har arbetat med men öppnar ändå sin wiki.</div><div class="line" id="LC220">&nbsp;&nbsp;&nbsp;&nbsp;<span class="k">\item</span> Han går in på sidan för person B samt C:s dokument och därmed synkroniseras hans dokument automatiskt.</div><div class="line" id="LC221">&nbsp;&nbsp;&nbsp;&nbsp;<span class="k">\item</span> Systemet försöker att sammanfoga dokumenten.</div><div class="line" id="LC222">&nbsp;&nbsp;&nbsp;&nbsp;<span class="k">\item</span> Eftersom olika stycken är ändrade i dokumenten sker en konflikt.</div><div class="line" id="LC223">&nbsp;&nbsp;&nbsp;&nbsp;<span class="k">\item</span> Användaren får en förfrågan om han vill lösa konflikten.</div><div class="line" id="LC224">&nbsp;&nbsp;&nbsp;&nbsp;<span class="k">\item</span> Användaren får upp dokumenten bredvid varandra och kan sedan ändra det han behagar.</div><div class="line" id="LC225">&nbsp;&nbsp;&nbsp;&nbsp;<span class="k">\item</span> Till sist sparar användaren dokumentet och känner sig nöjd.</div><div class="line" id="LC226">&nbsp;&nbsp;<span class="k">\end</span><span class="nb">{</span>enumerate<span class="nb">}</span></div><div class="line" id="LC227">&nbsp;&nbsp;<span class="k">\item</span> Alternativt händelseförlopp</div><div class="line" id="LC228">&nbsp;&nbsp;<span class="k">\\</span>Efter steg 5:</div><div class="line" id="LC229">&nbsp;&nbsp;<span class="k">\begin</span><span class="nb">{</span>enumerate<span class="nb">}</span>&nbsp;&nbsp;</div><div class="line" id="LC230">&nbsp;&nbsp;&nbsp;&nbsp;<span class="k">\item</span> Användaren inte har rättigheter att andra på filerna.</div><div class="line" id="LC231">&nbsp;&nbsp;&nbsp;&nbsp;<span class="k">\item</span> Användaren kontaktar administratören om hjälp.</div><div class="line" id="LC232">&nbsp;&nbsp;<span class="k">\end</span><span class="nb">{</span>enumerate<span class="nb">}</span>&nbsp;&nbsp;</div><div class="line" id="LC233">&nbsp;&nbsp;<span class="k">\item</span> Alternativt händelseförlopp</div><div class="line" id="LC234">&nbsp;&nbsp;<span class="k">\\</span>Efter steg 5: </div><div class="line" id="LC235">&nbsp;&nbsp;<span class="k">\begin</span><span class="nb">{</span>enumerate<span class="nb">}</span>&nbsp;&nbsp;</div><div class="line" id="LC236">&nbsp;&nbsp;&nbsp;&nbsp;<span class="k">\item</span> Användaren inte känner sig tillräckligt säker att ändra på filerna.</div><div class="line" id="LC237">&nbsp;&nbsp;&nbsp;&nbsp;<span class="k">\item</span> Användaren kontaktar då den/de personer som har skrivit filerna och ber om hjälp.</div><div class="line" id="LC238">&nbsp;&nbsp;<span class="k">\end</span><span class="nb">{</span>enumerate<span class="nb">}</span></div><div class="line" id="LC239">&nbsp;&nbsp;<span class="k">\item</span> Huvudscenario</div><div class="line" id="LC240">&nbsp;&nbsp;</div><div class="line" id="LC241">&nbsp;&nbsp;Det viktiga i detta use-case är att visa att man kan sammanfoga olika filer med varandra trots att rättigheter samt kompetens kan ställa till det.</div><div class="line" id="LC242">&nbsp;&nbsp;<span class="k">\item</span> Efterhandsvillkor&nbsp;&nbsp;</div><div class="line" id="LC243">&nbsp;&nbsp;&nbsp;&nbsp;<span class="k">\\</span>Success condition </div><div class="line" id="LC244">&nbsp;&nbsp;&nbsp;&nbsp;<span class="k">\\</span>Anvandaren lyckades med att sammanfoga (merge) två eller fler filer.</div><div class="line" id="LC245">&nbsp;&nbsp;&nbsp;&nbsp;<span class="k">\\</span>Failure condition</div><div class="line" id="LC246">&nbsp;&nbsp;&nbsp;&nbsp;<span class="k">\\</span>Anvandaren lyckades ej med en sammanfogning.</div><div class="line" id="LC247">&nbsp;&nbsp;&nbsp;&nbsp;</div><div class="line" id="LC248">&nbsp;&nbsp;<span class="k">\item</span> Speciella krav</div><div class="line" id="LC249">&nbsp;&nbsp;<span class="k">\\</span>Användaren ska vara medlem i en speciell wiki-grupp.</div><div class="line" id="LC250"><span class="k">\end</span><span class="nb">{</span>itemize<span class="nb">}</span></div><div class="line" id="LC251">&nbsp;</div><div class="line" id="LC252"><span class="k">\subsection</span><span class="nb">{</span>Rättstavning<span class="nb">}</span></div><div class="line" id="LC253"><span class="k">\begin</span><span class="nb">{</span>itemize<span class="nb">}</span></div><div class="line" id="LC254">&nbsp;&nbsp;<span class="k">\item</span> Kortfattad beskrivning</div><div class="line" id="LC255">&nbsp;&nbsp;<span class="k">\\</span>En användare orkar eller vill inte sitta och korrekturläsa hela dokumentet för att hitta stavfel utan vill använda en automatisk rättstavare.</div><div class="line" id="LC256">&nbsp;&nbsp;<span class="k">\item</span> Aktörer</div><div class="line" id="LC257">&nbsp;&nbsp;<span class="k">\\</span>En användare, person A, utan direkt datorvana.</div><div class="line" id="LC258">&nbsp;&nbsp;<span class="k">\item</span> Förhandvillkor</div><div class="line" id="LC259">&nbsp;&nbsp;<span class="k">\\</span>Person A har programmet installerat på datorn</div><div class="line" id="LC260">&nbsp;&nbsp;<span class="k">\item</span> Händelseförlopp</div><div class="line" id="LC261">&nbsp;&nbsp;<span class="k">\begin</span><span class="nb">{</span>enumerate<span class="nb">}</span></div><div class="line" id="LC262">&nbsp;&nbsp;&nbsp;&nbsp;<span class="k">\item</span> En användare vill eller har inte tid att läsa ett dokument som han vill göra tillgängligt för andra användare.</div><div class="line" id="LC263">&nbsp;&nbsp;&nbsp;&nbsp;<span class="k">\item</span> Användaren går in på dokumentet.</div><div class="line" id="LC264">&nbsp;&nbsp;&nbsp;&nbsp;<span class="k">\item</span> Användaren väljer sedan att rättstava dokumentet.</div><div class="line" id="LC265">&nbsp;&nbsp;&nbsp;&nbsp;<span class="k">\item</span> För varje fel som rättstavningskontrollen hittar får användaren en fråga om han vill ändra det befintliga ordet till det som rättstavningskontrollen föreslår.</div><div class="line" id="LC266">&nbsp;&nbsp;&nbsp;&nbsp;<span class="k">\item</span> När användaren är klar sparar han dokumentet.</div><div class="line" id="LC267">&nbsp;&nbsp;<span class="k">\end</span><span class="nb">{</span>enumerate<span class="nb">}</span></div><div class="line" id="LC268">&nbsp;&nbsp;<span class="k">\item</span> Huvudscenario</div><div class="line" id="LC269">&nbsp;&nbsp;<span class="k">\\</span>Det viktiga delen i det här use-caset är att en användare skall kunna använda automatisk rättstavning för att snabba upp dokumenthanteringen.</div><div class="line" id="LC270">&nbsp;&nbsp;<span class="k">\item</span> Post-condition</div><div class="line" id="LC271">&nbsp;&nbsp;<span class="k">\\</span> Antingen hittar rättstavningskontrollen inga fel och godkänner dokumentet direkt eller så hittar den något som den klagar på och användaren får välja om det ska rättas till.</div><div class="line" id="LC272">&nbsp;&nbsp;<span class="k">\item</span> Speciella krav</div><div class="line" id="LC273">&nbsp;&nbsp;<span class="k">\\</span>Användaren skall läs och skriv rättigheter för att kunna göra en rättstavning.</div><div class="line" id="LC274"><span class="k">\end</span><span class="nb">{</span>itemize<span class="nb">}</span></div><div class="line" id="LC275"><span class="k">\end</span><span class="nb">{</span>document<span class="nb">}</span></div><div class="line" id="LC276">&nbsp;</div></pre></div>
            
          </td>
        </tr>
      </table>
    
  </div>


      </div>
    </div>
    
  


  </div>

      
      
      <div class="push"></div>
    </div>
    
    <div id="footer">
      <div class="site">
        <div class="info">
          <div class="links">
            <a href="http://github.com/blog/148-github-shirts-now-available">Shirts</a> |
            <a href="http://github.com/blog">Blog</a> |
            <a href="http://support.github.com/">Support</a> |
            <a href="http://github.com/training">Training</a> |
            <a href="http://github.com/contact">Contact</a> |
            <a href="http://groups.google.com/group/github/">Google Group</a> |
            <a href="http://github.com/guides/the-github-api">API</a> |
            <a href="http://twitter.com/github">Status</a>
          </div>
          <div class="company">
            <span id="_rrt" title="0.39286s from xc88-s00039">GitHub</span>
            is <a href="http://logicalawesome.com/">Logical Awesome</a> &copy;2009 | <a href="/site/terms">Terms of Service</a> | <a href="/site/privacy">Privacy Policy</a>
          </div>
        </div>
        <div class="sponsor">
          <a href="http://engineyard.com"><img src="/images/modules/footer/engine_yard_logo.png" alt="Engine Yard" /></a>
          <div>
            Hosting provided by our<br /> partners at Engine Yard
          </div>
        </div>
      </div>
    </div>
    
    <div id="coming_soon" style="display:none;">
      This feature is coming soon.  Sit tight!
    </div>

    
        <script type="text/javascript">
    var gaJsHost = (("https:" == document.location.protocol) ? "https://ssl." : "http://www.");
    document.write(unescape("%3Cscript src='" + gaJsHost + "google-analytics.com/ga.js' type='text/javascript'%3E%3C/script%3E"));
    </script>
    <script type="text/javascript">
    var pageTracker = _gat._getTracker("UA-3769691-2");
    pageTracker._initData();
    pageTracker._trackPageview();
    </script>

    
  </body>
</html>

