% Standardinkluderingsfil
%
%  untitled
%
%  Created by David Granqvist on 2008-09-08.
%  Modified by Martin Erola
%

% Set document format/class
\documentclass[a4paper,twoside]{article}

%%%%%%%%%%%%%%%%%%%
% Include packages
%
\usepackage[utf8]{inputenc}   % Use utf-8 encoding for foreign characters
\usepackage[swedish]{babel}   % Support for swedish letters
\usepackage{fullpage}         % Setup for fullpage use
\usepackage{fancyhdr}         % Running Headers and footers
\usepackage{boxedminipage}    % Surround parts of graphics with box
\usepackage{listings}         % Package for including code in the document
\usepackage{ifpdf}            % Recommended way for checking for PDFLaTeX:
\usepackage{tabularx}         % Tabeller med automatisk stretch
% \usepackage[nofancy]{svninfo} % Extract Subversion info about the file
% \usepackage{color}          % Color
% \usepackage{lastpage}       % Total page count

% Graphics
\ifpdf
\usepackage[pdftex]{graphicx}
\else
\usepackage{graphicx}
\fi

%%%%%%%%%%%%%%%%%%%%%%%%%%%%%%%%%%%%%%%%%%%%%%%%%%%%%%%%%%
% Uncomment some of the following if you use the features
%

% Multipart figures
%\usepackage{subfigure}

% More symbols
%\usepackage{amsmath}
%\usepackage{amssymb}
%\usepackage{latexsym}

% If you want to generate a toc for each chapter (use with book)
% \usepackage{minitoc}

%%%%%%%%%%%%%%%%%%%%
% Document settings
%

% Header
\pagestyle{fancy}
% Sätter en marginal mellan header och (ovanstående?) text %
\setlength\headsep{10pt}
% Sätter höjden på headern
\setlength{\headheight}{32pt}

% Sätter styckesinställningar
\setlength\parindent{0pt}
\setlength\parskip{10pt}



\ifpdf
  \DeclareGraphicsExtensions{.pdf, .jpg, .tif, .png}
  \pdfinfo{            
    /Title  (Use-case)
    /Author (PUM-grupp 1)
  }
\else
  \DeclareGraphicsExtensions{.eps, .jpg}
\fi

\title{Use-case}
\author{PUM-grupp 1}
\date{\today}

\begin{document}

\maketitle\thispagestyle{empty}

\newpage

\section{Inledning}
\textsc{\LARGE Use-case}

\subsection{Exempel}
\begin{itemize}
	\item Brief description
	
	Snabb beskrivning av vad som hander under caset.
	\item Actors
	Vilka personer ar med i caset?
	
	\item Preconditions
	
	Antaganden for att caset ska kunna "`genomforas"'
	\item Basic flow of events
	
	Ga steg for steg igenom vad som hander i caset.
	\item Alternate flow of events
	
	Andra satt caset kan utspela sig pa, t.ex. att nagot gar fel.
	\item Key scenarios
	
	Det viktigaste scenariot i caset(?).
	\item Post-conditions
	
	Mojliga utgangar av caset.
	\item Special Requirements
	
	Speciella krav for caset.
\end{itemize}

\subsection{Bläddra}
\begin{itemize}
	\item Brief description
	\\En person vill finna en speciell artikel.
	\item Actors
	\\En användare person A utan större datorvana.
	\item Preconditions
	\\Person A har programmet installerat på sin dator.
	\item Basic flow of events
	\\Person A startar programmet.
	\\Programmet synkroniserar ändringar i artiklar med andra medlemmar som är online i wikigruppen.
	\\Person A skriver in sökord som matchar artikeln han vill söka efter i en sökruta.
	\\En lista med möjliga artiklar dyker upp och person A väljer en av dem.
	\\Artikeln visas i ett fönster och person A kan läsa den.
	\item Alternate flow of events
	\\Person A startar programmet.
	\\Programmet synkroniserar ändringar i artiklar med andra medlemmar som är online i wikigruppen.
	\\Person A navigerar till sin önskade artikel genom länkar från andra artiklar.	
	\\Artikeln visas i ett fönster och person A kan läsa den.
	\item Key scenarios
	\\Den viktigaste delen i detta use-case är att person A kan välja mellan två sätt att finna sin artikel. Antingen hitta den genom länkar från andra artiklar eller genom att söka efter den.
	\item Post-conditions
	\\Detta use-case utgår från att användaren är online och får uppdateringar på artiklar men användaren kan bläddra efter artiklar offline.
	\item Special Requirements
	\\Användaren skall kunna vara medlem i en speciell wikigrupp.
	\\Användaren skall ha läsrättigheter i wikigruppen(behövs endast för att ta emot nya uppdateringar).
\end{itemize}

\subsection{Allmän distribuering}
\begin{itemize}
	\item Brief description
	\\En person skriver en artikel, en annan person läser artikeln och en annan redigerar den.
	\item Actors
	\\En användare person A utan större datorvana.
	\\En användare person B utan större datorvana.
	\\En användare person C utan större datorvana.
	\item Preconditions
	\\Person A, B och C har programmet installerat på sin dator.
	\item Basic flow of events
	\\Person A startar programmet.
	\\Programmet synkroniserar ändringar i artiklar med andra medlemmar som är online i wikigruppen.
	\\Person A skriver en ny artikel.
	\\Person B kommer online och får uppdateringar innehållande artikeln från person A.
	\\Person A går offline.
	\\Person C kommer online och får uppdateringar innehållande artikeln från person B.
	\\Person B går offline.
	\\Person C väljer att redigera artikeln.
	\\Person A kommer online och får uppdateringar innehållande ändringarna i artikeln från person C.
	\\Person C går offline.
	\\Person B kommer online och får uppdateringar från person A innehållande ändringarna i artikeln.
	\item Key scenarios
	\\Den viktigaste delen i detta use-case är att visa på vilket sätt ändringar sprids i ett distribuerat system.
	\item Post-conditions
	\\Detta use-case utgår från att flera användare har programmet och är medlemmar i samma wikigrupp.
	\item Special Requirements
	\\Användarna skall kunna vara medlem i en speciell wikigrupp.
	\\Användare A skall ha skriv och läs-rättigheter i wikigruppen.
	\\Användare B skall ha läsrättigheter i wikigruppen.
	\\Användare C skall ha skriv och läs-rättigheter i wikigruppen.
\end{itemize}


\subsection{Läsa en artikel}
\begin{itemize}
	\item Brief description
	\\En person skall läsa en artikel i sin wiki online.
	\item Actors
	\\En användare person A utan större datorvana.
	\item Preconditions
	\\Person A är online och har programmet installerat på sin dator.
	\item Basic flow of events
	\\Person A startar programmet.
	\\Programmet synkroniserar ändringar i artiklar med andra medlemmar som är online i wikigruppen.
	\\Person A väljer en uppdaterad artikel han/hon vill läsa.
	\\Artikeln visas och person A läser den.
	\item Alternate flow of events
	\\Person A startar programmet.
	\\Programmet söker efter uppdateringar från andra medlemmar i wikigruppen men finner inga.
	\\Person A låter programmet stå på i bakgrunden i väntan på nya uppdateringar.	
	\item Key scenarios
	\\Den viktigaste delen i use-caset är när programmet synkroniserar artiklarna med de andra medlemmarna
	\item Post-conditions
	\\Detta use-case utgår från att användaren är online. Användaren kan lika gärna läsa artiklar offline men inga nya uppdateringar kan då hämtas.
	\item Special Requirements
	\\Användaren skall kunna vara medlem i en speciell wikigrupp.
	\\Användaren skall ha läsrättigheter i wikigruppen.
\end{itemize}

\subsection{Skriva en artikel}
\begin{itemize}
	\item Brief description
	\\En person skall skriva en artikel i sin wiki online.
	\item Actors
	\\En användare person A utan större datorvana.
	\item Preconditions
	\\Person A är online och har programmet installerat på sin dator.
	\item Basic flow of events
	\\Person A startar programmet.
	\\Person A väljer att skapa en ny artikel.
	\\Ett redigeringsfönster visas och person A skriver sin artikel.
	\\När person A är färdig väljer han att publicera sin artikel som nu kan hämtas och läsas av de andra medlemmarna i gruppen.
	\item Alternate flow of events
	\\Person A startar programmet.
	\\Programmet synkroniserar ändringar i artiklar med andra medlemmar som är online i wikigruppen.
	\\Person A läser en uppdaterad artikel och bestämmer sig för att ändra på den.
	\\Person A väljer att redigera artikeln och ett redigeringsfönster med artikeln i visas.
	\\Person A gör de önskvärda ändringarna	och publicerar den nya artikeln	som nu kan hämtas och läsas av de andra medlemmarna i gruppen.
	\item Key scenarios
	\\De viktigaste delarna i detta use-case är när person A redigerar en artikel i det redigerbara fönstret samt när person A är färdig och publicerar artikeln.
	\item Post-conditions
	\\Detta use-case utgår från att användaren är online. Användaren kan lika gärna skriva artiklar offline men inga  andra användare kan då läsa dem innan användaren kopplar upp sig.
	\item Special Requirements
	\\Användaren skall kunna vara medlem i en speciell wikigrupp.
	\\Användaren skall ha läsrättigheter i wikigruppen.
	\\Användaren skall ha skrivrättigheter i wikigruppen.
\end{itemize}

\subsection{Ta tillbaks en gammal version av en artikel}
\begin{itemize}
	\item Brief description
	\\En person finner att en artikel blivit felaktigt uppdaterad och vill återställa sin gamla kopia lokalt.
	\item Actors
	\\En användare person A utan större datorvana.
	\item Preconditions
	\\Person A är online och har programmet installerat på sin dator.
	\item Basic flow of events
	\\Person A startar programmet.
	\\Programmet synkroniserar ändringar i artiklar med andra medlemmar som är online i wikigruppen.
	\\Person A läser en av de uppdaterade artiklarna och ogillar de nya ändringarna.
	\\Person A väljer att återskapa en äldre version av artikeln och ett fönster med möjliga äldre versioner visas.
	\\Person A väljer den version av artikeln han vill ha och artikeln återskapas lokalt.
	\item Key scenarios
	\\Den viktigaste delen i detta use-case är att person A kan välja vilken version han vill gå tillbaks till.
	\item Post-conditions
	\\Detta use-case utgår från att användaren är online och får en uppdatering på en artikel men användaren kan återskapa äldre versioner av artiklar även offline.
	\item Special Requirements
	\\Användaren skall kunna vara medlem i en speciell wikigrupp.
	\\Användaren skall ha läsrättigheter i wikigruppen(behövs endast för att ta emot nya uppdateringar).
	\\Användaren har ändringar av den aktuella artikeln sparade lokalt.
\end{itemize}


\subsection{Sammanslagning}
\begin{itemize}
	\item Brief description
	
	Person ska merga sina dokument eller annat arbeta med andra personers arbeten.
	\item Actors
	
	Anvandare
	Administrator	
	\item Preconditions
	
	Ett distribuerat wiki-system ar konfigurerat, satt igang och anvands.
	\item Basic flow of events
	
	\begin{enumerate}
		\item Person A har arbetat med dokument pa sin dator och vill satta ihop detta med person B:s och person C:s dokument.
		\item Person A har vet inte vad person B och person C har arbetat med men oppnar anda sin wiki.
		\item Han gar in pa sidan for person B samt C:s dokument och valjer att synkronisera.
		\item Systemet forsoker att sammanfoga dokumenten.
		\item Eftersom olika stycken ar andrade i dokumenten sker en konflikt.
		\item Anvandaren far en forfragan om han vill losa konflikten.
		\item Anvandaren far upp dokumenten brevid varnadra och kan sedan andra det han behagar.
		\item Till sist spara anvandaren dokumentet och kanner sig nojd.
	\end{enumerate}
	\item Alternate flow of events
	
	Efter steg 5. Om anvandaren inte har rattigheter att andra pa filerna.
	Anvandaren kontaktar administratoren om hjalp.
	\item Alternate flow of events
	Efter steg 5. Om anvandaren inte kanner sig tillrackligt saker att andra pa filerna.
	Anvandaren kontaktar da den/de personer som har skrivit filerna och ber om hjalp.
	\item Key scenarios
	
	Att sammanfoga olika filer med varandra trots att rattigheter samt kompetens kan stalla till det.
	\item Post-conditions	
	
		Success condition 
		\\
		Anvandaren lyckades med att sammanfoga (merge) tva eller fler filer.
		\\
		Failure condition
		\\
		Anvandaren lyckades ej med en sammanfogning.
		
	\item Special Requirements
	
	???
\end{itemize}

\end{document}
