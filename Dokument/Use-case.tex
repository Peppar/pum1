% Standardinkluderingsfil
%
%  untitled
%
%  Created by David Granqvist on 2008-09-08.
%  Modified by Martin Erola
%

% Set document format/class
\documentclass[a4paper,twoside]{article}

%%%%%%%%%%%%%%%%%%%
% Include packages
%
\usepackage[utf8]{inputenc}   % Use utf-8 encoding for foreign characters
\usepackage[swedish]{babel}   % Support for swedish letters
\usepackage{fullpage}         % Setup for fullpage use
\usepackage{fancyhdr}         % Running Headers and footers
\usepackage{boxedminipage}    % Surround parts of graphics with box
\usepackage{listings}         % Package for including code in the document
\usepackage{ifpdf}            % Recommended way for checking for PDFLaTeX:
\usepackage{tabularx}         % Tabeller med automatisk stretch
% \usepackage[nofancy]{svninfo} % Extract Subversion info about the file
% \usepackage{color}          % Color
% \usepackage{lastpage}       % Total page count

% Graphics
\ifpdf
\usepackage[pdftex]{graphicx}
\else
\usepackage{graphicx}
\fi

%%%%%%%%%%%%%%%%%%%%%%%%%%%%%%%%%%%%%%%%%%%%%%%%%%%%%%%%%%
% Uncomment some of the following if you use the features
%

% Multipart figures
%\usepackage{subfigure}

% More symbols
%\usepackage{amsmath}
%\usepackage{amssymb}
%\usepackage{latexsym}

% If you want to generate a toc for each chapter (use with book)
% \usepackage{minitoc}

%%%%%%%%%%%%%%%%%%%%
% Document settings
%

% Header
\pagestyle{fancy}
% Sätter en marginal mellan header och (ovanstående?) text %
\setlength\headsep{10pt}
% Sätter höjden på headern
\setlength{\headheight}{32pt}

% Sätter styckesinställningar
\setlength\parindent{0pt}
\setlength\parskip{10pt}



\ifpdf
  \DeclareGraphicsExtensions{.pdf, .jpg, .tif, .png}
  \pdfinfo{            
    /Title  (Use-case)
    /Author (PUM-grupp 1)
  }
\else
  \DeclareGraphicsExtensions{.eps, .jpg}
\fi

\title{Use-case}
\author{PUM-grupp 1}
\date{\today}

\begin{document}

\maketitle\thispagestyle{empty}

\newpage
% Dokumenthistorik och innehållsförteckning %

{\centering \Large{Dokumenthistorik\\}}

\vspace{10pt}
\begin{tabularx}{\textwidth}{ |l|l|X|l|l| }
  \hline
    \textbf{version} & \textbf{datum} & \textbf{utförda ändringar} & \textbf{utförda av} & \textbf{granskad} \\
	\hline 
  1.0 & 2009-02-12 &  Första versionen klar för inlämning  & Alla & Alla   \\
  \hline
\end{tabularx}

\newpage
\setcounter{tocdepth}{2}
\tableofcontents
\newpage
% Dokumenthistorik och innehållsförteckning %
\section{Inledning}

%\textsc{\LARGE Användarfall}

\subsection{Exempel}
\begin{itemize}
	\item Kortfattad beskrivning - Kort beskrivning av användarfallet.
	\item Aktörer - Beskrivning av de aktörer som medverkar.
	\item Förhandsvillkor - Antaganden för att fallet ska kunna genomföras.
	\item Händelseförlopp - Stegvis genomgång av händelseförloppet.
	\item Alternativt händelseförlopp - Andra sätt fallet kan utspela sig på, till exempel att något går fel.
	\item Huvudscenario - Det viktigaste scenariot i fallet.
	\item Efterhandsvillkor - Möjliga utgångar av fallet.
	\item Speciella krav - Speciella krav för fallet.
\end{itemize}

\section{Användarfall} 

För att lättare förstå vad användarna ska kunna åstadkomma med vårt system har gruppen satt upp ett antal användarfall för att försöka täcka de olika funktioner som kommer att finnas.

\subsection{Installera}

\begin{itemize}
	\item Kortfattad beskrivning
	\\En person vill installera dwiki på sin dator.
	\item Aktörer
	\\En användare, person A, utan mycket datorvana.
	\item Förhandsvillkor
	\\Person A har kör linux på sin dator.
	\item Händelseförlopp
	\begin{enumerate}
		\item Person A öppnar terminalen.
		\item Person A skriver inte kommandot \textit{sudo apt-get install dwiki}
		\item Pakethanteraren installerar dwiki på datorn.
	\end{enumerate}
	\item Alternativt händelseförlopp
	\begin{enumerate}
		\item Person A öppnar terminalen.
		\item Person A skriver inte kommandot \textit{sudo apt-get install dwiki}
		\item Pakethanterar finner att alla beroenden inte är uppfyllda (exempelsevis att git inte är installerat på datorn).
		\item Pakethanteraren installerar alla program som saknas för att kunna köra dwiki.
		\item Pakethanteraren installerar dwiki på datorn.
	\end{enumerate}	
	\item Huvudscenario
	\\Den viktigaste i detta användarfall är att dwiki finns registrerat på en paketserver och går att installera direkt genom pakethanteraren i linux.
	\item Efterhandsvillkor
	\begin{itemize}
	\item Lyckades installera dwiki.
	\\Användaren fick dwiki installerat med hjälp av pakethanteraren
	\item Lyckades inte installera dwiki.
	\\Användaren fick inte dwiki insallerat. Detta kan bero på att användaren inte är uppkopplad mot en paketserver som tillhandahåller dwiki eller att användaren inte är ansluten till internet.
	\end{itemize}
	\item Speciella krav
	\\Användaren använder en linuxdator och är uppkopplad som en paketserver som tillhandahåller dwiki.
\end{itemize}

\subsection{Bläddra}

\begin{itemize}
	\item Kortfattad beskrivning
	\\En person vill finna en speciell artikel.
	\item Aktörer
	\\En användare, person A, utan större datorvana.
	\item Förhandsvillkor
	\\Person A har dwiki installerat på sin dator.
	\item Händelseförlopp
	\begin{enumerate}
		\item Person A startar dwiki.
		\item Person A skriver in sökord som matchar artikeln han vill söka efter i en sökruta.
		\item En lista med möjliga artiklar dyker upp och person A väljer en av dem.
		\item Artikeln visas i ett fönster och person A kan läsa den.
	\end{enumerate}
	\item Alternativt händelseförlopp
	\begin{enumerate}
		\item Person A startar dwiki.
		\item Person A navigerar till sin önskade artikel genom länkar från andra artiklar.	
		\item Artikeln visas i ett fönster och person A kan läsa den.
	\end{enumerate}	
	\item Huvudscenario
	\\Den viktigaste delen i detta användarfall är att person A kan välja mellan två sätt att finna sin artikel. Antingen med hjälp av länkar från andra artiklar eller genom en sökfunktion.
	\item Efterhandsvillkor
	\begin{itemize}
	\item Fann önskad artikel
	\\Användaren fick tag på den artikel som han/hon önskade
	\item Fann inte önskad artikel
	\\Användaren fick inte tag på den artikel han/hon önskade, antingen på grund av bristande sökord, felaktiga länkningar eller så kanske inte användarens wiki fått hem alla artiklar från övriga användare i gruppen.
	\end{itemize}
	\item Speciella krav
	\\En startsida på användarens wiki med länkar till andra wiki-sidor krävs för att det alternativa händelseförloppet ska kunna genomföras.
\end{itemize}

\subsection{Läsa en artikel}
\begin{itemize}
	\item Kortfattad beskrivning
	\\En person skall läsa en artikel i sin wiki online.
	\item Aktörer
	\\En användare, person A, utan större datorvana.
	\item Förhandsvillkor
	\\Person A är online och har dwiki installerat på sin dator.
	\item Händelseförlopp
	\begin{enumerate}
		\item Person A startar dwiki.
		\item dwiki synkroniserar ändringar i artiklar med andra medlemmar som är online i wikigruppen.
		\item Person A väljer en uppdaterad artikel han/hon vill läsa.
		\item Artikeln visas och person A läser den.
	\end{enumerate}
	\item Alternativt händelseförlopp
	\\Efter steg 1:
	\begin{enumerate}	
		\item dwiki söker efter uppdateringar från andra medlemmar i wikigruppen men finner inga.
		\item Person A låter dwiki stå på i bakgrunden i väntan på nya uppdateringar.	
	\end{enumerate}	
	\item Huvudscenario
	\\Den viktigaste delen i användarfallet är när dwiki synkroniserar artiklarna med de andra medlemmarna
	\item Efterhandsvillkor
	\begin{itemize}
		\item Användaren lyckades hitta uppdateringar.
		\\Person A hitta nya uppdateringar från andra användare och synkroniserade automatiskt sina dokument med denna användares.
		\item Användaren lyckades inte hitta nya uppdateringar.
		\\Person A lyckades inte hitta nya uppdateringar från några andra användare.
	\end{itemize}	
	\item Speciella krav
	\\Användaren skall kunna vara medlem i en speciell wikigrupp.
	\\Användaren skall ha läsrättigheter i wikigruppen.
\end{itemize}

\subsection{Skriva en artikel}
\begin{itemize}
	\item Kortfattad beskrivning
	\\En person skall skriva en artikel i sin wiki online.
	\item Aktörer
	\\En användare, person A, utan större datorvana.
	\item Förhandsvillkor
	\\Person A är online och har dwiki installerat på sin dator.
	\item Händelseförlopp
	\begin{enumerate}
		\item Person A startar dwiki.
		\item Person A väljer att skapa en ny artikel.
		\item Ett redigeringsfönster visas och person A skriver sin artikel.
		\item När person A är färdig väljer han att publicera sin artikel som nu kan hämtas och läsas av de andra medlemmarna i gruppen.
	\end{enumerate}
	\item Alternativt händelseförlopp
	\\Efter steg 1:
	\begin{enumerate}	
		\item dwiki synkroniserar ändringar i artiklar med andra medlemmar som är online i wikigruppen.
		\item Person A läser en uppdaterad artikel och bestämmer sig för att ändra på den.
		\item Person A väljer att redigera artikeln och ett redigeringsfönster med artikeln i visas.
		\item Person A gör de önskvärda ändringarna	och publicerar den nya artikeln	som nu kan hämtas och läsas av de andra medlemmarna i gruppen.
	\end{enumerate}
	\item Huvudscenario
	\\De viktigaste delarna i detta användarfall är när person A redigerar en artikel i det redigerbara fönstret samt när person A är färdig och publicerar artikeln.
	\item Efterhandsvillkor
	\begin{itemize}
		\item Användaren sparar en artikel
		\\Person A skriver klart en artikel och anser att den är färdig och sparar denna på wiki:n så att andra användaren kan komma åt den.
		\item Användaren avbryter skapandet av en artikel
		\\Person A börjar skriva en ny artikel men tycker inte att den är bra så han avbryter skapandet och alla ändringar han har gjort försvinner.
	\end{itemize}	
	\item Speciella krav
	\\Användaren skall kunna vara medlem i en speciell wikigrupp.
	\\Användaren skall ha läsrättigheter i wikigruppen.
	\\Användaren skall ha skrivrättigheter i wikigruppen.
\end{itemize}

\subsection{Ta tillbaks en gammal version av en artikel}
\begin{itemize}
	\item Kortfattad beskrivning
	\\En person finner att en artikel blivit felaktigt uppdaterad och vill återställa sin gamla kopia lokalt.
	\item Aktörer
	\\En användare av systemet, person A, utan större datorvana.
	\item Förhandsvillkor
	\\Person A är online och har dwiki installerat på sin dator.
	\item Händelseförlopp
	\begin{enumerate}
		\item Person A startar dwiki.
		\item dwiki synkroniserar ändringar i artiklar med andra medlemmar som är online i wikigruppen.
		\item Person A läser en av de uppdaterade artiklarna och ogillar de nya ändringarna.
		\item Person A väljer att återskapa en äldre version av artikeln och ett fönster med möjliga äldre versioner visas.
		\item Person A väljer den version av artikeln han vill ha och artikeln återskapas lokalt.
	\end{enumerate}	
	\item Huvudscenario
	\\Den viktigaste delen i detta användarfall är att person A kan välja vilken version han vill gå tillbaks till.
	\item Efterhandsvillkor	
	\begin{itemize}
		\item Lyckades med att återskapa en artikel
		\\Användaren lyckades att återskapa en gammal version av en artikel.
		\item Lyckades inte att återskapa en artikel.
		\\Något gick fel på vägen och artikeln återskapades inte. Den nuvarande versionen finns kvar som innan.
	\end{itemize}
	\item Speciella krav
	\\Användaren skall kunna vara medlem i en speciell wikigrupp.
	\\Användaren skall ha läsrättigheter i wikigruppen(behövs endast för att ta emot nya uppdateringar).
	\\Användaren har ändringar av den aktuella artikeln sparade lokalt.
\end{itemize}


\subsection{Sammanslagning}
\begin{itemize}
	\item Kortfattad beskrivning
	\\En användare ska sammanfoga sina dokument eller annat arbete med andra personers arbeten.
	\item Aktörer
	\\Tre användare A, B och C som inte har någon större datorvana.
	\item Förhandsvillkor
	\\Ett dwiki system är konfigurerat, satt igång och används.
	\item Händelseförlopp
	\begin{enumerate}
		\item Person A startar dwiki och får in uppdateringar av samma artikel från Person B och person C.
		\item uppdateringarna kolliderar och Användare A får nu välja om han vill behålla kopian från Person B, kopian från person C, sin egen lokala kopia eller slå samman artiklarna manuellt.
		\item Person A väljer att behålla kopian från Person C som sparas ned lokalt på hans dator.
	\end{enumerate}
	\item Alternativt händelseförlopp
	\begin{enumerate}	
		\item Person A startar dwiki och får in uppdateringar av samma artikel från Person B och person C.
		\item uppdateringarna kolliderar och Användare A får nu välja om han vill behålla kopian från Person B, kopian från person C, sin egen lokala kopia eller slå samman artiklarna manuellt.
		\item Person A väljer att slå samman artiklarna manuellt.
		\item Ett redigeringsfönster dyker nu upp där ändringarna från Person B och Person C är tydligt utmarkerade. Person A kan nu redigera hela artikeln och spara det han vill ha kvar.
		\item När person A är färdig väljer han att spara ner den redigerade kopian lokalt.
	\end{enumerate}	
	\item Huvudscenario
	
	Det viktiga i detta användarfall är att visa att man kan lösa konflikter i sammanslagning antingen manuellt eller välja att spara en kopia rakt av.
	\item Efterhandsvillkor
		\begin{itemize}	
		\item Lyckad sammanfogning
		\\Användaren lyckades med att sammanfoga (merge) två artiklar.
		\item Icke lyckad sammanfogning
		\\Användaren lyckades ej med en sammanfogning, detta kan bero på att användaren inte var bekväm med att välja vad han vill ha med i sin artikel och inte. Användaren kan då överlåta beslutet till dem som har skrivit ändringarna.
		\end{itemize}
	\item Speciella krav
	\\Användaren ska vara medlem i en dwiki-grupp.
\end{itemize}

\subsection{Rättstavning}
\begin{itemize}
	\item Kortfattad beskrivning
	\\En användare orkar eller vill inte sitta och korrekturläsa hela artikeln för att hitta stavfel utan vill använda en automatisk rättstavare.
	\item Aktörer
	\\En användare, person A, utan direkt datorvana.
	\item Förhandvillkor
	\\Person A har dwiki installerat på datorn
	\item Händelseförlopp
	\begin{enumerate}
		\item En användare vill eller har inte tid att läsa en artikel som han vill göra tillgängligt för andra användare.
		\item Användaren redigerar artikeln.
		\item Användaren väljer sedan att rättstava artikeln.
		\item För varje fel som rättstavningskontrollen hittar får användaren en fråga om han vill ändra det befintliga ordet till det som rättstavningskontrollen föreslår.
		\item När användaren är klar sparar han artikeln.
	\end{enumerate}
	\item Huvudscenario
	\\Det viktiga delen i det här användarfall är att en användare skall kunna använda automatisk rättstavning för att granska sin artikel efter stavfel.
	\item Efterhandskrav
	\begin{itemize}
	\item Inga fel
	\\Rättstavningskontrollen hittar inga fel och godkänner artikeln direkt.
	\item Fel hittades
	\\Rättstavningskontrollen hittar något som den uppfattar som felstavat och användaren får en rättningsförfrågan.
	\end{itemize}	
	\item Speciella krav
	\\Användaren måste ha stavningskontrollen installerad.
\end{itemize}
\end{document}
